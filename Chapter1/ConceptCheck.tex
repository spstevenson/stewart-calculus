\documentclass{article}

\usepackage{amsmath}

\renewcommand{\thesection}{} % remove section nmbers
\begin{document}
	\section{Concept Check}

	\begin{enumerate}

		% Q1
		\item
		\begin{enumerate}
			\item What is a function? What are its domain and range?

			A function maps a set of inputs to a set of outputs. 

			A function is a rule that assigns each element $x$ in a
			set $D$ exactly one element, called $f(x)$ in a set $E$.

			The domain is the set $D$ of input elements of the function. The
			range is the set $E$ of output values.

			\item What is the graph of a function?

			A graph is a graphical representation of a function that shows
			the domain (usually on the $x$-axis) to the output values in
			the range $y$-axis via a curve.

			\item How can you tell whether a given curve is the graph of
			a function?

			A curve is a graph of the function if it passes the vertical line
			test. If any vertical passes through two or more points, the curve
			is not the graph of a function.
		\end{enumerate}

		% Q2 TODO
		\item Discuss four ways of representing a function. Illustrate your
			discussion with examples.

		% Q3
		\item
		\begin{enumerate}
			\item What is an even function? How can you tell if a function
				is even by looking at its graph? Give three examples
				of an even function?

				An even function is one where $f(x) = f(-x)$. An even
				function id symmetrical around the $y$-axis. Three
				even functions are $f(x) = x^2$, $g(x) = \cos x$ and
				$h(x) = |x|$.

			\item What is an odd function? How can you tell a function is
				odd by looking at its graph? Give examples of an odd
				function?

				An odd function is one where $f(x) = -f(-x)$. An odd
				function is symmetrical about the origin. Three
				odd functions are $f(x) = x^3$, $g(x) = \sin x$ and
				$h(x) = x$.
		\end{enumerate}

		% Q4
		\item What is an increasing function?
		
		An increasing function is one where $f(x_1) < f(x_2)$ whenever $x_1 < x_2$.

		% Q5
		\item What is a mathematical model?

		A mathematical model is a set of equations or mathematical expressions that
		describe a real world phenomena or problem. We can derive mathematical 
		conclusions from mathematical models.

		% Q6
		\item Give an example of each type of function.
		\begin{enumerate}
			\item Linear function

				$$f(x) = x + 3$$

			\item Power function

				$$f(x) = x^3$$

			\item Exponential function

				$$f(x) = e^x$$

			\item Quadratic function

				$$f(x) = x^2 + x + 1$$

			\item Polynomial of degree 5

				$$f(x) = x^5 + x^4 + x^3 + x^2 + x + 1$$

			\item Rational function

				$$f(x) = \frac{x +1}{x^2 + 2x  +1}$$
		\end{enumerate}	

		% Q7 TODO
		\item Sketch by hand, on the same axes, the graphs of the following
			functions.

		\begin{enumerate}
			\item $f(x) = x$

			\item $g(x) = x^2$

			\item $h(x) = x^3$

			\item $j(x) = x^4$
		\end{enumerate}

		% Q8 TODO
		\item Draw, by hand, a rough sketch of the graph of each function.

		\begin{enumerate}
			\item $y = \sin x$

			\item $y = \tan x$

			\item $y = 2^x$

			\item $y = 1/x$

			\item $y = |x|$

			\item $y = \sqrt{x}$
		\end{enumerate}

		% Q9
		\item Suppose that $f$ has domain $A$ and $g$ has domain $B$.

		\begin{enumerate}
			\item What is the domain of $f + g$?

			The domain of $f + g$ is $A \cap B$


			\item What is the domain of $fg$?

			The domain of $fg$ is $A \cap B$

			\item What is the domain of $f/g$

			The domain of $f/g$ is $A \cap B, g \ne 0$ 
			
		\end{enumerate}

		% Q10
		\item How is the composite function $f \circ g$ defined? What is
			its domain?

		A composite function is where the output of one function is used as
		the input for another function.

		$$f \circ g = f(g(x))$$

		% Q11
		\item Suppose the graph of $f$ is given. Write an equation for each
		of the graphs that are obtained from the graph of $f$ as follows:

		\begin{enumerate}
			\item Shift 2 units upwards

				$f(x) + 2$

			\item Shift 2 units downwards

				$$f(x) - 2$$

			\item Shift 2 units to the right

				$$f(x - 2)$$

			\item Shift 2 units to the left

				$$f(x + 2)$$

			\item Reflect about the $x$-axis

				$$-f(x)$$

			\item Reflect about the $y$-axis

				$$f(-x)$$

			\item Stretch vertically by a factor of 2.

				$$2f(x)$$

			\item Shrink vertically by a factor of 2.

				$$\frac{1}{2}f(x)$$

			\item Stretch horizontally by a factor of 2.

				$$f(x/2)$$

			\item Shrink horizontally by a factor of 2.

				$$f(2x)$$
		\end{enumerate}

		% Q12 TODO
		\item Explain what each of the following means and illustrate with
			a sketch.

		\begin{enumerate}
			\item $\lim \limits _{x \to a} f(x) = L$

			\item $\lim \limits _{x \to a^{+}} f(x) = L$

			\item $\lim \limits _{x \to a^{-}} f(x) = L$

			\item $\lim \limits _{x \to a} f(x) = \infty$

			\item $\lim \limits _{x \to \infty} f(x) = L$
		\end{enumerate}

		% Q13 TODO
		\item Describe several ways in which a limit can fail to exist.
			Illustrate with sketches.

		% Q14
		\item State the following Limit Laws.

		\begin{enumerate}
			\item Sum Law

			The limit of the sum is the sum of the limits (provided
			the limits exist.)

			$$\lim \limits _{x \to a} f(x) + g(x) = \lim \limits _{x \to a} f(x) + \lim \limits _{x \to a} g(x)$$

			\item Difference Law

			The limit of the difference is the difference of the limits (provided
			the limits exist.)

			$$\lim \limits _{x \to a} f(x) - g(x) = \lim \limits _{x \to a} f(x) - \lim \limits _{x \to a} g(x)$$

			\item Constant Multiple Law

			$$\lim \limits _{x \to a} c f(x) = c \lim \limits _{x \to a} f(x)$$

			\item Product Law

			The limit of the product is the product of the limits (provided the limits exist)

			$$\lim \limits _{x \to a} f(x) g(x) = \lim \limits _{x \to a} f(x) \cdot \lim \limits _{x \to a} g(x)$$

			\item Quotient Law

			The limit of the quotient is the quotient of the limits (provided the limits exists and
			the denominator is non-zero)

			$$\lim \limits _{x \to a} \frac{f(x}{g(x)} = \frac{\lim \limits _{x \to a} f(x)}{\lim \limits _{x \to a} g(x)}$$

			\item Power Law

			$$ \lim \limits _{x \to a} f(x)^{n} = [\lim \limits _{x \to a} f(x)]^{n}$$

			\item Root Law

			$$\lim \limits _{x \to a} \sqrt[n]{f(x)} = \sqrt[n]{\lim \limits _{x \to a} f(x)}$$

		\end{enumerate}

		% Q15
		\item What does the Squeeze Theorem say?

		The squeeze theorem says, if

		$h(x) \leq f(x) \leq g(x)$ when $x$ is near $a$

		and 

		$$\lim \limits _{x \to a} h(x) = \lim \limits _{x \to a} g(x) = l$$

		then

		$$\lim \limits _{x \to a} f(x) = L$$

		% Q16
		\item
		\begin{enumerate}
			\item What does it mean for $f$ to be continuous at $a$?

			A function $f$ is continuous at a number $a$ if

			$$\lim \limits _{x \to a} f(x) = f(a)$$

			\item What does it mean for $f$ to be continuous on the
			interval $(-\infty, \infty$? What can you say about the
			graph of such a function?

			A function is continuous on an interval if it is continuous
			at every point along the interval. The graph will be smooth
			with no breaks or discontinuities.
		\end{enumerate}

		% Q17
		\item What does the Intermediate Value Theorem say?

		The Intermediate Value Therem says the for a function continuous
		along an interval $[a, b]$

		If $f(a) < N < f(b)$ there exists a number $c$ such that $f(c) = N$.

		% Q18 TODO

		\item
		\begin{enumerate}
			\item What does it mean to say that the line $x = a$ is a 
			vertical asymptote of the curve $y = f(x)$? Draw curves
			to illustrate the various possiblities.

			\item What does it mean to say that the line $y = L$ is a
			horizontal asymptote of the curve $y = f(x)$? Draw curves
			to illustrate the various possibilities.
		\end{enumerate}
	\end{enumerate}
\end{document}
