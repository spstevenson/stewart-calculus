\documentclass{article}

\renewcommand{\thesection}{} % Remove the section number

\usepackage{amsmath}

\begin{document}

\section{TRUE-FALSE QUIZ}

Determine whether the statement is true or false. If it is true, explain
why. If it is false, explain why or give an example that disproves the
statement.

\begin{enumerate}
	% Q1
	\item If $f$ is a function. then $f(s + t) = f(s) + f(t)$.

	Lets say $f(s) = 2s + 1$

	$s = 2$
	$t = 3$

	$f(2 + 3) = 2(2 + 3) + 1 = 2(5) + 1 = 10 + 1 = 11$

	$f(2) + f(3) = 2(2) + 1 + 2(3) + 1 = 4 + 1 + 6 + 1 = 12$

	So this statement is false.

	% Q2
	\item If $f(s) = f(t)$, then $s = t$.

	Lets say $f(x) = x^2$

	$s = 2, t = -2$

	$$f(s) = 4, f(t) = 4, s \ne t$$

	So this statement is false.

	% Q3
	\item If $f$ is a function, then $f(3x) = 3f(x)$.

	$$f(x) = x^2$$

	$$f(3x) = (3x)^2 = 9x^2$$

	$$3f(x) = 3x^2$$

	$$9x^2 \ne 3x^2$$

	So this statement is false.

	%Q4
	\item If $x_1 < x_2$ and $f$ is a decreasing function, then
		$f(x_1) > f(x_2)$

		This is true.

	% Q5
	\item A vertical line intersects the graph of a function only once.

		This is true. If the line intersected multiple times then
		there would be multiple output values for the input value
		at that point which would break the definition of a function.

	% Q6
	\item If $f$ and $g$ are functions, then $f \cdot g = g \cdot f$.

		This is not true. If $f(x) = 2x + 3$ and $g(x) = 3x + 1$.

		$$f(g(x)) = 2(3x + 1) + 3 = 6x + 5$$

		$$g(f(x)) = 3(2x + 3) + 1 = 6x + 10$$

	% Q7
	\item $\lim \limits _{x \to 4} (\frac{2x}{x-4} - \frac{8}{x-4}) =
		\lim \limits _{x \to 4} \frac{2x}{x-4} - \lim \limits _{x \to 4} \frac{8}{x-4}$

		This is true. The limit of the difference is the difference of the limits where
		limits exist according to the 2nd limit law.

	% Q8
	\item $\lim \limits _{x \to 1} \frac{x^2 + 6x -7}{x^2 + 5x - 6} = 
		\frac{\lim \limits _{x \to 1} (x^2 + 6x -7)}{\lim \limits _{x \to 1} (x^2 + 5x - 6)}$

		The limit of the quotient is the quotient of the limits
		provided the limit of the denominator is not zero.

		$$\lim \limits _{x \to 1} x^2 + 5x - 6 = 0$$

		As the limit of the denominator is zero this statement is false.

	% Q9
	\item $\lim \limits _{x \to 1} \frac{x-3}{x^2 + 2x - 4} = 
		\frac{\lim \limits _{x \to 1} (x-3)}{\lim\limits _{x \to 1} (x^2 + 2x - 4)}$

		This is true. In this case $\lim \limits _{x \to 1} (x^2 + 2x - 4) \ne 0$ so
		the limit of the quotient is the quotient of the limits.

	% Q10
	\item If $\lim \limits _{x \to 5} f(x) = 2$ and $\lim \limits _{x \to 5} g(x) = 0$,
		then $\lim _{x \to 5} [f(x)/g(x)]$ does not exist.

		This is false. The limit could still exist.

		E.g. $f(x) = x - 3$ and $g(x) = x^2 - 3x - 10$

		$$\lim \limits _{x \to 5} \frac{x-3}{x^2 - 3x - 10}$$

		$$\lim \limits _{x \to 5} \frac{(x - 5) + 2}{(x-5)(x+2)}$$
		$$\lim \limits _{x \to 5} \frac{2}{x+2}$$

		$$\frac{\lim \limits _{x \to 5} 2}{\lim \limits _{x \to 5} x + 2}$$

		$$\frac{2}{7}$$

	% Q11
	\item If $\lim _{x \to 5} f(x) = 0$ and $\lim _{x \to 5} g(x) = 0$, then
		$\lim _{x \to 5} [f(x)/g(x)]$ does not exist.

		This is false. Like before...

		E.g. $f(x) = x - 5$ and $g(x) = (x-5)(x+3)$

		$$\lim \limits _{x \to 5} \frac{x-5}{(x-5)(x+3)}$$

		$$\lim \limits _{x \to 5} \frac{0}{x+3}$$

		$$\frac{ \lim \limits _{x \to 5} 0}{\lim \limits _{x \to 5} x + 3}$$

		$$\frac{0}{8} = 0$$

	% Q12
	\item If $\lim _{x \to 6} [f(x)g(x)]$ exists, then the limit must be $f(6)g(6)$

		This is false.

	% Q13
	\item If $p$ is a polynomial, then $\lim _{x \to b} p(x) = p(b)$

		This is true. Polynomials are continuous functions. This means

		$$\lim \limits _{x \to a} f(x) = p(a)$$

	% Q14
	\item If $\lim \limits _{x \to 0} f(x) = \infty$ and $\lim \limits _{x \to 0} g(x) = \infty$
		then $\lim \limits _{x \to 0} [f(x) - g(x)] = 0$.

		This is false.

		If $f(x) = \frac{1}{x}$ and $g(x) = \frac{1}{x^2}$

		$$\lim \limits _{x \to 0} [\frac{1}{x} - \frac{1}{x^2}]$$

		$$\lim \limits _{x \to 0} \frac{1}{x^2} = \infty$$
\end{enumerate}

\end{document}
