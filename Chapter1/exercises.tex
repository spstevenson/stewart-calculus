\documentclass{article}

\renewcommand{\thesection}{}

\usepackage{amsmath}

\begin{document}

\section{Exercises}

\begin{enumerate}

% Q1
\item Let $f$ be the function whose graph is given.

	\begin{enumerate}
		\item Estimate the value of $f(2)$

			$$f(2) = 1.5$$

		\item Estimate the values of $x$ such that $f(x) = 3$.

			$$f(2.2) = 3$$

			$$f(5.5) = 3$$

		\item State the domain of $f$.

			The domain of $f$ is $[-6, 6]$.

		\item State the range of $f$.

			The range of $f$ is $[-4, 4]$

		\item On what interval is $f$ increasing?

			$f$ is increasing over $[-4, 4]$

		\item If $f$ even, odd or neither even nor odd? Explain.

			$f$ is odd as it is symmetrical around the origin.

			This means that $f(-x) = -f(x)$
	\end{enumerate}

	% Q2
	\item Determine whether each curve is the graph of a function of $x$.
		If it is, state the domain and range of the function.

	\begin{enumerate}
		\item

		This is not a function as it does not pass the vertical line test.

		\item

		This is a function.

		The domain is $[-3, 3]$. The range is $[-2, 3]$.
	\end{enumerate}

	\item[3-6] Find the domain and range of the function. Write your answer
		in interval notation.

	% Q3
	\item $f(x) = 2/(3x - 1)$

		Domain is $(-\infty , \infty)$.

		Range is $(-\infty, 0) \cup (0, \infty)$

	% Q4
	\item $g(x) = \sqrt{16 - x^4}$

		Domain is $[-2, 2]$, the range is $[0, \infty)$

	% Q5
	\item $y = 1 + \sin x$

		The domain is $(-\infty, \infty)$ the range is $[0, 2]$

	% Q6
	\item $y = \tan 2x$

		The domain is $[x : x \text{ is a real number } \ne k\pi \text{ for all integers } k]$.

		The range is $[-\infty, \infty]$

	% Q7
	\item Suppose that the graph of $f$ is given. Describe how the graphs of the 
		following functions can be obtained from the graph of $f$.

	\begin{enumerate}
		\item $y = f(x) + 8$

			The graph of $f$ is moved upwards by 8.

		\item $y = f(x + 8)$

			The graph is shifted by 8 to the left.

		\item $y = 1 + 2f(x)$

			The graph of $f$ is stretched vertically by a factor of
			2 and moved upwards by 1.

		\item $y = f(x - 2) - 2$

			The graph of $f$ is shifted to the right by 2 and 
			downwards by 2.

		\item $y = -f(x)$

			The graph is flipped around the $x$-axis.

		\item $y = 3 - f(x)$

			The graph is flipped around the $x$-axis and upwards
			by 3.

	\end{enumerate}

	% Q8 TODO
	\item The graph of $f$ is given. Draw the graphs of the following functions.

	% Q 9-14 TODO
	\item[9-14] Use transformations to sketch the graph of the function.

	% Q15
	\item Determine whether $f$ is even, odd, or neither even nor odd.

	\begin{enumerate}
		\item $f(x) = 2x^5 - 3x^2 + 2$

		$$f(-x) = 2(-x)^5 - 3(-x)^2 + 2$$

		$$f(-x) = -2x^5 - 3x^2 + 2$$

		So $f(x)$ is neither even nor odd.

		\item $f(x) =  x^3 - x^7$

		$$f(-x) = (-x)^3 - (-x)^7$$

		$$f(-x) = -x^3 + x^7$$

		$$-x^3 + x^7 = -(x^3 - x^7)$$

		So this function is odd.

		\item $f(x) = \cos(x^2)$

		$$f(-x) = \cos((-x)^2)$$

		$$f(-x) = \cos(x^2)$$

		$$f(-x) = f(x)$$

		So this function is even.

		\item $f(x) = 1 + \sin x$

		$$f(-x) = 1 + \sin(-x)$$

		$$f(-x) = 1 - \sin x$$

		This function is neither even nor odd.
	\end{enumerate}

	% Q16
	\item Find an expression for the function whose graph consists of the
		line segment from the point $(-2,2)$ to the point $(-1, 0)$
		together with the top half of the circle with center the origin
		and radius 1.

		The straight line will have equation

		$$y = -2x - 1$$

		And the circle has equation

		$$x^2 + y^2 = r^2$$

		Because $r = 1$

		$$y^2 = 1 - x^2$$

		The upper half of the circle will be

		$$y = \sqrt{1 - x^2}$$

		So using a piecewise function we get...

		$$f(x) = \begin{cases}
			-2x - 1 & \text{ if } -2 \leq x \leq -1 \\
			\sqrt{1 - x^2} & \text{ if } -1 < x \leq 1
		\end{cases}
		$$
\end{enumerate}

\end{document}
