\documentclass{article}

\renewcommand{\thesection}{}

\usepackage{amsmath}
\usepackage{amssymb}

\begin{document}

\section{Exercises}

\begin{enumerate}

% Q1
\item Let $f$ be the function whose graph is given.

	\begin{enumerate}
		\item Estimate the value of $f(2)$

			$$f(2) = 1.5$$

		\item Estimate the values of $x$ such that $f(x) = 3$.

			$$f(2.2) = 3$$

			$$f(5.5) = 3$$

		\item State the domain of $f$.

			The domain of $f$ is $[-6, 6]$.

		\item State the range of $f$.

			The range of $f$ is $[-4, 4]$

		\item On what interval is $f$ increasing?

			$f$ is increasing over $[-4, 4]$

		\item If $f$ even, odd or neither even nor odd? Explain.

			$f$ is odd as it is symmetrical around the origin.

			This means that $f(-x) = -f(x)$
	\end{enumerate}

	% Q2
	\item Determine whether each curve is the graph of a function of $x$.
		If it is, state the domain and range of the function.

	\begin{enumerate}
		\item

		This is not a function as it does not pass the vertical line test.

		\item

		This is a function.

		The domain is $[-3, 3]$. The range is $[-2, 3]$.
	\end{enumerate}

	\item[3-6] Find the domain and range of the function. Write your answer
		in interval notation.

	% Q3
	\item $f(x) = 2/(3x - 1)$

		Domain is $(-\infty , \infty)$.

		Range is $(-\infty, 0) \cup (0, \infty)$

	% Q4
	\item $g(x) = \sqrt{16 - x^4}$

		Domain is $[-2, 2]$, the range is $[0, \infty)$

	% Q5
	\item $y = 1 + \sin x$

		The domain is $(-\infty, \infty)$ the range is $[0, 2]$

	% Q6
	\item $y = \tan 2x$

		The domain is $[x : x \text{ is a real number } \ne k\pi \text{ for all integers } k]$.

		The range is $[-\infty, \infty]$

	% Q7
	\item Suppose that the graph of $f$ is given. Describe how the graphs of the 
		following functions can be obtained from the graph of $f$.

	\begin{enumerate}
		\item $y = f(x) + 8$

			The graph of $f$ is moved upwards by 8.

		\item $y = f(x + 8)$

			The graph is shifted by 8 to the left.

		\item $y = 1 + 2f(x)$

			The graph of $f$ is stretched vertically by a factor of
			2 and moved upwards by 1.

		\item $y = f(x - 2) - 2$

			The graph of $f$ is shifted to the right by 2 and 
			downwards by 2.

		\item $y = -f(x)$

			The graph is flipped around the $x$-axis.

		\item $y = 3 - f(x)$

			The graph is flipped around the $x$-axis and upwards
			by 3.

	\end{enumerate}

	% Q8 TODO
	\item The graph of $f$ is given. Draw the graphs of the following functions.

	% Q 9-14 TODO
	\item[9-14] Use transformations to sketch the graph of the function.

	% Q15
	\item Determine whether $f$ is even, odd, or neither even nor odd.

	\begin{enumerate}
		\item $f(x) = 2x^5 - 3x^2 + 2$

		$$f(-x) = 2(-x)^5 - 3(-x)^2 + 2$$

		$$f(-x) = -2x^5 - 3x^2 + 2$$

		So $f(x)$ is neither even nor odd.

		\item $f(x) =  x^3 - x^7$

		$$f(-x) = (-x)^3 - (-x)^7$$

		$$f(-x) = -x^3 + x^7$$

		$$-x^3 + x^7 = -(x^3 - x^7)$$

		So this function is odd.

		\item $f(x) = \cos(x^2)$

		$$f(-x) = \cos((-x)^2)$$

		$$f(-x) = \cos(x^2)$$

		$$f(-x) = f(x)$$

		So this function is even.

		\item $f(x) = 1 + \sin x$

		$$f(-x) = 1 + \sin(-x)$$

		$$f(-x) = 1 - \sin x$$

		This function is neither even nor odd.
	\end{enumerate}

	% Q16
	\item Find an expression for the function whose graph consists of the
		line segment from the point $(-2,2)$ to the point $(-1, 0)$
		together with the top half of the circle with center the origin
		and radius 1.

		The straight line will have equation

		$$y = -2x - 1$$

		And the circle has equation

		$$x^2 + y^2 = r^2$$

		Because $r = 1$

		$$y^2 = 1 - x^2$$

		The upper half of the circle will be

		$$y = \sqrt{1 - x^2}$$

		So using a piecewise function we get...

		$$f(x) = \begin{cases}
			-2x - 1 & \text{ if } -2 \leq x \leq -1 \\
			\sqrt{1 - x^2} & \text{ if } -1 < x \leq 1
		\end{cases}
		$$

	% Q17
	\item If $f(x) = \sqrt{x}$ and $g(x) = \sin x$, find the functions (a)
		$f \circ g$, (b) $g \circ f$, (c) $f \circ f$, (d) $g \circ g$,
		and their domains.

		\begin{enumerate}
			\item $f \circ g$

				$$\sqrt{\sin x}$$

				The domain of this is 

				$${x \in \mathbb{R} | 2k\pi \leq x \leq 2k\pi + \pi, k \in \mathbb{Z}}$$	

			\item $g \circ f$

				The function is...

				$$\sin (\sqrt{x})$$

				And the domain is...

				$$[0, \infty)$$

			\item $f \circ f$

				The function is...

				$$\sqrt{\sqrt{x}} = x^{\frac{1}{4}} = \sqrt[4]{x}$$

				The domain is...

				$$[0, \infty)$$

			\item $g \circ g$

				The function is...

				$$\sin (\sin x)$$

				The domain is...

				$$(-\infty, \infty)$$
		\end{enumerate}

		% Q 17
		\item Express the function $F(x) = 1/\sqrt{x + \sqrt{x}}$ as a composition
			of three functions.

			$$a(x) = \frac{1}{x}$$
			$$b(x) = \sqrt{x}$$
			$$c(x) = x + \sqrt{x}$$

			And the composition is $a \circ b \circ c$

		% Q 19 TODO

		% Q20 TODO

		\item[21-36] Find the limit.

		% Q21
		\item $\lim \limits _{x \to 0} \cos(x + \sin x)$

			This can be solved using the direct substitution principle.

			$$\cos(0 + \sin 0) = \cos(0 + 0) = 1$$

		% Q22
		\item $\lim \limits _{x \to 3} \frac{x^2 - 9}{x^2 + 2x - 3}$

			This can also be solved with the direct substitution principle.

			$$\frac{3^2 - 9}{3^2 + 2(3) - 3} = \frac{9 - 9}{9 + 6 - 3} = \frac{0}{12}$$

		% Q23
		\item $\lim \limits _{x \to -3} \frac{x^2 - 9}{x^2 + 2x - 3}$

			Because the denominator goes to zero, we must find an equivalent function.

			$$\frac{(x+3)(x-1)}{(x+3)(x-3)}$$

			$$\frac{x-2}{x-3}$$

			The limit of this equivalent function is $\frac{-3-2}{-3-3} = \frac{-5}{-6} = \frac{5}{6}$

		% Q24
		\item $\lim \limits _{x \to 1^{+}} \frac{x^2 - 9}{x^2 + 2x - 3}$

			First the simplified function is...

			$$\frac{x-3}{x-1}$$

			In the denominator, as $x$ grows closer to 1 from the positive side
			the denominator becomes a very small positive number. Therefore
			the limit of the function is $\infty$.

		% Q25
		\item $\lim \limits _{h \to 0} \frac{(h-1)^3 + 1}{h}$

			First lets expand out the numerator...

			$$(h-1)^3 = h^3 - 3h^2 +3h - 1$$

			So the exanded out numerator is...

			$$\frac{h^3 - 3h^2 + 3h}{h}$$

			We can cancel out the denominator to get an equivalent function

			$$\lim \limits _{h \to 0} h^2 - 3h + 3 = 3$$

		% Q26
		\item $\lim \limits _{t \to 2} \frac{t^2 - 4}{t^3 - 8}$

			First lets simplify the function

			$$t^2 - 4 = (t + 2)(t-2)$$

			$$t^3 - 8 = (t-2)(t^2 + 2t + 4)$$

			And we have...

			$$\frac{(t+2)(t-2)}{(t-2)(t^2 + 2t + 4)}$$

			After cancelling...

			$$\frac{t+2}{t^2+2t+4)}$$

			The we can use Direct Substitution with the simplified function

			$$\lim \limits _{t \to 2} \frac{t+2}{t^2+2t+4} = \frac{1}{3}$$	

		% Q27
		\item $\lim \limits _{r \to 9} \frac{\sqrt{r}}{(r-9)^4}$

		As $r$ goes to 9 the numerator will go to 3 whilst the denominator will become
		very small therefore

		$$\lim \limits _{r \to 9} \frac{\sqrt{r}}{(r-9)^4} = \infty$$

		% Q28
		\item $\lim \limits _{v \to 4^{+}} \frac{4 - v}{|4 - v|}$
		
		When $v > 4$ then this function is equal to

		$$\frac{4-v}{-(4-v)} = -1$$

		So

		$$\lim \limits _{v \to 4^{+}} \frac{4-v}{|4-v|} = -1$$

		% Q29
		\item $\lim \limits _{s \to 16} \frac{4 - \sqrt{s}}{s - 16}$

			First we rationalize the denominator...

			$$\frac{4-\sqrt{s}}{s-16} \times \frac{4+\sqrt{s}}{4+\sqrt{s}}$$

			$$\frac{16-s}{(s-16)(4+\sqrt{s}}$$

			$$-\frac{1}{\sqrt{s} + 4}$$

			$$\lim \limits _{s \to 16} -\frac{1}{\sqrt{s} + 4} = -\frac{1}{8}$$

		% Q30
		\item $\lim \limits _{v \to 2} \frac{v^2 + 2v -8}{v^4 - 16}$

			First lets factorise top and bottom.

			$$\frac{(v+4)(v-2)}{(v+4)(v-4)}$$

			So we can simplify...

			$$\frac{v-2}{v-4}$$

			$$\lim \limits _{v \to 2} \frac{v-2}{v-4} = 0$$

		% Q31 
		\item $\lim \limits _{x \to \infty} \frac{1 + 2x - x^2}{1-x+2x^2}$

		So we multiply by $\frac{\frac{1}{x^2}}{\frac{1}{x^2}}$

		And we get...

		$$\frac{ \frac{1}{x^2} + \frac{2x}{x^2} - \frac{x^2}{x^2}}{\frac{1}{x^2} - \frac{x}{x^2} + \frac{2x^2}{x^2}}$$

		This simplfies to

		$$\frac{ \frac{1}{x^2} + \frac{1}{x} - 1}{\frac{1}{x^2} - \frac{1}{x} + 2}$$

		So as $x \to \infty$ then this goes to $-\frac{1}{2}$

		% Q32
		\item $\lim \limits _{x \to -\infty} \frac{1 - 2x^2 - x^4}{5 + x - 3x^4}$

			First we multiply by $\frac{ \frac{1}{x^4} }{ \frac{1}{x^4} }$

			And we get...

			$$\frac{ \frac{1}{x^4} - \frac{2x^2}{x^4} - \frac{x^4}{x^4} }{\frac{5}{x^4} + \frac{x}{x^4} - \frac{3x^4}{x^4} }$$

			Which we simplify...

			$$\frac{ \frac{1}{x^4} - \frac{2}{x^2} - 1}{ \frac{5}{x^4} + \frac{1}{x^3} - 3}$$

			As $x \to -\infty$ this equals $\frac{1}{3}$.

		% Q33
		\item $\lim \limits _{x \to \infty} (\sqrt{x^2 + 4x + 1} - x)$

			First let's rationalize the expression.

			$$(\sqrt{x^2 + 4x + 1} - x) \times \frac{\sqrt{x^2 + 4x + 1} + x}{\sqrt{x^2 + 4x + 1} + x}$$

			$$ = \frac{4x + 1}{\sqrt{x^2 + 4x + 1} + x}$$

			Now let's multiply by $\frac{\frac{1}{x}}{\frac{1}{x}}$

			$$ = \frac{ \frac{4x}{x} + \frac{1}{x} }{ \sqrt{ \frac{x^2}{x^2} + \frac{4x}{x^2} + \frac{1}{x^2}} + \frac{x}{x}}$$

			Which we can simplify...

			$$ = \frac{ 4 + \frac{1}{x} }{ \sqrt{1 + \frac{4}{x} + \frac{1}{x^2}} + 1}$$

			As $x \to \infty$ then this expression goes to $2$.

		% Q34
		\item $\lim \limits _{x \to 1} (\frac{1}{x-1} + \frac{1}{x^2 - 3x + 2})$

			First lets cross multiply the fractions...

			$$ = \frac{ (x^2 - 3x + 2) + (x - 1)}{(x-1)(x^2 - 3x + 2)}$$

			Fractionalize the denominator...

			$$ = \frac{(x-1)^2}{(x-1)(x^2 - 3x + 2)}$$

			$$ = \frac{x-1}{x^2 - 3x + 2}$$

			$$ = \frac{x-1}{(x-1)(x-2)}$$

			$$ = \frac{1}{x-2}$$

			$$\lim \limits _{x \to 1} \frac{1}{x-2} = -1$$

		% Q35
		\item $\lim \limits _{x \to 0} \frac{\cot 2x}{\csc x}$

		Let's use trig identities to figure out the limit.

		$$\frac{\cot 2x}{\csc x} = \frac{\sin x}{\tan 2x}$$

		Using $\tan u = \frac{\sin x}{\cos x}$

		$$\frac{\sin x}{\tan 2x} = \frac{\cos 2x \sin x}{\sin 2x}$$

		Using $\sin 2u = 2 \sin u \cos u$

		$$ = \frac{\cos 2x \sin x}{ 2 \sin x \cos x} = \frac{\cos 2x}{2 \cos x}$$

		$$\lim \limits _{x \to 0} \frac{\cos 2x}{2 \cos x} = \frac{1}{2}$$

		% 36
		\item $\lim \limits _{t \to 0} \frac{t^3}{\tan ^{3} 2t}$

			$$ [\frac{t}{\tan 2t}]^3$$

			$$ [\frac{t \cos 2t}{\sin 2t} ]^3 $$

			$$\frac{8}{8} \times [\frac{t \cos 2t}{\sin 2t} ]^3$$

			$$[\frac{2t \cos 2t}{2 \sin 2t}]^3$$

			$$[\frac{\cos 2t}{2} \times \frac{2t}{\sin 2t}]^3$$

			Using the third limit law

			$$[ \lim \limits _{t \to 0} \frac{\cos 2t}{2} \times \lim \limits _{t \to 0} \frac{2t}{\sin 2t}]^3$$

			Because $\lim \limits _{\theta \to a} \frac{\sin \theta}{\theta} = 1$

			$$[ \lim \limits _{t \to 0} \frac{\cos 2t}{2} ]^3 = \frac{1}{8}$$

			So

			$$\lim \limits _{t \to 0} \frac{t^3}{\tan ^{3} 2t} = \frac{1}{8}$$

		\item[37-38] Use graphs to discover the asymptotes of the curve. Then prove what you have discovered.

		% Q 37 TODO
		\item $y = \frac{\cos^2 x}{x^2}$

		% Q 38 TODO
		\item $y = \sqrt{x^2 + x + 1} - \sqrt{x^2 - x}$

		% Q 39 
		\item if $2x - 1 \leq f(x) \leq x^2$ for $0 < x < 3$, find $\lim _{x \to 1} f(x)$.

			Using the squeeze theorem.

			If $g(x) \leq f(x) \leq h(x)$

			and $\lim \limits _{x \to a} g(x) = \lim \limits _{x \to a} h(x) = L$

			then $\lim \limits _{x \to a} f(x) = L$

			$$\lim \limits _{x \to 1} 2x - 1 = 1$$

			$$\lim \limits _{x \to 1} x^2 = 1$$

			so $\lim \limits _{x \to 1} f(x) = 1$

		% Q 40
		\item Prove that $\lim _{x \to 0} x^2 \cos(1/x^2) = 0$.

			Using the squeeze theorem.

			$$-1 \leq \cos(\frac{1}{x^2}) \leq 1$$
			$$-x^2 \leq x^2 \cos(\frac{1}{x^2}) \leq x^2$$

			$$\lim \limits _{x \to 0} -x^2 = 0$$
			$$\lim \limits _{x \to 0} x^2 = 0$$

			so

			$$\lim \limits _{x \to 0} x^2 \cos(1/x^2) = 0$$

		\item[41-44] Prove the statement using the precise definition of a limit.

		% Q 41
		\item $\lim \limits _{x \to 2} (14 - 5x) = 4$

			If $|x - 2| < \delta$

			then we must find a number $\epsilon$ where $|(14-5x) -4| < \epsilon$

			$$|10 - 5x| < \epsilon$$
			$$|2 - x| < \frac{\epsilon}{5}$$

			$$|2-x| = |-(2-x)| = |x-2|$$

			So $|x - 2| < \frac{epsilon}{5}$

			$$\delta = \frac{epsilon}{5}$$

			Therefore if

			$$|x-2| < \delta$$

			Then

			$$5|x-2| < \epsilon$$

			So $|(14 - 5x) -4 < \epsilon$

			and $\lim \limits _{x \to 2} (14 - 5x) = 4$	

		% Q42
		\item $\lim \limits _{x \to 0} \sqrt[3]{x} = 0$

			If there is a positive $\epsilon$ then we need to find
			a number $\delta$ such that

			if $0 < |x| < \delta$ then $|\sqrt[3]{x}| < \epsilon$

			$$|\sqrt[3]{x}| = \sqrt[3]{|x|}$$ so

			$$|x| < \epsilon^3$$

			Therefore $ \delta = \epsilon^3$

			If $0 < |x| < \delta$ then $\sqrt[3]{|x|} < \epsilon$ so 

			$$\lim \limits _{x \to 0} \sqrt[3]{x} = 0$$

		% Q43
		\item $\lim \limits _{x \to \infty} \frac{1}{x^4} = 0$

			If $x > M$

			then we must find a value $\epsilon$ where $\frac{1}{x^4} < \epsilon$

			$$x^4 > \frac{1}{\epsilon}$$

			$$x > \frac{1}{\sqrt[4]{\epsilon}}$$

			So

			$$M = \frac{1}{\sqrt[4]{\epsilon}}$$

			So if $x > M = \frac{1}{\sqrt[4]{\epsilon}}$ then $\frac{1}{x^4} < \epsilon$

			So $\lim \limits _{x \to \infty} \frac{1}{x^4} = 0$

		% Q44
		\item $\lim \limits _{x \to 4^{+}} \frac{2}{\sqrt{x-4}} = \infty$

			So if $0 < x-4 < \delta$ then there must be some
			value $N$ where $\frac{2}{\sqrt{x-4}} > N$.

			$$frac{\sqrt{x-4}}{2} < \frac{1}{N}$$
			$$\sqrt{x-4} < \frac{2}{N}$$

			$$x - 4 < \frac{4}{N^2}$$

			So $\delta = \frac{4}{N^2}$

			Therefore if $0 < x-4 < \delta = \frac{4}{N^2}$ then $\frac{2}{\sqrt{x-4}} > N$

			So $\lim \limits _{x \to 4^{+}} = \infty$

		% Q45 TODO

		\item Show that each function is continuous on its domain. State the domain.

			\begin{enumerate}
				\item $g(x) = \frac{\sqrt{x^2 - 9}}{x^2 - 2}$

					If $f$ and $g$ are continiuous so is $\frac{f}{g}$.

					As $\sqrt{x^2 - 9}$ is a root function of a polynomial it is continuous.

					As $x^2 - 2$ is a polynomial function it is a continuous function.

					So $g(x)$ is continuous.

					The domain of $g(x)$ is $(-\infty, -3] \cup [3, \infty)$

				\item $h(x) = \sqrt[4]{x} + x^3 \cos x$

					If $\sqrt[4]{x}$ and $x^3 \cos x$ are continuous then $h(x)$ is continuous.

					$\sqrt[4]{x}$ is continuous as it is a root function.

					$x^3 \cos x$ is continuous if both $x^3$ and $\cos x$ are continuous

					$x^3$ is continuos as a cube function, $\cos x$ is continuous as a trigonometric function.

					The domain is $[0, \infty)$
			\end{enumerate}

		\item[47-48] Use the intermediate value theorem to show that there is a root of the equation in the given
				interval.

			% Q47
			\item $x^5 - x^3 + 3x - 5 = 0 (1,2)$

			If $f(x) = x^5 - x^3 + 3x - 5$

			$f(1) = -2$

			and

			$f(2) = 25$

			Because $f(1) = -2$ and $f(2) = 25$ there exists a number $N$ where $1 < N < 2$ and $f(N) = 0$.

			% Q48
			\item $2 \sin x = 3 - 2x, (0,1)$

				$f(x) = 2 \sin x + 2x - 3$

				$f(0) = -3$

				$f(1) = 0.6829...$

			Because $f(0) = -3$ and $f(1) = 0.6829$ there exists a number $N$ where $0 < N < 1$ and $f(N) = 0$
			
	\end{enumerate}
\end{document}
