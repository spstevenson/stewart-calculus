\documentclass{article}

\renewcommand{\thesection}{} % remove section
\renewcommand{\labelenumiii}{(\roman{enumiii})}


\usepackage{amsmath}

\begin{document}
	\section{1.3 Exercises}
	
	\begin{enumerate}
		\item If a ball is thrown into the air with a velocity of 10 m/s, its height in metres
			\emph{t} seconds later is given by $y = 10 - 4.9t^2$.
			
			\begin{enumerate}
				\item Find the average velocity for the time period beginning when $t = 1.5$ and
					lasting
				\begin{enumerate}
					\item 0.5 second
					
					Difference quotient:
					$$\frac{10(2) - 4.9(2)^2 - (10(1.5) - 4.9(1.5)^2)}{0.5}$$
					$$\frac{0.4 - 3.975}{0.5}$$
					$$\frac{-3.575}{0.5} = -7.15 \text{m/s}$$
					
					\item 0.1 second
					
					Difference quotient:
					$$\frac{10(1.6) - 4.9(1.6)^2 - (10(1.5) - 4.9(1.5)^2)}{0.1}$$
					$$\frac{3.456 - 3.975)}{0.1}$$
					$$\frac{-0.519}{0.1} = -5.19 \text{m/s}$$
					
					\item 0.05 second
					
					Difference quotient
					
					$$\frac{10(1.55) - 4.9(1.55)^2 -  (10(1.5) - 4.9(1.5)^2)}{0.05}$$
					$$\frac{ 3.72775 - 3.975}{0.05}$$
					$$\frac{ -0.24725}{0.05} = - 4.945 \text{m/s}$$
					
					\item 0.01 second
					
					Difference quotient
					
					$$\frac{10(1.51) - 4.9(1.51)^2 -  (10(1.5) - 4.9(1.5)^2)}{0.01}$$
					$$\frac{ 3.92751 - 3.975 }{0.01}$$
					$$\frac{ -0.04749}{0.01} = - 4.749\text{m/s}$$					
					
				\end{enumerate}
				\item Estimate the instantaneous velocity when $t = 1.5$.
				
					From the results from part (a) it looks like the instantaneous velocity is converging on -- 4.7 m/s.
				
				
			\end{enumerate}
			
		\item If an arrow is shot upward on the moon with a velocity of 58 m/s, its height in metres \emph{t} seconds
			later is given by $h = 58t - 0.83t^2$.
			
			\begin{enumerate}
			
				\item Find the average velocity over the given time intervals:
				
				\begin{enumerate}
					\item $[1, 2]$
					
						Difference quotient
						
						$$\frac{58(2) - 0.83(2)^2 - (58(1) - 0.83(1)^2)}{2 - 1}$$
						$$\frac{116 - 3.32 - (58 - 0.83)}{1}$$
						$$112.68 - 57.17 = 55.51$$
						
					\item $[1, 1.5]$
					
						Difference quotient
						
						$$\frac{58(1.5) - 0.83(1.5)^2 - (58(1) - 0.83(1)^2)}{1.5 - 1}$$
						$$\frac{87 - 0.83(2.25) - (58 - 0.83)}{0.5}$$
						$$\frac{87 - 1.8675 - 57.17}{0.5}$$
						$$\frac{27.9625}{0.5} = 55.925$$
						
					\item $[1, 1.1]$
					
						Difference quotient
						
						$$\frac{58(1.1) - 0.83(1.1)^2 - (58(1) - 0.83(1)^2)}{1.1 - 1}$$
						$$\frac{63.8 - 0.83(1.21) - (58 - 0.83)}{0.1}$$
						$$\frac{63.8 - 1.0043 - 57.17}{0.1}$$
						$$\frac{5.6257}{0.1} = 56.257$$
						
					\item $[1, 1.01]$
					
						Difference quotient
						
						$$\frac{58(1.01) - 0.83(1.01)^2 - (58(1) - 0.83(1)^2)}{1.01 - 1}$$
						$$\frac{58.58 - 0.83(1.0201) - (58 - 0.83)}{0.01}$$
						$$\frac{58.58 - 0.846683 - 57.17}{0.01}$$
						$$\frac{0.563317}{0.01} = 56.3317$$

					\item $[1, 1.001]$
					
						$$\frac{58(1.001) - 0.83(1.001)^2 -  (58(1) - 0.83(1)^2)}{1.001 - 1}$$
						$$\frac{58.058 - 0.83(1.002001) -  (58 - 0.83)}{0.001}$$
						$$\frac{58.058 - 0.83166083 - 57.17}{0.001}$$
						$$\frac{0.05633917}{0.001} = 56.33917$$
						
				\end{enumerate}
				
				\item Estimate the instantaneous velocity when $t = 1$.
				
					It looks as if the instantaneous velocity is converging on 56.4 m/s.
			\end{enumerate}
			
			\item For the function \emph{f} whose graph is given, state the value of each quantity, if it exists.
				If it does not exist, explain why.
				
				\begin{enumerate}
				
					\item $\lim\limits_{x\to 1} f(x)$
					
						The limit of $f(x)$ as \emph{x} goes to 1 is 2.
						
					\item $\lim\limits_{x \to 3^{-}} f(x)$
					
						The limit of $f(x)$ as \emph{x} goes to 3 from the left is 1.
						
					\item $\lim\limits_{x \to 3^{+}} f(x)$
					
						The limit of $f(x)$ as \emph{x} approaches 3 from the right is 4.
						
					\item $\lim\limits_{x \to 3} f(x)$
					
						The limit of $f(x)$ as \emph{x} does not exist as $\lim\limits_{x \to 3^{-}} f(x)$ is not equal
						to $\lim\limits_{x \to 3^{+}} f(x)$.
						
					\item $f(3)$
					
						$f(3) = 3$
				\end{enumerate}
				
			\item For the function \emph{f} whose graph is given, state the value of each quantity, if it exists. If it does not
				exist, explain why.
				
				\begin{enumerate}
				
					\item $\lim \limits_{x \to 0} f(x)$
					
						The limit of $f(x)$ as \emph{x} approaches 0 is 3.
						
					\item $\lim \limits_{x \to 3^{-}} f(x)$
					
						The limit of $f(x)$ as \emph{x} approaches 3 from the left is 4.
						
					\item $\lim \limits_{x \to 3^{+}} f(x)$
					
						The limit of $f(x)$ as \emph{x} approaches 3 from the right is 2.
						
					\item $\lim \limits_{x \to 3} f(x)$
					
						The limit of $f(x)$ as \emph{x} approaches 3 does not exist as 
						$\lim\limits_{x \to 3^{-}} f(x)$ is not equal to $\lim\limits_{x \to 3^{+}} f(x)$.
						
					\item $f(3)$
					
						$f(3) = 3$
						
				\end{enumerate}
				
			\item For the function \emph{g} whose graph is given, state the value of each quantity, if it exists. If it does
				not exist explain why.
				
				\begin{enumerate}
				
					\item $\lim \limits_{t \to 0^{-}} g(t)$
					
						The limit of $g(t)$ as \emph{t} approaches 0 from the left is -1.
						
					\item $\lim \limits_{t \to 0^{+}} g(t)$
					
						The limit of $g(t)$ as \emph{t} approaches 0 from the right -2.
						
					\item $\lim \limits_{t \to 0} g(t)$
					
						The limit of $g(t)$ as \emph{t} approaches 0 does not exist as 
						$\lim\limits_{t \to 0^{-}} g(t)$ is not equal to $\lim\limits_{t \to 0^{+}} g(t)$.
						
					\item $\lim \limits_{t \to 2^{-}} g(t)$
					
						The limit of $g(t)$ as \emph{t} approaches 2 from the left is 2.
						
					\item $\lim \limits_{t \to 2^{+}} g(t)$
					
						The limit of $g(t)$ as \emph{t} approaches 2 from the right is 0.
						
					\item $\lim \limits_{t \to 2} g(t)$
					
						The limit of $g(t)$ as \emph{t} approaches 2 does not exist as 
						$\lim\limits_{t \to 2^{-}} g(t)$ is not equal to $\lim\limits_{t \to 2^{+}} g(t)$.
						
					\item $g(2)$
					
						$g(2) = 1$
						
					\item $\lim \limits_{t \to 4} g(t)$
					
						The limit of $g(t)$ as \emph{t} approaches 4 is 3.
				\end{enumerate}
				

				\item Sketch the graph of the following function and use it to determine the values of
					\emph{a} for which $\lim_{x \to a} f(x)$ exists.
					
 \begin{displaymath}
   f(x) = \left\{
     \begin{array}{ll}
       1 +\sin x & \text{if } x < 0\\
       \cos x &  \text{if } 0 \leq x \leq \pi \\
       \sin x & \text{if } x > \pi 
     \end{array}
   \right.
\end{displaymath} 

			\item[7-10] Sketch the graph of an example of a function \emph{f} that satisfies all of the 
					given conditions.
					
			\item $\lim \limits_{x \to 0^{-}} f(x) = -1, \lim \limits_{x \to 0^{+}}f(x) = 2, f(0) = 1$

			\item[11--14] Guess the value of the liimit (if it exists) by evaluating the function at the given numbers (correct to six decimal places).
			
			\item $\lim \limits_{x \to 2} \frac{x^2 - 2x}{x^2 - x - 2}$ 
			
			  \emph{x} = 2.5, 2.1, 2.05, 2.01, 2.005 , 2.001, 1.9, 1.95, 1.99, 1.995, 1.999
			  
			  $$\frac{x^2 - 2x}{x^2 - x - 2} = \frac{x(x - 2)}{(x + 1)(x - 2)}$$
			  
			  $$f(2.5) = \frac{2.5(2.5 - 2)}{(2.5 + 1)(2.5 - 2)}$$
			  
			  $$f(2.5) = \frac{2.5(0.5)}{(3.5)(0.5)} = \frac{1.25}{1.75} = 0.71428571428$$
			  
			  $$f(2.1) = \frac{2.1(2.1 - 2)}{(2.1 + 1)(2.1 - 2)}$$
			  $$f(2.1) = \frac{2.1(0.1)}{(3.1)(0.1)} = \frac{0.21}{0.31} = 0.67741935483$$
			  
			  $$f(2.05) = \frac{2.05(2.05 - 2)}{(2.05 + 1)(2.05 - 2)}$$
			  $$f(2.05) = \frac{2.05(0.05)}{3.05(0.05)} = \frac{0.1025}{0.1525} = 0.67213114754$$
			  
			  $$f(2.01) = \frac{2.01(2.01 - 2)}{(2.01 + 1)(2.01 - 2)}$$
			  $$f(2.01) = \frac{2.01(0.01)}{3.01(0.01)} = \frac{0.0201}{0.0301} = 0.66777408637$$
			  
			  $$f(2.005) = \frac{2.005(2.005 - 2)}{(2.005 + 1)(2.005 - 2)} $$ 
			  $$f(2.005) = \frac{2.005(0.005)}{3.005(0.005)} = \frac{2.005}{3.005} = 0.66722129783$$
			  
			  $$f(2.001) = \frac{2.001(2.001 - 2)}{(2.001 + 1)(2.001 - 2)}$$
			  $$f(2.001) = \frac{2.001}{3.001} = 0.66677774075$$
			  
			  $$f(1.9) = \frac{1.9(1.9 - 2)}{(1.9 + 1)(1.9 - 2)}$$
			  $$f(1.9) = \frac{1.9}{2.9} = 0.65517241379$$
			  
			  $$f(1.95) = \frac{1.95(1.95 - 2)}{(1.95 + 1)(1.95 - 2)}$$
			  $$f(1.95) = \frac{1.95}{2.95} = 0.66101694915$$
			  
			  $$f(1.99) = \frac{1.99(1.99 - 2)}{(1.99 + 1)(1.99 - 2)}$$
			  $$f(1.99) = \frac{1.99}{2.99} = 0.66555183946$$
			  
			  $$f(1.995) = \frac{1.995(1.995 - 2)}{(1.995 + 1)(1.995 - 2)}$$
			  $$f(1.995) = \frac{1.995}{2.995} = 0.66611018363$$
			  
			  $$f(1.999) = \frac{1.999(1.999 - 2)}{(1.999 + 1)(1.999 - 2)}$$
			  $$f(1.999) = \frac{1.999}{2.999} = 0.6665555185$$
\begin{center}
			  \begin{tabular}{|c|c|}
			  	\hline
			  	$x$ & $f(x)$ \\
			  	\hline \hline			  	
			  	2.5 & 0.714286 \\
			  	2.1 &  0.677419 \\
			  	2.05 &  0.672131 \\
			  	2.01 &  0.667774 \\
			  	2.005 &  0.667221 \\
			  	2.001 & 0.666778 \\
			  	1.999 &  0.666556 \\
			  	1.995 & 0.666110 \\
			  	1.99 & 0.665552 \\
			  	1.95 &  0.661017 \\
			  	1.9 &  0.655172 \\
			  	\hline
			  \end{tabular}
\end{center}

			\item $\lim \limits_{x \to -1} \frac{x^2 - 2x}{x^2 - x - 2}$
			
				\emph{x} = 0, -0.5, -0.9, -0.95, -0.99, -0.999, -2, -1.5, -1.1, -1.01, -1.001
				
				$$\frac{x^2 - 2x}{x^2 - x - 2} = \frac{x(x-2)}{(x + 1)(x - 2)} = \frac{x}{x+1}$$
				
				$$f(0) = \frac{0}{0 + 1} = 0$$
				
				$$f(-0.5) = \frac{-0.5}{1 - 0.5} = \frac{-0.5}{0.5} = -1$$
				
				$$f(-0.9) = \frac{-0.9}{1 - 0.9} = \frac{-0.9}{0.1} = -9$$
				
				$$f(-0.95) = \frac{-0.95}{1 - 0.95} = \frac{-0.95}{0.05} = -19$$
				
				$$f(-0.99) = \frac{-0.99}{1- 0.99} = \frac{-0.99}{0.01} = -99$$
				
				$$f(-0.999) = \frac{-0.999}{1 - 0.999} = \frac{-0.999}{0.001} = -999$$
				
				$$f(-2) = \frac{-2}{1-2} = \frac{-2}{-1} = 2$$
				
				$$f(-1.5)  = \frac{-1.5}{1 - 1.5} = \frac{-1.5}{-0.5} = 3$$
				
				$$f(-1.1) = \frac{-1.1}{1 - 1.1} = \frac{-1.1}{-0.1} = 11$$
				
				$$f(-1.01) = \frac{-1.01}{1 - 1.01} = \frac{-1.01}{-0.01} = 101$$
				
				$$f(-1.001) = \frac{-1.001}{1-1.001} = \frac{-1.001}{-0.001} = 1001$$
				
				\begin{center}
				\begin{tabular}{|c|c|}
					\hline
					$x$ & $f(x)$ \\
					\hline \hline
					0 & 0 \\
					-0.5 & -1 \\
					-0.9 & -9 \\
					-0.95 & -19 \\
					-0.99 & -99 \\
					-0.999 & -999 \\
					-1.001 & 1001 \\
					-1.01 & 101 \\
					-1.1 & 11 \\
					-1.5 & 3 \\
					-2 & 2 \\
					\hline
				\end{tabular}
				\end{center}
				
				The limits are $\lim \limits _{x \to -1^{+}}  \frac{x^2 - 2x}{x^2 - x - 2} = -\infty$ and
				$\lim \limits _{x \to -1^{-}} \frac{x^2 - 2x}{x^2 - x - 2} = +\infty$
				
			\item $\lim \limits _{x \to 0} \frac{\sin x}{x + \tan x}$
			
				$x = \pm 1, \pm 0.5, \pm 0.2, \pm 0.1, \pm 0.05, \pm 0.01$
				
				$$f(1) = \frac{\sin 1}{1 + \tan 1} = 0.329033$$
				$$f(0.5) = \frac{\sin 0.5}{0.5 + \tan 0.5} = 0.458209$$
				$$f(0.2) = \frac{\sin 0.2}{0.2 + \tan 0.2} = 0.493331$$
				$$f(0.1) = \frac{\sin 0.1}{0.1 + \tan 0.1} = 0.498333$$
				$$f(0.05) = \frac{\sin 0.05}{0.05 + \tan 0.05} = 0.499583$$
				$$f(0.01) = \frac{\sin 0.01}{0.01 + \tan 0.01} = 0.499983$$
				$$f(-0.01) = \frac{\sin -0.01}{-0.01 + \tan -0.01} = 0.499983$$
				$$f(-0.05) = \frac{\sin -0.05}{-0.05 + \tan -0.05} = 0.499583$$
				$$f(-0.1) = \frac{\sin - 0.1} {-0.1 + \tan -0.1} = 0.498333$$
				$$f(-0.2) = \frac{\sin -0.2}{-0.2 + \tan -0.2} = 0.493331$$
				$$f(-0.5) = \frac{\sin -0.5}{-0.5 + \tan -0.5} = 0.458209$$
				$$f(-1) = \frac{\sin -1}{-1 + \tan -1} =0.329033$$
				
				\begin{center}
				\begin{tabular}{|c|c|}
				\hline
				$x$ & $f(x)$ \\
				\hline \hline
				1 &  0.32903 \\
				0.5 & 0.458209 \\
				0.2 &  0.493331 \\
				0.1 &  0.498333 \\
				0.05 & 0.499583 \\
				0.01 &  0.499983 \\
				-0.01 & 0.499983 \\
				-0.05 & 0.499583\\
				-0.1 &  0.498333 \\
				-0.2 &  0.493331 \\
				-0.5 & 0.458209 \\
				-1 & 0.329033 \\
				\hline
				\end{tabular}
				\end{center}
				
				It looks like $\lim \limits _{x \to 0} \frac{\sin x}{x + \tan x} = 0.5$
			
			\item 	$\lim \limits _{h \to 0} \frac{(2 + h)^5 - 32}{h}$
			
			$h = \pm 0.5, \pm 0.1, \pm 0.01, \pm 0.001, \pm 0.0001$
			
			$$f(0.5) = \frac{(2 + 0.5)^5 -32}{0.5} = \frac{2.5^5 - 32}{0.5} = \frac{97.65625 - 32}{0.5} = \frac{65.65625}{0.5} = 131.3125$$
			
			$$f(0.1) = \frac{(2 + 0.1)^5 - 32}{0.1} = \frac{2.1^5 -32}{0.1} = \frac{40.84101 - 32}{0.1} = \frac{8.84101}{0.1} = 88.4101$$
			
			$$f(0.01) = \frac{(2 + 0.01)^5 - 32}{0.01} = \frac{2.01^5 - 32}{0.01} = \frac{32.8080401001 - 32}{0.01} $$
			$$= \frac{0.8080401001}{0.01} = 80.80401001$$
			
			$$f(0.001) = \frac{(2+ 0.001)^5 - 32}{0.001} = \frac{2.001^5 - 32}{0.001} = $$
			$$\frac{32.080080040010001 - 32}{0.001} = \frac{0.080080040010001}{0.001} = 80.080040010001$$
			
			$$f(0.0001) = \frac{(2 + 0.0001)^5 - 32}{0.0001} = \frac{2.0001^5  - 32}{0.0001} = \frac{32.008000800040001 - 32}{0.0001} $$
			$$= \frac{0.008000800040001}{0.0001} = 80.00800040001$$
			
			$$f(-0.0001) = \frac{(2 - 0.0001)^5 - 32}{-0.0001}  = \frac{1.9999^5 - 32}{-0.0001} = \frac{31.992000799960001-32}{-0.0001}$$
			$$ = \frac{-0.007999200039999}{-0.0001} = 79.99200039999$$
			
			$$f(-0.001) = \frac{(2 - 0.001)^5 - 32}{-0.001}  =\frac{1.999^5 - 32}{-0.001} = \frac{31.920079960009999 - 32}{-0.001} = $$
			$$\frac{-0.079920039990001}{-0.001} = 79.920039990001$$
			
			$$f(-0.01) = \frac{(2-0.01)^5 - 32}{-0.01} = \frac{1.99^5 - 32}{-0.01} = \frac{31.2079600999 - 32}{-0.01} = $$
			$$\frac{-0.7920399001}{-0.01} = 79.20399001$$
			
			$$f(-0.1) = \frac{(2- 0.1)^5 - 32}{-0.1} = \frac{1.9^5 - 32}{-0.1} = \frac{24.76099 -32}{-0.1} = \frac{-7.23901}{-0.1} = 72.3901$$
			
			$$f(-0.5) = \frac{(2- 0.5)^2 - 32}{-0.5} = \frac{1.5^5 -32}{-0.5} = \frac{7.59375 - 32}{-0.5} = \frac{-24.40625}{-0.5} = 48.8125$$
			
			\begin{center}
				\begin{tabular}{|c|c|}
				\hline
				$x$ & $f(x)$ \\
				\hline \hline
				0.5 &  131.3125 \\
				0.1 & 88.4101 \\
				0.01 &  80.804010 \\
				0.001 &   80.080040 \\
				0.0001 & 80.008000 \\
				-0.0001 &  79.992000 \\
				-0.001 & 79.920040 \\
				-0.01 & 79.203990 \\
				-0.1 &  72.3901 \\
				-0.5 & 48.8125 \\
				\hline
				\end{tabular}
			\end{center}
			
			It looks like $\lim \limits _{h \to 0} \frac{(2 + h)^5 - 32}{h} = 80$
			
			\item[15--18] Use a table of values to estimate the value of the limit. If you have a graphing
				device, use it to confirm your result graphically.
				
			\item $\lim \limits _{x \to 0} \frac{\sqrt{x + 4} - 2}{x}$
			
				$$f(1) = \frac{\sqrt{1 + 4} - 2}{1} = \sqrt{5} - 2 = $$
				$$2.2360679774997897 - 2 = 0.2360679774997897$$
				
				$$f(0.5) = \frac{\sqrt{0.5 + 4} - 2}{0.5} =\frac{\sqrt{4.5} - 2}{0.5}$$
				$$= \frac{2.1213203435596426 - 2}{0.5} = \frac{0.1213203435596426}{0.5}$$
				$$ = 0.2426406871192852$$
				
				$$f(0.1) = \frac{\sqrt{0.1 + 4} - 2}{0.1} = \frac{\sqrt{4.1} - 2}{0.1}$$
				$$ = \frac{2.0248456731316587 - 2}{0.1} = \frac{0.0248456731316587}{0.1}$$
				$$ = 0.248456731316587$$
				
				$$f(0.01) = \frac{\sqrt{0.01 + 4}-2}{0.01} = \frac{\sqrt{4.01} - 2}{0.01}$$
				$$= \frac{2.0024984394500786 - 2}{0.01} = \frac{0.0024984394500786}{0.01}$$
				$$ = 0.24984394500786$$
				
				$$f(0.001) = \frac{\sqrt{0.001 + 4}-2}{0.001} = \frac{\sqrt{4.001} -2}{0.001}$$
				$$= \frac{2.0002499843769528-2}{0.001} = \frac{0.0002499843769528}{0.001}$$
				$$ = 0.2499843769528$$
				
				$$f(-0.001) = \frac{\sqrt{4 - 0.001} - 2}{-0.001}  =\frac{\sqrt{3.999} - 2}{-0.001}$$
				$$ = \frac{1.9997499843730466 - 2}{-0.001} = \frac{-0.0002500156269534}{-0.001}$$
				$$ = 0.2500156269534$$
				
				$$f(-0.01) = \frac{\sqrt{4 - 0.01} - 2}{-0.01} = \frac{\sqrt{3.99} - 2}{-0.01}$$
				$$= \frac{1.9974984355438179 - 2}{-0.01} = \frac{-0.0025015644561821}{-0.01}$$
				$$ = 0.25015644561821$$
				
			\begin{center}
				\begin{tabular}{|c|c|}
				\hline
				$x$ & $f(x)$ \\
				\hline \hline
				1 &  0.2360679774997897 \\
				0.5 & 0.2426406871192852 \\
				0.1 &   0.248456731316587 \\
				0.01 &  0.24984394500786 \\
				0.001 & 0.2499843769528 \\
				-0.001 &  0.2500156269534 \\
				-0.01 & 0.25015644561821 \\
				\hline
				\end{tabular}
			\end{center}
			
			It looks like $\lim \limits _{x \to 0} \frac{\sqrt{x + 4} - 2}{x} = 0.25$
			
			\item $\lim \limits _{x \to 0} \frac{\tan 3x}{\tan 5x}$
	\end{enumerate}
\end{document}