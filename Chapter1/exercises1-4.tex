\documentclass{article}

\renewcommand{\thesection}{} % remove section
\renewcommand{\labelenumiii}{(\roman{enumiii})}

\usepackage{amsmath}

\begin{document}
	\section{1.4 Exercises}
	
	\begin{enumerate}
		\item Given that
		
		$\lim \limits _{x \to 2} f(x) = 4$ \hfill $\lim \limits _{x \to 2} g(x) = -2$ \hfill $\lim \limits _{x \to 2} h(x) = 0$
		
		find the limits that exist. If the limit does not exist, explain why.
		
		\begin{enumerate}
		
			\item $\lim \limits _{x \to 2} [ f(x) + 5g(x)]$
				
			From rule 1:
				$$ = \lim \limits _{x \to 2} f(x) + \lim \limits _{x \to 2} 5g(x)$$
				
			From rule 3:
				$$ = \lim \limits _{x \to 2} f(x) + 5\lim \limits _{x \to 2} g(x)$$

			Substituting in the given limits:
				$$ = (4) + 5(-2) = -6$$
				
		
			\item $\lim \limits _{x \to 2} [g(x)]^3$
			
			From rule 6:
			
				$$ = [\lim \limits _{x \to 2} g(x)]^3$$
				
			Substituting in the given limits:
			
				$$ = [-2]^3 = -8$$				
				
			\item $\lim \limits _{x \to 2} \sqrt{f(x)}$
			
			From rule 11:
				
			$$ = \sqrt{\lim \limits _{x \to 2} f(x)}$$
			
			Substituting in the given limits:
			
			$$ = \sqrt{4} = 2$$
			
			\item $\lim \limits _{x \to 2} \frac{3f(x)}{g(x)}$
			
				From rule 3:
			
				$$ = 3 \lim \limits _{x \to 2} \frac{f(x)}{g(x)}$$
				
				From rule 5:
				
				$$ = 3 \frac{\lim \limits _{x \to 2} f(x)}{\lim \limits _{x \to 2} g(x)}$$
				
				Substituting in the given limits:
				
				$$ 3 \frac{4}{-2} = 3 \times -2 = -6$$
				
			\item $\lim \limits _{x \to 2} \frac{g(x)}{h(x)}$
			
				This limit does not exist. We cannot apply rule 5 as $\lim \limits _{x \to 2} h(x) = 0$.
				
			\item $\lim \limits _{x \to 2} \frac{g(x)h(x)}{f(x)}$
			
				From rule 5:
			
				$$ \frac{\lim \limits _{x \to 2} g(x)h(x)}{\lim \limits _{x \to 2} f(x)}$$
				
				From rule 4:
				
				$$ \frac{\lim \limits _{x \to 2} g(x) \lim \limits _{x \to 2} h(x)}{\lim \limits _{x \to 2} f(x)}$$
				
				Substituting in the given limits:
				
				$$\frac{(-2)(0)}{4} = 0$$
		\end{enumerate}
		
		\item The graphs of $f$ and $g$ are given. Use them to evaluate each limit, if it exists. If the limit
			does not exist, explain why.
			
		\item[3--9] Evaluate the limit and justify each step by indicating the appropriate Limit Law(s).
		
		\item $\lim \limits _{x \to 3} (5x^3 - 3x^2 + x - 6)$
		
			Using Limit Laws 1 and 2...
			
			$$\lim \limits _{x \to 3} 5x^3 - \lim \limits _{x \to 3} 3x^2 + \lim \limits _{x \to 3} x - \lim \limits _{x \to 3} 6$$
			
			Using Limit Law 3:
			
			$$5\lim \limits _{x \to 3} x^3 - 3 \lim \limits _{x \to 3} x^2 + \lim \limits _{x \to 3} x - \lim \limits _{x \to 3} 6$$
			
			Using Limit Law 6:
			
			$$5[\lim \limits _{x \to 3} x]^3 - 3 [\lim \limits _{x \to 3} x]^2 + \lim \limits _{x \to 3} x - \lim \limits _{x \to 3} 6$$
			
			Using Limit Laws 7 and 8:
			
			$$5[3]^3 - 3 [3]^2 + 3 -  6 = 5(27) - 3(9) - 3 = 105$$			
			
		\item $\lim \limits _{t \to -1} (t^2 + 1)^3 (t + 3)^5$
		
			Using Limit Law 4:
			
			$$[\lim \limits _{t \to 1} (t^2 + 1)^3] \cdot [\lim \limits _{t \to 1} (t + 3)^5]$$
			
			Using Limit Law 6:
			
			$$[\lim \limits _{t \to 1} (t^2 + 1)]^3 \cdot [\lim \limits _{t \to 1} (t + 3)]^5$$
			
			Using Limit Law 1:
			
			$$[\lim \limits _{t \to 1} t^2 + \lim \limits _{t \to 1} 1]^3 \cdot [\lim \limits _{t \to 1} t + \lim \limits _{t \to 1} 3]^5$$
			
			Using Limit Law 4:
			
			$$[[\lim \limits _{t \to 1} t]^2 + \lim \limits _{t \to 1} 1]^3 \cdot [\lim \limits _{t \to 1} t + \lim \limits _{t \to 1} 3]^5$$
			
			Using Limit Laws 7 and 8:
			
			$$(1^2 + 1)^3 (1 + 3)^5 = 2^3 4^5 = 8192$$
			
		\item $\lim \limits _{t \to -2} \frac{t^4 - 2}{2t^2 - 3t + 2}$
		
			Using Limit Law 5:
			
			$$\frac{\lim \limits _{t \to -2} t^4 - 2}{\lim \limits _{t \to -2} 2t^2 -3t +2}$$
		
			Using Limit Laws 1 and 2:
			
			$$\frac{\lim \limits _{t \to -2} t^4 - \lim \limits _{t \to -2} 2}
			{\lim \limits _{t \to -2} 2t^2 - \lim \limits _{t \to -2} 3t + \lim \limits _{t \to -2} 2}$$
			
			Using Limit Law 3:
			
			$$\frac{\lim \limits _{t \to -2} t^4 - \lim \limits _{t \to -2} 2}
			{2 \lim \limits _{t \to -2} t^2 - 3 \lim \limits _{t \to -2} t + \lim \limits _{t \to -2} 2}$$
			
			Using Limit Law 6:
			
			$$\frac{[\lim \limits _{t \to -2} t]^4 - \lim \limits _{t \to -2} 2}
			{2 [\lim \limits _{t \to -2} t]^2 - 3 \lim \limits _{t \to -2} t + \lim \limits _{t \to -2} 2}$$
			
			Using Limit Laws 7 and 8:
			
			$$\frac{(-2)^4 -  2}{2(-2)^2 - 3(-2) +  2} = \frac{16 -2}{8 + 6 + 2} = \frac{14}{16} = \frac{7}{8}$$
			
		\item $\lim \limits _{u \to -2} \sqrt{u^4 + 3u + 6}$
		
			Using Limit Law 11:
			
			$$\sqrt{\lim \limits _{u \to -2} u^4 + 3u +6}$$
			
			Using Limit Laws 1 and 2:
			
			$$\sqrt{\lim \limits _{u \to -2} u^4 + \lim \limits _{u \to -2} 3u + \lim \limits _{u \to -2} 6}$$
			
			Using Limit Laws 3 and 6:
			
			$$\sqrt{[\lim \limits _{u \to -2} u]^2 + 3 \lim \limits _{u \to -2} u + \lim \limits _{u \to -2} 6}$$
			
			Using Limit Laws 7 and 8:
			
			$$\sqrt{(-2) ^2 + 3(-2) + 6} = \sqrt{4 - 6 + 6} = 2$$
			
		\item $\lim \limits _{x \to 8} ( 1 + \sqrt[3]{x})(2 - 6x^2 + x^3)$
		
		Using Limit Law 4:
		
			$$\lim \limits _{x \to 8} (1 + \sqrt[3]{x}) \cdot \lim \limits _{x \to 8} (2 -6x^2 + x^3)$$
			
		Using Limit Laws 1 and 2:
		
			$$(\lim \limits _{x \to 8} 1 + \lim \limits _{x \to 8} \sqrt[3]{x}) \cdot 
				(\lim \limits _{x \to 8} 2 - \lim \limits _{x \to 8} 6x^2 + \lim\limits _{x \to 8} x^3)$$
				
		Using Limit Law 3:
				
			$$(\lim \limits _{x \to 8} 1 + \lim \limits _{x \to 8} \sqrt[3]{x}) \cdot 
				(\lim \limits _{x \to 8} 2 - 6 \lim \limits _{x \to 8} x^2 + \lim\limits _{x \to 8} x^3)$$
				
		Using Limit Laws 6 and 11:
		
			$$(\lim \limits _{x \to 8} 1 +  \sqrt[3]{\lim \limits _{x \to 8} x}) \cdot 
				(\lim \limits _{x \to 8} 2 - 6 [\lim \limits _{x \to 8} x]^2 + [\lim\limits _{x \to 8} x]^3)$$
				
		Using Limit Laws 7 and 8:
		
			$$( 1 +  \sqrt[3]{8}) \cdot 
				( 2 - 6(8)^2 + (8)^3) = 3 \cdot 130 = 390$$		
				
		\item $\lim \limits _{x \to 0} \frac{\cos ^4 x}{5 + 2x^3}$
		
			Using Limit Law 5:
			
			$$\frac{\lim \limits _{x \to 0} \cos ^4 x}{\lim \limits _{x \to 0} 5 + 2x^3}$$
			
			Using Limit Laws 1 and 3:
			
			$$\frac{\lim\limits _{x \to 0} \cos ^4 x}{\lim \limits _{x \to 0} 5 + 2 \lim \limits _{x \to 0} x^3}$$
			
			Using Limit Law 6:
			
			$$\frac{[\lim \limits _{x \to 0} \cos x]^4}{\lim \limits _{x \to 0} 5 + 2 [\lim \limits _{x \to 0} x]^3}$$
			
			Using Limit Laws 7 and 8, $\lim \limits _{x \to 0} \cos x$ follows the Direct Substitution Principle.
			
			$$\frac{1}{5}$$
			
		\item $\lim \limits _{\theta \to \pi / 2} \theta \sin \theta$
		
			Using Limit Law 4:
			
			$$\lim \limits _{\theta \to \pi / 2} \theta \cdot \lim \limits _{\theta \to \pi / 2} \sin \theta$$
			
			Using Limit Law 8 and Direct Substitution Principle:
			
			$$\pi /2 \cdot 1= \frac{\pi}{2}$$
		
		\item{}

		\begin{enumerate}
			\item What is wrong with the following equation?

			$$\frac{x^2 + x - 6}{x-2} = x + 3$$

			In this equation the left hand side is undefined when $x=2$, whereas
			the right hand side is defined.

			\item In view of part (a), explain why the equation

			$$\lim \limits _{x \to 2} \frac{x^2 + x - 6}{x-2} = \lim \limits _{x \to 2} (x+3)$$

			is correct.

			This is correct as $\lim \limits _{x \to a} g(x) = \lim \limits _{x \to a} f(x)$ when $f(x) = g(x)$
			except at $a$.
		\end{enumerate}

		\item[11-28] Evaluate the limit, if it exists.

		\item $\lim \limits _{x \to 5} \frac{x^2-6x+5}{x-5}$

		$$\lim \limits _{x \to 5} \frac{x^2-6x+5}{x-5} = \lim \limits _{x \to 5} \frac{(x-1)(x-5)}{x-5}$$
		$$ = \lim \limits _{x \to 5} x-1$$

		So the limit of $\lim \limits _{x \to 5} x - 1 = -4$ by the Direct Substitution Principle.

		\item $\lim \limits _{x \to -4} \frac{x^2 + 5x + 4}{x^2 + 3x - 4}$

		Factorising...

		$$\lim \limits _{x \to -4} \frac{(x+1)(x+4)}{(x+4)(x-1)} = \lim \limits _{x \to -4} \frac{x+1}{x-1}$$

		From the Direct Substitution Principle $\lim \limits _{x \to -4} \frac{x+1}{x-1} = \frac{3}{5}$.

		\item $\lim \limits _{x \to 5} \frac{x^2 - 5x + 6}{x-5}$

		If we factorise this we get...

		$$\lim \limits _{x \to 5} \frac{(x-2)(x-3)}{x-5}$$

		So this limit does not exist.

		\item $\lim \limits _{x \to -1} \frac{x^2 - 4x}{x^2 - 3x - 4}$

		This factorises to...

		$$\lim \limits _{x \to -1} \frac{x(x - 4)}{(x-4)(x+1)} = \lim \limits _{x \to -1} \frac{x}{x+1}$$

		So this limit does not exist.

		\item $\lim \limits _{t \to -3} \frac{t^2 - 9}{2t^2 + 7t + 3}$

		This factorises to

		$$\lim \limits _{t \to -3} \frac{(t+3)(t-3)}{(t+3)(2t+1)}$$

		$$\lim \limits _{t \to -3} \frac{t-3}{2t+1}$$

		By the direct substitution principle the limit is $\frac{6}{5}$.

		\item $\lim \limits _{x \to -1} \frac{2x^2 + 3x + 1}{x^2 - 2x -3}$

		If we factorise...

		$$\lim \limits _{x \to -1} \frac{(2x+1)(x+1)}{(x-3)(x+1)}$$

		$$\lim \limits _{x \to -1} \frac{2x+1}{x-3}$$

		By direct substitution this ends up as $\frac{1}{5}$.

		\item $\lim \limits _{h \to 0} \frac{(-5 + h)^2 - 25}{h}$

		If we expand the binomial...

		$$\lim \limits _{h \to 0} \frac{h^2 - 10h + 25 - 25}{h}$$

		$$\lim \limits _{h \to 0} \frac{h^2 -10h}{h}$$

		$$\lim \limits _{h \to 0} h - 10 = -10$$

		\item $\lim \limits _{h \to 0} \frac{\sqrt{1+h} - 1}{h}$

		We need manipulate the fraction so we can determine the limit.

		$$\lim \limits _{h \to 0} \frac{\sqrt{1+h}-1}{h} \cdot (\frac{\sqrt{1+h}+1}{\sqrt{1+h}+1})$$

		$$\lim \limits _{h \to 0} \frac{1+h-1}{h(\sqrt{1+h}+1)}$$

		$$\lim \limits _{h \to 0} \frac{1}{\sqrt{1+h} +1} = \frac{1}{2}$$

		\item $\lim \limits _{x \to -2} \frac{x+2}{x^3 +8}$

		We need to factorise the denominator.

		$$\lim \limits _{x \to -2} \frac{x+2}{(x+2)(x^2 -2x +4)}$$

		$$\lim \limits _{x \to -2} \frac{1}{x^2-2x+4} = \frac{1}{12}$$

		\item $\lim \limits _{x \to -1} \frac{x^2 + 2x + 1}{x^4 - 1}$

			We need to do lots of factorising.

			$$\lim \limits _{x \to -1} \frac{(x + 1)^2}{x^4 -1}$$

			$$\lim \limits _{x \to -1} \frac{(x+1)^2}{(x^2+1)(x^2-1)}$$

			$$\lim \limits _{x \to -1} \frac{(x+1)^2}{(x^2+1)(x+1)(x-1)}$$

			Cancelling out the factors.

			$$\lim \limits _{x \to -1} \frac{x+1}{(x^2+1)(x-1)} = 0$$

		\item $\lim \limits _{h \to 0} \frac{\sqrt{9+h}-3}{h}$

			$$\lim \limits _{h \to 0} \frac{(\sqrt{9+h}-3)(\sqrt{9+h}+3)}{h(\sqrt{9+h}+3}$$

			$$\lim \limits _{h \to 0} \frac{9+h-9}{h(\sqrt{9+h}+3)}$$

			$$\lim \limits _{h \to 0} \frac{h}{h(\sqrt{9+h}+3)}$$

			$$\lim \limits _{h \to 0} \frac{1}{\sqrt{9+h}+3)} = \frac{1}{6}$$

		\item $\lim \limits _{h \to 0} \frac{(3+h)^{-1} - 3^{-1}}{h}$

			$$\lim \limits _{h \to 0} \frac{\frac{1}{3+h} - \frac{1}{3}}{h}$$

			$$\lim \limits _{h \to 0} \frac{\frac{3 - (3+h)}{3(3+h)}}{h}$$

			$$\lim \limits _{h \to 0} \frac{-h}{3h(3+h)}$$

			$$\lim \limits _{h \to 0} \frac{-1}{3h + 9} = - \frac{1}{9}$$

		\item $\lim \limits _{x \to 16} \frac{4 - \sqrt{x}}{16x - x^2}$

			$$\lim \limits _{x \to 16} \frac{4-\sqrt{x}}{x(16-x)}$$

			$$\lim \limits _{x \to 16} \frac{(4-\sqrt{x})(4+\sqrt{x}}{x(16-x)(4+\sqrt{x})}$$

			$$\lim \limits _{x \to 16} \frac{16-x}{(x(16-x)(4+\sqrt{x})}$$

			$$\lim \limits _{x \to 16} \frac{1}{x(4+\sqrt{x}} = \frac{1}{24}$$

		\item $\lim \limits _{x \to 0} ( \frac{1}{t} - \frac{1}{t^2 + t})$


			$$\lim \limits _{x \to 0} \frac{t^2 + t - t}{t(t^2+t)}$$

			$$\lim \limits _{x \to 0} \frac{t^2}{t^2(t+1)}$$

			$$\lim \limits _{x \to 0} \frac{1}{t+1} = 1$$

		\item $\lim \limits _{x \to -4} \frac{\frac{1}{4} + \frac{1}{x}}{4+x}$

			$$\lim \limits _{x \to -4} \frac{\frac{x+4}{4x}}{4+x}$$

			$$\lim \limits _{x \to -4} \frac{x+4}{4x{x+4}}$$

			$$\lim \limits _{x \to -4} \frac{1}{4x} = -\frac{1}{16}$$

		\item $\lim \limits _{x \to -4} \frac{\sqrt{x^2 + 9} - 5}{x+4}$

			$$\lim \limits _{x \to -4} \frac{(\sqrt{x^2+9}-5)(\sqrt{x^2+9}+5}{(x+4)(\sqrt{x^2+9}+5)}$$

			$$\lim \limits _{x \to -4} \frac{x^2-16}{(x+4)(\sqrt{x^2+9}+5})$$

			$$\lim \limits _{x \to -4} \frac{(x-4)(x+4)}{(x+4)(\sqrt{x^2+9}+5})$$

			$$\lim \limits _{x \to -4} \frac{x-4}{\sqrt{x^2+9}+5} = - \frac{8}{10}$$

		\item $\lim \limits _{h \to 0} \frac{\frac{1}{(x+h)^2}-\frac{1}{x^2}}{h}$

			$$\lim \limits _{h \to 0} \frac{\frac{x^2 - (x+h)^2}{x^2(x+h)^2}}{h}$$

			$$\lim \limits _{h \to 0} \frac{x^2 - (x+h)^2}{hx^2(x+h)^2}$$

			$$\lim \limits _{h \to 0} \frac{x^2-x^2 -2xh - h^2}{hx^2(x+h)^2}$$

			$$\lim \limits _{h \to 0} \frac{-2xh-h^2}{hx^2(x+h)^2}$$

			$$\lim \limits _{h \to 0} \frac{-2x-h}{x^2(x+h)^2} = -\frac{2}{x^3}$$

		\item If $4x \leq f(x) \leq x^2 - 4x + 7$ for $x \geq 0$, find $\lim \limits _{x \to 4} f(x)$

			$$\lim \limits _{x \to 4} 4x - 9 = 7$$

			$$\lim \limits _{x \to 4} x^2 - 4x + 7 = 7$$

			So according to the squeeze theorem...

			$$\lim \limits _{x \to 4} f(x) = 7$$

		\item[37-42] Find the limit, if it exists. If the limit does not exist, explain why.

		\item $\lim \limits _{x \to 3} (2x + |x-3|)$

			Take the right hand limit

			$$\lim \limits _{x \to 3^{+}} (2x + |x-3|) = 2x + (x-3) = 6$$

			Take the left hand limit

			$$\lim \limits _{x \to 3^{-}} (2x + |x-3|) = 2x - (x-3) = 6$$

			The two sides of the limit are the same so $\lim \limits _{x \to 3} (2x + |x-3|) = 6$

		\item $\lim \limits _{x \to -6} \frac{2x+12}{|x+6|}$

			$$|x+6| = x+6 \text{ if } x \geq -6, -(x+6) \text{ if } x < -6$$

			Let's work out the right handed limit


			$$\lim \limits _{x \to -6^{+}} \frac{2(x+6)}{x+6} = 2$$

			Working out the left handed limit

			$$\lim \limits _{x \to -6^{-}} \frac{2(x+6)}{-(x+6)} = -2$$

			Because the left and right hand limits are not equal the limit does not exist.

		\item $\lim \limits _{x \to 0.5^{-}} \frac{2x-1}{|2x^3-x^2}$

			Juggling the denominator

			$$\lim \limits _{x \to 0.5^{-}} \frac{2x-1}{x^2|2x-1|}$$

			$$|2x-1| = 2x-1 \text{ if } x \geq 0.5, -(2x-1) \text{ if } x < 0.5$$

			So

			$$\lim \limits _{x \to 0.5^{-}} \frac{2x-1}{-x^2(2x-1} = -\frac{1}{x^2} = -4$$

		\item $\lim \limits _{x \to -2} \frac{2 - |x|}{2+x}$

			$$|x| = x \text{ if } x\geq 0, -x \text{ if } x < 0$$

			$$\lim \limits _{x \to -2} \frac{2-(-x)}{2+x} = \frac{2+x}{2+x} = 1$$

		\item $\lim \limits _{x \to 0^{+}} (\frac{1}{x} - \frac{1}{|x|}$

			If $x \geq 0$ then$|x| = x$ so

			$$\lim \limits _{x \to 0^{+}} (\frac{1}{x}-\frac{1}{x}) = 0$$

		\item Let $g(x)=\frac{x^2+x-6}{|x-2|}$

		\begin{enumerate}

			\item Find

			\begin{enumerate}
			
				\item $\lim \limits _{x \to 2^{+}} g(x)$

				$$\frac{x^2+x-6}{|x-2|} = \frac{(x+3)(x-2)}{|x-2|}$$

				If $x > 2$ then $|x-2| = x-2$

				$$\lim \limits _{x \to 2^{+}} \frac{(x+3)(x-2)}{x-2}$$

				$$\lim \limits _{x \to 2^{+}} x+3 = 5$$

			\end{enumerate}
		\end{enumerate}		

		\item

		\begin{enumerate}

			\item If the symbol $[[]]$ denotes the greatest integer function
				defined in Example 8, evaluate

			\begin{enumerate}
				\item $\lim \limits _{x \to -2^{+}} [[x]]$

				Since $[[x]] = -2 \text{ for } -2 \geq x < -1$ then

				$$\lim \limits _{x \to -2^{+}} [[x]] = -2$$

				\item $\lim \limits _{x \to -2} [[x]]$

				We need to work out the left hand limit.

				Because $[[x]] = -3 \text{ for } -3 \leq x < -2$

				$$\lim \limits _{x \to -2^{-}} [[x]] = -3$$

				Because the left hand limit is not equal to the right hand limit
				the limit does not exist.

				\item $\lim \limits _{x \to -2.4} [[x]]$

				First we do the left hand limit.

				$$\lim \limits _{x \to -2.4^{-}} [[x]]$$

				Because $[[x]]=-3$ when $-2.6 \leq x < -2.4$

				$$\lim \limits _{x \to -2.4^{-}} [[x]] = -3$$

				The right hand limit

				$$\lim \limits _{x \to -2.4^{+}} [[x]]$$

				Because $[[x]] = -3$ when $-2.4 \leq x < 2.2$

				$$\lim \limits _{x \to -2.4^{+}} [[x]] = -3$$

				so $\lim \limits _{x \to -2.4} [[x]]=-3$
			\end{enumerate}

		\end{enumerate}
		\item In the theory of relativity, the Lorentz contraction formula

			$$L = L_{0}\sqrt{1 - v^2/c^2}$$

			expresses the length $L$ of an object as a function of its
			velocity $v$ with respect to an observer, where $L_{0}$ is the
			length of the object at rest and $c$ is the speed of light. Find
			$\lim _{v \to c^{-}} L$ and interpret the result. Why is the left-
			hand limit necessary?

			We can get this left-hand limit via direct substitution.

			$$\lim \limits _{L \to c^{-}} L_{0}\sqrt{1 - v^2/c^2}$$

			$$L_{0}\sqrt{1-c^2/c^2} = L_{0} \sqrt{1 - 1} = 0$$

			We can see from the limit as $v$ tends to $c$, the length of the
			object approaches zero. We must take the left-handed limit as the
			speed of light $c$ cannot be reached or exceeded.

		\item[49-56]

		\item $$\lim \limits _{t \to 0} \frac{\sin^2 3t}{t^2}$$

			We can split the power into it's two parts.

			$$\frac{\sin 3t}{t} \times \frac{\sin 3t}{t}$$

			And multiply each part by three over three.

			$$\frac{\sin3t}{3t} \times \frac{\sin 3t}{3t} \times \frac{9}{1}$$

			Using the fact that $\lim \limits _{\theta \to 0} \frac{\sin \theta}{\theta}  = 1$ the limit
			is 9.

		\item $\lim \limits _{x \to 0} \frac{\sin 3x}{5x^3 - 4x}$

			First we can factor the denominator and express as two fractions.

			$$\lim \limits _{x \to 0} \frac{\sin 3x}{x} \times \frac{1}{5x^2 - 4}$$

			Then multiply by $\frac{3}{3}$.

			$$\lim \limits _{x \to 0} \frac{\sin 3x}{3x} \times \frac{3}{5x^2 - 4} = -\frac{3}{4}$$

		\item $\lim \limits _{x \to 0} \frac{\sin 3x \sin 5x}{x^2}$

			First we can split into two different fractions.

			$$\lim \limits _{x \to 0} \frac{\sin3x}{x} \frac{\sin5x}{x}$$

			And multiply by $\frac{3}{3} \frac{5}{5}$.

			$$\lim \limits _{x \to 0} \frac{3\sin 3x}{3x} \frac{5 \sin 5x}{5x} = 3 \times 5 = 15$$

		\item $\lim \limits _{\theta \to 0} \frac{\sin \theta}{\theta + \tan \theta}$

			If we divide both the numerator and denominator by $\sin \theta$.

			$$\lim \limits _{\theta \to 0} \frac{1}{\frac{\theta}{\sin \theta} + \frac{1}{\cos \theta}}$$

		\item $\lim \limits _{x \to 0} \frac{\sin{x^2}}{x}$

			If we multiply the top and bottom by $x$.

			$$\lim \limits _{x \to 0} x \frac{\sin{x^2}}{x^2} = 0$$

			
		\item If $p$ is a polynomial, show that $\lim _{x \to a} p(x) = p(a)$

			Let's look at the general form of a limit polynomial.

			$$\lim \limits _{x \to a} b_{n}x^{n} + b_{n-1}x^{n-1} + ... + b_{0}$$

			We can use the limit laws to show this, starting with Limit Law 1.

			$$\lim \limits _{x \to a} b_{n}x^{n} + \lim \limits _{x \to a} b_{n-1}x^{n-1} + ... + \lim \limits _{x \to a} b_{0}$$

			From Limit Law 3:

			$$b_{n} \lim \limits _{x \to a} x^{n} + b_{n-1} \lim \limits _{x \to a} x^{n-1} + ... + \lim \limits _{x \to a} b_{0}$$

			From Limit Law 6:
			
			$$b_{n} [\lim \limits _{x \to a} x]^{n} + b_{n-1} [\lim \limits _{x \to a} x ]^{n-1} + ... + \lim \limits _{x \to a} b_{0}$$

			Finally from limit laws 7 and 8:

			$$b_{n} a^{n} + b_{n-1} a^{n-1} + ... + b_{0}$$

		\item If $r$ is a rational function, use Exercise 57 to show that
			$\lim _{x \to a} r(x) = r(a)$ for every function in the domain of $r$.

			A rational function is of the form $\frac{f(x)}{g(x)}$ where $f$ and $g$
			are polynomial functions. So we are trying to find the limit of...

			$$\lim \limits _{x \to a} \frac{f(x)}{g(x)}$$

			We can of course use limit law 5.

			$$\frac{\lim \limits _{x \to a} f(x)}{\lim \limits _{x \to a} g(x)}$$

			Because $f$ and $g$ are polynomials we showed in the last exercise that
			for a polynomial $\lim _{x \to a} p(x) = p(a)$.

			And so...

			$$\frac{\lim \limits _{x \to a} f(x)}{\lim \limits _{x \to a} g(x)} = \frac{f(a)}{g(a)}$$

		\item To prove that sine has the Direct Substitution Property we need to show
			that $\lim _{x \to a} \sin x = \sin a$ for every number $a$. If we
			let $h = x - a$, then $x = a + h$ and $x \to a \Leftrightarrow h \to 0$.
			So an equivalent statement is that

			$$ \lim \limits _{h \to 0} \sin(a+h) = \sin a$$

			Use 1 to show that this is true.

			We can use the trigonmetric identity to expand this out.

			$$\lim \limits _{h \to 0} \sin a \cos h + \sin h \cos a$$

			From 1 we know that $\lim \limits _{\theta \to 0} \cos \theta = 1$
			and $\lim \limits _{\theta \to 0} \sin \theta = 0$.

			So...

			$$1 \times \sin a + 0 \times \cos a = \sin a$$

		\item Prove that cosine has the Direct Substitution Property

		From the previous exercise...

		$$\lim \limits _{x \to a} \cos x = \cos a$$

		If we let $h = x -a$, then $x = a + h$ and $x \to a \Leftrightarrow h \to 0$.
		So an equivalent statement is that

		$$ \lim \limits _{h \to 0} \cos(a+h) = \cos a$$

		We can use the trigonometric identity to expand this out.

		$$\lim \limits _{h \to 0} \cos a \cos h - \sin h \sin h$$

		From 1 we know that $\lim \limits _{\theta \to 0} \cos \theta = 1$
		and $\lim \limits _{\theta \to 0} \sin \theta = 0$

		So...

		$$\cos a \times 1 - \sin a \times 0 = \cos a$$
		
	\end{enumerate}

\end{document}
