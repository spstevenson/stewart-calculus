\documentclass{article}

\renewcommand{\thesection}{} % remove section
\renewcommand{\labelenumiii}{(\roman{enumiii})}

\usepackage{amsmath}

\begin{document}
	\section{1.4 Exercises}
	
	\begin{enumerate}
		\item Given that
		
		$\lim \limits _{x \to 2} f(x) = 4$ \hfill $\lim \limits _{x \to 2} g(x) = -2$ \hfill $\lim \limits _{x \to 2} h(x) = 0$
		
		find the limits that exist. If the limit does not exist, explain why.
		
		\begin{enumerate}
		
			\item $\lim \limits _{x \to 2} [ f(x) + 5g(x)]$
				
			From rule 1:
				$$ = \lim \limits _{x \to 2} f(x) + \lim \limits _{x \to 2} 5g(x)$$
				
			From rule 3:
				$$ = \lim \limits _{x \to 2} f(x) + 5\lim \limits _{x \to 2} g(x)$$

			Substituting in the given limits:
				$$ = (4) + 5(-2) = -6$$
				
		
			\item $\lim \limits _{x \to 2} [g(x)]^3$
			
			From rule 6:
			
				$$ = [\lim \limits _{x \to 2} g(x)]^3$$
				
			Substituting in the given limits:
			
				$$ = [-2]^3 = -8$$				
				
			\item $\lim \limits _{x \to 2} \sqrt{f(x)}$
			
			From rule 11:
				
			$$ = \sqrt{\lim \limits _{x \to 2} f(x)}$$
			
			Substituting in the given limits:
			
			$$ = \sqrt{4} = 2$$
			
			\item $\lim \limits _{x \to 2} \frac{3f(x)}{g(x)}$
			
				From rule 3:
			
				$$ = 3 \lim \limits _{x \to 2} \frac{f(x)}{g(x)}$$
				
				From rule 5:
				
				$$ = 3 \frac{\lim \limits _{x \to 2} f(x)}{\lim \limits _{x \to 2} g(x)}$$
				
				Substituting in the given limits:
				
				$$ 3 \frac{4}{-2} = 3 \times -2 = -6$$
				
			\item $\lim \limits _{x \to 2} \frac{g(x)}{h(x)}$
			
				This limit does not exist. We cannot apply rule 5 as $\lim \limits _{x \to 2} h(x) = 0$.
				
			\item $\lim \limits _{x \to 2} \frac{g(x)h(x)}{f(x)}$
			
				From rule 5:
			
				$$ \frac{\lim \limits _{x \to 2} g(x)h(x)}{\lim \limits _{x \to 2} f(x)}$$
				
				From rule 4:
				
				$$ \frac{\lim \limits _{x \to 2} g(x) \lim \limits _{x \to 2} h(x)}{\lim \limits _{x \to 2} f(x)}$$
				
				Substituting in the given limits:
				
				$$\frac{(-2)(0)}{4} = 0$$
		\end{enumerate}
		
		\item The graphs of $f$ and $g$ are given. Use them to evaluate each limit, if it exists. If the limit
			does not exist, explain why.
			
		\item[3--9] Evaluate the limit and justify each step by indicating the appropriate Limit Law(s).
		
		\item $\lim \limits _{x \to 3} (5x^3 - 3x^2 + x - 6)$
		
			Using Limit Laws 1 and 2...
			
			$$\lim \limits _{x \to 3} 5x^3 - \lim \limits _{x \to 3} 3x^2 + \lim \limits _{x \to 3} x - \lim \limits _{x \to 3} 6$$
			
			Using Limit Law 3:
			
			$$5\lim \limits _{x \to 3} x^3 - 3 \lim \limits _{x \to 3} x^2 + \lim \limits _{x \to 3} x - \lim \limits _{x \to 3} 6$$
			
			Using Limit Law 6:
			
			$$5[\lim \limits _{x \to 3} x]^3 - 3 [\lim \limits _{x \to 3} x]^2 + \lim \limits _{x \to 3} x - \lim \limits _{x \to 3} 6$$
			
			Using Limit Laws 7 and 8:
			
			$$5[3]^3 - 3 [3]^2 + 3 -  6 = 5(27) - 3(9) - 3 = 105$$			
			
		\item $\lim \limits _{t \to -1} (t^2 + 1)^3 (t + 3)^5$
		
			Using Limit Law 4:
			
			$$[\lim \limits _{t \to 1} (t^2 + 1)^3] \cdot [\lim \limits _{t \to 1} (t + 3)^5]$$
			
			Using Limit Law 6:
			
			$$[\lim \limits _{t \to 1} (t^2 + 1)]^3 \cdot [\lim \limits _{t \to 1} (t + 3)]^5$$
			
			Using Limit Law 1:
			
			$$[\lim \limits _{t \to 1} t^2 + \lim \limits _{t \to 1} 1]^3 \cdot [\lim \limits _{t \to 1} t + \lim \limits _{t \to 1} 3]^5$$
			
			Using Limit Law 4:
			
			$$[[\lim \limits _{t \to 1} t]^2 + \lim \limits _{t \to 1} 1]^3 \cdot [\lim \limits _{t \to 1} t + \lim \limits _{t \to 1} 3]^5$$
			
			Using Limit Laws 7 and 8:
			
			$$(1^2 + 1)^3 (1 + 3)^5 = 2^3 4^5 = 8192$$
			
		\item $\lim \limits _{t \to -2} \frac{t^4 - 2}{2t^2 - 3t + 2}$
		
			Using Limit Law 5:
			
			$$\frac{\lim \limits _{t \to -2} t^4 - 2}{\lim \limits _{t \to -2} 2t^2 -3t +2}$$
		
			Using Limit Laws 1 and 2:
			
			$$\frac{\lim \limits _{t \to -2} t^4 - \lim \limits _{t \to -2} 2}
			{\lim \limits _{t \to -2} 2t^2 - \lim \limits _{t \to -2} 3t + \lim \limits _{t \to -2} 2}$$
			
			Using Limit Law 3:
			
			$$\frac{\lim \limits _{t \to -2} t^4 - \lim \limits _{t \to -2} 2}
			{2 \lim \limits _{t \to -2} t^2 - 3 \lim \limits _{t \to -2} t + \lim \limits _{t \to -2} 2}$$
			
			Using Limit Law 6:
			
			$$\frac{[\lim \limits _{t \to -2} t]^4 - \lim \limits _{t \to -2} 2}
			{2 [\lim \limits _{t \to -2} t]^2 - 3 \lim \limits _{t \to -2} t + \lim \limits _{t \to -2} 2}$$
			
			Using Limit Laws 7 and 8:
			
			$$\frac{(-2)^4 -  2}{2(-2)^2 - 3(-2) +  2} = \frac{16 -2}{8 + 6 + 2} = \frac{14}{16} = \frac{7}{8}$$
			
		\item $\lim \limits _{u \to -2} \sqrt{u^4 + 3u + 6}$
		
			Using Limit Law 11:
			
			$$\sqrt{\lim \limits _{u \to -2} u^4 + 3u +6}$$
			
			Using Limit Laws 1 and 2:
			
			$$\sqrt{\lim \limits _{u \to -2} u^4 + \lim \limits _{u \to -2} 3u + \lim \limits _{u \to -2} 6}$$
			
			Using Limit Laws 3 and 6:
			
			$$\sqrt{[\lim \limits _{u \to -2} u]^2 + 3 \lim \limits _{u \to -2} u + \lim \limits _{u \to -2} 6}$$
			
			Using Limit Laws 7 and 8:
			
			$$\sqrt{(-2) ^2 + 3(-2) + 6} = \sqrt{4 - 6 + 6} = 2$$
			
		\item $\lim \limits _{x \to 8} ( 1 + \sqrt[3]{x})(2 - 6x^2 + x^3)$
		
		Using Limit Law 4:
		
			$$\lim \limits _{x \to 8} (1 + \sqrt[3]{x}) \cdot \lim \limits _{x \to 8} (2 -6x^2 + x^3)$$
			
		Using Limit Laws 1 and 2:
		
			$$(\lim \limits _{x \to 8} 1 + \lim \limits _{x \to 8} \sqrt[3]{x}) \cdot 
				(\lim \limits _{x \to 8} 2 - \lim \limits _{x \to 8} 6x^2 + \lim\limits _{x \to 8} x^3)$$
				
		Using Limit Law 3:
				
			$$(\lim \limits _{x \to 8} 1 + \lim \limits _{x \to 8} \sqrt[3]{x}) \cdot 
				(\lim \limits _{x \to 8} 2 - 6 \lim \limits _{x \to 8} x^2 + \lim\limits _{x \to 8} x^3)$$
				
		Using Limit Laws 6 and 11:
		
			$$(\lim \limits _{x \to 8} 1 +  \sqrt[3]{\lim \limits _{x \to 8} x}) \cdot 
				(\lim \limits _{x \to 8} 2 - 6 [\lim \limits _{x \to 8} x]^2 + [\lim\limits _{x \to 8} x]^3)$$
				
		Using Limit Laws 7 and 8:
		
			$$( 1 +  \sqrt[3]{8}) \cdot 
				( 2 - 6(8)^2 + (8)^3) = 3 \cdot 130 = 390$$		
				
		\item $\lim \limits _{x \to 0} \frac{\cos ^4 x}{5 + 2x^3}$
		
			Using Limit Law 5:
			
			$$\frac{\lim \limits _{x \to 0} \cos ^4 x}{\lim \limits _{x \to 0} 5 + 2x^3}$$
			
			Using Limit Laws 1 and 3:
			
			$$\frac{\lim\limits _{x \to 0} \cos ^4 x}{\lim \limits _{x \to 0} 5 + 2 \lim \limits _{x \to 0} x^3}$$
			
			Using Limit Law 6:
			
			$$\frac{[\lim \limits _{x \to 0} \cos x]^4}{\lim \limits _{x \to 0} 5 + 2 [\lim \limits _{x \to 0} x]^3}$$
			
			Using Limit Laws 7 and 8, $\lim \limits _{x \to 0} \cos x$ follows the Direct Substitution Principle.
			
			$$\frac{1}{5}$$
			
		\item $\lim \limits _{\theta \to \pi / 2} \theta \sin \theta$
		
			Using Limit Law 4:
			
			$$\lim \limits _{\theta \to \pi / 2} \theta \cdot \lim \limits _{\theta \to \pi / 2} \sin \theta$$
			
			Using Limit Law 8 and Direct Substitution Principle:
			
			$$\pi /2 \cdot 1= \frac{\pi}{2}$$
		
		\item{}

		\begin{enumerate}
			\item What is wrong with the following equation?

			$$\frac{x^2 + x - 6}{x-2} = x + 3$$

			In this equation the left hand side is undefined when $x=2$, whereas
			the right hand side is defined.

			\item In view of part (a), explain why the equation

			$$\lim \limits _{x \to 2} \frac{x^2 + x - 6}{x-2} = \lim \limits _{x \to 2} (x+3)$$

			is correct.

			This is correct as $\lim \limits _{x \to a} g(x) = \lim \limits _{x \to a} f(x)$ when $f(x) = g(x)$
			except at $a$.
		\end{enumerate}

		\item[11-28] Evaluate the limit, if it exists.

		\item $\lim \limits _{x \to 5} \frac{x^2-6x+5}{x-5}$

		$$\lim \limits _{x \to 5} \frac{x^2-6x+5}{x-5} = \lim \limits _{x \to 5} \frac{(x-1)(x-5)}{x-5}$$
		$$ = \lim \limits _{x \to 5} x-1$$

		So the limit of $\lim \limits _{x \to 5} x - 1 = -4$ by the Direct Substitution Principle.

		\item $\lim \limits _{x \to -4} \frac{x^2 + 5x + 4}{x^2 + 3x - 4}$

		Factorising...

		$$\lim \limits _{x \to -4} \frac{(x+1)(x+4)}{(x+4)(x-1)} = \lim \limits _{x \to -4} \frac{x+1}{x-1}$$

		From the Direct Substitution Principle $\lim \limits _{x \to -4} \frac{x+1}{x-1} = \frac{3}{5}$.

		\item $\lim \limits _{x \to 5} \frac{x^2 - 5x + 6}{x-5}$

		If we factorise this we get...

		$$\lim \limits _{x \to 5} \frac{(x-2)(x-3)}{x-5}$$

		So this limit does not exist.

		\item $\lim \limits _{x \to -1} \frac{x^2 - 4x}{x^2 - 3x - 4}$

		This factorises to...

		$$\lim \limits _{x \to -1} \frac{x(x - 4)}{(x-4)(x+1)} = \lim \limits _{x \to -1} \frac{x}{x+1}$$

		So this limit does not exist.

		\item $\lim \limits _{t \to -3} \frac{t^2 - 9}{2t^2 + 7t + 3}$

		This factorises to

		$$\lim \limits _{t \to -3} \frac{(t+3)(t-3)}{(t+3)(2t+1)}$$

		$$\lim \limits _{t \to -3} \frac{t-3}{2t+1}$$

		By the direct substitution principle the limit is $\frac{6}{5}$.

		\item $\lim \limits _{x \to -1} \frac{2x^2 + 3x + 1}{x^2 - 2x -3}$

		If we factorise...

		$$\lim \limits _{x \to -1} \frac{(2x+1)(x+1)}{(x-3)(x+1)}$$

		$$\lim \limits _{x \to -1} \frac{2x+1}{x-3}$$

		By direct substitution this ends up as $\frac{1}{5}$.

		\item $\lim \limits _{h \to 0} \frac{(-5 + h)^2 - 25}{h}$

		If we expand the binomial...

		$$\lim \limits _{h \to 0} \frac{h^2 - 10h + 25 - 25}{h}$$

		$$\lim \limits _{h \to 0} \frac{h^2 -10h}{h}$$

		$$\lim \limits _{h \to 0} h - 10 = -10$$

		\item $\lim \limits _{h \to 0} \frac{\sqrt{1+h} - 1}{h}$

		We need manipulate the fraction so we can determine the limit.

		$$\lim \limits _{h \to 0} \frac{\sqrt{1+h}-1}{h} \cdot (\frac{\sqrt{1+h}+1}{\sqrt{1+h}+1})$$

		$$\lim \limits _{h \to 0} \frac{1+h-1}{h(\sqrt{1+h}+1)}$$

		$$\lim \limits _{h \to 0} \frac{1}{\sqrt{1+h} +1} = \frac{1}{2}$$

		\item $\lim \limits _{x \to -2} \frac{x+2}{x^3 +8}$

		We need to factorise the denominator.

		$$\lim \limits _{x \to -2} \frac{x+2}{(x+2)(x^2 -2x +4)}$$

		$$\lim \limits _{x \to -2} \frac{1}{x^2-2x+4} = \frac{1}{12}$$
		
	\end{enumerate}

\end{document}
