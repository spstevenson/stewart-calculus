\documentclass{article}

\renewcommand{\thesection}{} % Remove section numbers
\usepackage{amsmath}

\begin{document}

	\section{1.5 Continuity}

	\begin{enumerate}
		\item Write an equation that expresses the fact that a function $f$
			is continuous at the number 4.

			$$\lim \limits _{a \to 4} f(x) = f(a)$$

		\item If $f$ is continuous on $(-\infty, \infty)$, what can you
			say about its graph?

			There are no vertical asymptotes, jumps in value or
			undefined values in the graph.

		\item Explain why each function is continuous or discontinuous.

			\begin{enumerate}
				\item The temperature at a specific location as
					a function in time.

					The function is continuous as the temperature
					in a location cannot jump in value without passing
					through the intermediate values.

				\item The temperature at a specific time as a function of
					the distance due west from Paris.

					As before, the temperature cannot jump without passing
					through the intermediate values so this function is
					continuous.

				\item The altitude above sea level as a function of the distance
					due west from Paris.

					The altitude cannot jump without passing through the 
					intermediate values so this function is continuous.

				\item The cost of a taxi ride as a function of the distance travelled.

					Because at best the cost will be in 1p "jumps" (and is more often
					than not in 10-20p jumps) this function is discontinuous.

				\item The current in the circuit for the lights in a room as a function of
					time.

					The Heaviside function models the current in a circuit as a function
					of time. This function is discontinuous as for all practical purposes
					the current jumps from zero to its final value without passing though
					the intermediate values when the lights are switched on.
			\end{enumerate}

			\item Suppose $f$ and $g$ are continuous functions such that $g(2) = 6$
				and $\lim \limits _{x \to 2} [3f(x) + f(x)g(x)] = 36$. Find $f(2)$.

				Let's use our limit laws to prepare the equation.

				$$\lim \limits _{x \to 2} 3f(x) + \lim \limits _{x \to 2} f(x)g(x) = 36$$
				$$3\lim \limits _{x \to 2} f(x) + [\lim \limits _{x to 2} f(x)][\lim \limits _{x \to 2} g(x)] = 36$$

				Because $g$ is continuous $\lim \limits _{x \to 2} g(x) = g(2) = 6$

				$$3\lim \limits _{x \to 2} f(x) + 6\lim \limits _{x \to 2} f(x) = 36$$
				$$9\lim \limits _{x \to 2} f(x) = 36$$
				$$\lim \limits _{x \to 2} f(x) = 4$$

				Because $f$ is continuous $\lim \limits_{x \to 2} f(x) = f(2) = 4$	

			\item[12-13] Use the definition of continuity and the properties of limits
					to show that the function is continuous at the given
					number $a$.

			\item $f(x) = 3x^4 - 5x +\sqrt[3]{x^2 + 4}, a = 2$

				\begin{enumerate}
					\item $f(a)$ is defined (that is, $a$ is in the domain of $f$)

					$f(2)$ is defined.

					$$3(2)^2 - 5(2) + \sqrt[3]{(2)^2 + 4} = 12 - 10 + 2 = 4$$

					\item $\lim \limits _{x \to 2} f(2)$ exists.

					$$3[\lim \limits _{x \to 2} f(2)]^2 - 5\lim \limits _{x \to 2} \sqrt[3]{\lim \limits _{x \to 2} f(x)}$$

					%TODO: Finish this...
				\end{enumerate}

			\item $f(x) = (x + 2x^3)^4, a = -1$

				$f(a)$ is defined as...

				$$(-1 + 2(-1)^3)^4 = (-3)^4 = 81$$

				The limits exist...

				$$\lim \limits _{x \to -1} (x + 2x^3)^4 ...$$

				$$\lim \limits _{a to -1} f(a) = f(-1)$$ 

				%TODO: Finish this...

			\item Use the definition of continuity and the properties of
				limits to show that the function $f(x) = x\sqrt{16 - x^2}$
				is continuous on the interval $[-4,4]$.	

				We take $-4 \leq a \leq 4$.

				$$\lim \limits _{a \to x} x\sqrt{16 - x^2}$$

				$$[\lim \limits _{a \to x} x ] \dot [\lim \limits _{a \to x } \sqrt{16 - x^2}$$

				$$[\lim \limits _{a \to x} x] \cdot \sqrt{  \lim \limits _{a \to x} (16-x^{2})}$$

				$$[\lim \limits _{a \to x} x] \cdot \sqrt{ \lim \limits _{a \to x} 16 - \lim \limits _{a \to x} x^{2}}$$
				
				$$ a \sqrt{16 - a^2}$$

				So $\lim \limits {x \to a} f(x) = f(a)$

			\item[15-18] Explain why the function is discontinuous at the
				the given number $a$.

			\item[19-24] Explain, using Theorems 4,5,6 and 8, why the function
				is continuous at every number in its domain. State the domain.

			\item $F(x) = \frac{2x^2 - x - 1}{x^2 + 1}$

				By Theorem 5 any rational function is continuous wherever
				it is defined. Because $x^2 + 1$ has no real roots the
				domain is $(-\infty, \infty)$.
				
			\item $G(x) = \sqrt[3]{x}(1+x^3)$

			From Theorem 6, polynomials and root functions are continuous.
			Therefore, $\sqrt[3]{x}$ and $1 + x^3$ are both continuous.

			From Theorem 4, if $f$ and $g$ are continuous at $a$, then
			$fg$ are also continuous at $a$. Therefore $\sqrt[3]{x}(1+x^3)$
			is continuous. 

			The domain is $(-\infty, \infty)$.

			\item $Q(x) = \frac{\sqrt[3]{x-2}}{x^3-2}$

			From Theorem 6 we know polynomials are continuous, therefore
			$x-2$ and $x^3-2$ are both continuous.

			From Theorem 8, we know the composite of two functions at $a$
			is continuous if both functions are continuous at $a$. From 
			Theorem 6 we know root functions are continuous so
			$\sqrt[3]{x-2}$ is continuous.

			From Theorem 4, the quotient of two functions is continuous
			at $a$ if both functions are continuous at $a$ and the denominator
			is not zero.

			So $\frac{\sqrt[3]{x-2}}{x^3-2}$ is continuous.

			The domain is $(-\infty, \sqrt[3]{2})(\sqrt[3]{2}, \infty)$

			\item $h(x) = \frac{\sin x}{x+1}$

			From Theorem 6 we know that polynomials and trigonometric
			functions are continuous. So $\sin x$ and $x+1$ are continuous.

			From theorem 4, if $f$ and $g$ are continuous at $a$, then
			$\frac{f}{g}$ is also continuous at $a$, therefore
			$\frac{\sin x}{x+1}$ is continuous.

			The domain is $(-\infty, -1)(-1, \infty)$.

			\item $M(x) = \sqrt{1 + \frac{1}{x}}$

			From Theorem 5, we know that every rational function is continuous
			wherever it is defined. So $\frac{1}{x}$ is continuous. Constants
			are also continuous, so 1 is continuous.

			From Theorem 4 we know that if $f$ and $g$ are continuous at $a$
			then $f + g$ is continuous at $a$ so $1 + \frac{1}{x}$ is continuous.

			From theorem 6 we know that root functions are continuous across all
			numbers in their domains.

			So $M(x) = \sqrt{1 + \frac{1}{x}}$ is continuous.

			\item $F(x) = \sin(\cos(\sin x))$

			From theorem 6 we know that trigonometric functions are continuous.

			From theorem 8 we know that a continuous function of a continuous function
			is continuous, so $\sin(\cos(\sin x))$ is continuous.

			\item[25-26] Locate the discontinuities of the function and illustrate by
			graphing.

			\item $y = \frac{1}{1 + \sin x}$

			\item $y = \tan \sqrt{x}$

			\item[27-28] Use continuity to evaluate the limit.			

			\item $\lim \limits _{x \to 4} \frac{5 + \sqrt{x}}{\sqrt{5 + x}}$

			$$\lim \limits _{x \to 4} \frac{5 + \sqrt{x}}{\sqrt{5 + x}} = 
				\frac{\lim \limits _{x \to 4} 5 + \sqrt{x}}{\lim \limits _{x \to 4} \sqrt{5 + x}}$$

			Taking the numerator and limit law 1...

			$$\lim \limits _{x \to 4} 5 + \sqrt{x} = \lim \limits _{x \to 4} 5 + \lim \limits _{x \to 4} \sqrt{x}$$

			$$\lim \limits _{x \to 4} 5 +\sqrt{\lim \limits _{x \to 4} x} = 5 + \sqrt{4} = 7$$

			For the denominator we have...

			$$\lim \limits _{x \to 4} \sqrt{5 + x}$$

			From theorem 7 $\lim \limits _{x \to a} f(g(x)) = f(\lim \limits _{x \to a} g(x))$, so

			$$\lim \limits _{x \to 4} \sqrt{5 + x} = \sqrt{\lim \limits _{x \to 4} 5 + x}$$

			$$\sqrt{\lim \limits _{x \to 4} 5 + x } = \sqrt{\lim \limits _{x \to 4} 5 + \lim \limits _{x \to 4} x}
				= \sqrt{5 + 4} = \sqrt{9}  = 3$$

			So putting everything together...

			$$\lim \limits _{x \to 4} \frac{5 + \sqrt{x}}{\sqrt{5 + x}} = \frac{7}{3}$$

			\item $\lim \limits _{x \to \pi} \sin (x + \sin x)$

			From theorem 7 

			$$\lim \limits _{x \to \pi} \sin (x + \sin x) = \sin (\lim \limits _{x \to \pi} x + \sin x)$$

			We can then apply the standard limit laws...

			$$\sin (\lim \limits _{x \to \pi} x + \lim \limits _{x \to \pi} \sin x$$

			Because $\sin$ obeys the direct substition principle 

			$$\lim \limits _{x \to \pi} x + \lim \limits _{x \to \pi} \sin x 
				= \pi + 0 = \pi$$

			We plug this into sin...
		
			$$\sin \pi = 0$$

			\item For what value of the constant $c$ is the function $f$ continuous on $(-\infty, \infty)$?

				$$f(x) = 
				\begin{cases}
					cx^2 + 2x & \text{if } x < 2 \\
					x^3 - cx & \text{if } x \geq 2
				\end{cases}$$

				In order for this function to be continuous $cx^2 + 2x = x^3 - cx$ when $x = 2$.

				$$4c + 4 = 8 - 2c$$
				$$6c = 4$$
				$$c = \frac{2}{3}$$

			\item Find the values of $a$ and $b$ that make $f$ continuous everywhere.

				$$f(x) = 
					\begin{cases}
						\frac{x^2 - 4}{x-2} & \text{if } x < 2 \\
						ax^2 - bx + 3 & \text{if } 2 \leq x < 3 \\
						2x - a + b & \text{if } x \geq 3
					\end{cases}$$

	\end{enumerate}
\end{document}
