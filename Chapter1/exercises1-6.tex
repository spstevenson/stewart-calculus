\documentclass{article}

\renewcommand{\thesection}{} %Remove section numbers

\usepackage{amsmath}

\begin{document}
	\section{1.6 Limits Involving Infinity}

	\begin{enumerate}
		% Q1 - TODO - include the graph
		\item For the function $f$ whose graph is given, state the
			following.

		\begin{enumerate}
			\item $\lim \limits _{x \to \infty} f(x)$

			$$\lim \limits _{x \to \infty} f(x) = -2$$

			\item $\lim \limits _{x \to -\infty} f(x)$

			$$\lim \limits _{x \to -\infty} f(x) = 2$$

			\item $\lim \limits _{x \to 1} f(x)$

			$$\lim \limits _{x \to 1} f(x) = \infty$$

			\item $\lim \limits _{x \to 3} f(x)$

			$$\lim \limits _{x \to 3} f(x) = -\infty$$

			\item The equations of the asymptotes.

			$$y = -2$$

			$$y = 2$$

			$$x = 1$$

			$$x = 3$$
			
		\end{enumerate}

		% Q2 TODO include the graphs

		\item For the function $g$ whose graph is given, state the following.

		\begin{enumerate}
			\item $\lim \limits _{x \to \infty} g(x)$

			$$\lim \limits _{x \to \infty} g(x) = 2$$

			\item $\lim \limits _{x \to -\infty} g(x)$

			$$\lim \limits _{x \to -\infty} g(x) = -1$$

			\item $\lim \limits _{x \to 0} g(x)$

			$$\lim \limits _{x \to 0} g(x) = -\infty$$

			\item $\lim \limits _{x \to 2^{-}} g(x)$

			$$\lim \limits _{x \to 2^{-}} g(x) = -\infty$$

			\item $\lim \limits _{x \to 2^{+}} g(x)$

			$$\lim \limits _{x \to 2^{+}} g(x) = \infty$$

			\item The equations of the asymptotes.

			$$y = -1$$

			$$y = 2$$

			$$x = 0$$

			$$x = 2$$
		\end{enumerate}

		\item[3-8] Sketch the graph of an example of a function $f$ that
			satisfies all the given conditions.

		% Q3 TODO

		\item $\lim \limits _{x \to 0} f(x) = -\infty, \lim \limits _{x \to -\infty} f(x) = 5,
			\lim \limits _{x \to \infty} f(x) = -5$


		% Q4 TODO
		\item $\lim \limits _{x \to 0^{+}} f(x) = \infty, \lim \limits _{x \to 0^{-}} f(x) = -\infty,
			\lim \limits _{x \to \infty} f(x) = 1, \lim \limits _{x \to -\infty} = 1$

		% Q5 TODO
		\item $\lim \limits _{x \to 2} f(x) = -\infty, \lim \limits _{x \to \infty} f(x) = \infty,
			\lim \limits _{x \to -\infty} f(x) = 0, \lim \limits _{x \to 0^{+}} f(x) = \infty,
			\lim \limits _{x \to 0^{-}} f(x) = -\infty$

		% Q6 TODO
		\item $\lim \limits _{x \to \infty} f(x) = 3, \lim \limits _{x \to 2^{-}} f(x) = \infty,
			\lim \limits _{x \to 2^{+}} f(x) = -\infty, f \text{ is odd}$

		% Q7 TODO
		\item $f(0) = 3, \lim \limits _{x \to 0^{-}} f(x) = 4, \lim \limits _{x \to 0^{+}} f(x) = 2,
			\lim \limits _{x \to -\infty} f(x) = -\infty, \lim \limits _{x \to 4^{-}} f(x) = -\infty,
			\lim \limits _{x \to 4^{+}} = \infty, \lim \limits _{x \to \infty} f(x) = 3$

		% Q8 TODO
		\item $\lim \limits _{x \to 3} f(x) = -\infty, \lim \limits _{x \to \infty} f(x) = 2,
			f(0) = 0, f \text{ is even}$

		% Q9
		\item Guess the value of the limit

			$$\lim \limits _{x \to \infty} \frac{x^2}{2^x}$$

		By evaluating the function $f(x) = x^2/2^x$ for $x = 0,1,2,3,4,5,6,7,8,9,10,20,50 \text{ and } 100$.
		Then use a graph of $f$ to support your guess.

		\item[12-33] Find the limit.

		% Q13
		\item $\lim \limits _{x \to -3^{+}} \frac{x+2}{x+3}$

			As $x$ approaches $-3$ from the right the denominator becomes
			a very small negative number whilst the numerator approaches
			$-1$. Thus intuitively...

			$$\lim \limits _{x \to -3^{+}} \frac{x+2}{x+3} = \infty$$

		% Q14
		\item $\lim \limits _{x \to 5^{-}} \frac{6}{x-5}$

			As $x$ approaches $5$ from the left hand side the denominator
			becomes a very small negative number. Thus intuitively...

			$$\lim \limits _{x \to 5^{-}} \frac{6}{x-5} = -\infty$$

		% Q15
		\item $\lim \limits _{x \to 1} \frac{2-x}{(x-1)^2}$

			As $x$ approaches $1$ the denominator becomes extremely
			small whilst the numerator approaches $1$. So intuitively...

			$$\lim \limits _{x \to 1} \frac{2-x}{(x-1)^2} = \infty$$

		% Q16
		\item $\lim \limits _{x \to \pi^{-}} \cot x$

			$$\cot x = \frac{\cos x}{\sin x}$$

			$$\lim \limits _{x \to \pi^{-}} \cos x = -1$$

			$$\lim \limits _{x \to \pi^{-}} \sin x = 0$$

			Because when $x$ goes to $\pi$ $\sin x = 0$ and $\cos x$ is
			negative and not near 0.

			$$\lim \limits _{x \to \pi^{-}} \cot x = -\infty$$

		% Q17
		\item $\lim \limits _{x \to 2\pi^{-}} x \csc x$

			An equivalent expression is

			$$x \csc x = \frac{x}{\sin x}$$

			As $x \to 2\pi^{-}$ then $\sin x$ goes to zero, while
			$x$ goes to $2\pi$, a positive number not approaching zero.

			Therefore.

			$$\lim \limits _{x \to 2\pi^{-}} x \csc x = \infty$$

		% Q18
		\item $\lim \limits _{x \to 2^{-}} \frac{x^2 - 2x}{x^2 - 4x + 4}$

			Lets start by simplifying the expression.

			$$\lim \limits _{x \to 2^{-}} \frac{x}{x-2}$$

			As $x \to 2^{-}$, $x -2$ becomes a very small negative number.
			$x$ goes to 2, a positive number not approaching 0. Thus

			$$\lim \limits _{x \to 2^{-}} \frac{x^2 - 2x}{x^2 - 4x + 4} = -\infty$$
	\end{enumerate}
\end{document}
