\documentclass{article}

\renewcommand{\thesection}{}

\usepackage{amsmath}
\usepackage{gensymb}

\begin{document}

	\section{2.1 Exercises}

	\begin{enumerate}
		% TODO: Q1
		\item
		\begin{enumerate}
			\item Find the slope of the tangent line to the parabola
				$y = 4x - x^2$ at the point $(1,3)$

			\begin{enumerate}
				\item using Definition 1

				\item using Equation 2
			\end{enumerate}

			\item Find an equation of the tangent line in part (a).

			\item Graph an equation of the tangent line in part (a).
		\end{enumerate}

		% TODO: Q2
		\item
		\begin{enumerate}
			\item Find the slope of the tangent line to the curve $y = x^3$ at
				the point $(-1,-1)$.

			\begin{enumerate}
				\item using Definition 1

				\item using Definition 2
			\end{enumerate}

			\item Find an equation of the tangent line in part (a).

			\item Graph the curve and the tangent line in successively smaller
				viewing rectangles centered at $(-1,-1)$ until the curve and
				the line appear to coincide.
		\end{enumerate}

		\item[3-6] Find an equation of the tangent line to the curve at the given point.
		% Q3
		\item $y = 4x - 3x^2, (2, -4)$

			$$\lim \limits _{x \to 2} \frac{4x - 3x^2 - (-4)}{x-2}$$

			$$\lim \limits _{x \to 2} \frac{4x - 3x^2 + 4}{x-2}$$

			$$\lim \limits _{x \to 2} \frac{-(3x^2 - 4x - 4}{x-2}$$

			$$\lim \limits _{x \to 2} \frac{-(3x+2)(x-2)}{x-2}$$

			$$\lim \limits _{x \to 2} -3x-2$$

			$$m = -3(2) - 2 = -8$$

			The equation of the tangent line would be

			$$y - (-4) = -8(x - 2)$$

			$$y + 4 = -8x + 16$$
			
			$$y = -8x - 12$$

		% Q4
		\item $y = 2x^3 - 5x, (-1, 3)$

			$$m = \lim \limits _{h \to 0} \frac{2(-1+h)^3 - 5(-1+h) - (2(-1)^3 - 5(-1))}{h}$$

			$$m = \lim \limits _{h \to 0} \frac{2(-1+h)^3 - 5(-1 + h) - 3}{h}$$

			$$m = \lim \limits _{h \to 0} \frac{2(h^3 - 3h^2 + 3h -1) -3 + 5 - 5h}{h}$$

			$$m = \lim \limits _{h \to 0} \frac{2h^3 - 6h^2 + 6h -2 + 2 - 5h}{h}$$

			$$m = \lim \limits _{h \to 0} \frac{2h^3 - 6h^2 + 6h - 5h}{h}$$

			$$m = \lim \limits _{h \to 0} \frac{2h^3 - 6h^2 + h}{h}$$

			$$m = \lim \limits _{h \to 0} 6h + 1 = 1$$

			So we find the equation by

			$$y - 3 = x - (-1)$$

			$$y - 3 = x + 1$$

			$$y = x + 4$$
		% Q5
		\item $y = \sqrt{x}, (1,1)$

			$$\lim \limits _{x \to a} \frac{f(x) - f(a)}{x - a}$$

			$$\lim \limits _{x \to 1} \frac{\sqrt{x} - 1}{x-1}$$

			$$\frac{\sqrt{x} - 1}{x-1} \times \frac{\sqrt{x} + 1}{\sqrt{x} + 1} = 
				\frac{(\sqrt{x} - 1)(\sqrt{x} + 1)}{(x-1)(\sqrt{x}+1)}$$

			$$(\sqrt{x}-1)(\sqrt{x}+1) = x-1$$

			$$\frac{x-1}{(x-1)(\sqrt{x}+1)} = \frac{1}{\sqrt{x}+1}$$

			$$\lim \limits _{x \to 1} \frac{1}{\frac{x} + 1} = \frac{1}{2}$$

			$$y - 1 = \frac{1}{2}(x-1)$$

			$$y = \frac{1}{2}x + \frac{1}{2}$$

		% Q6
		\item $y = \frac{2x + 1}{x+2}, (1,1)$

			$$m = \lim \limits _{x \to a} \frac{f(x) - f(a)}{x -a}$$

			$$m = \lim \limits _{x \to 1} \frac{(2x+1) - (2(1)+1)}{x-1}$$

			$$m = \lim \limits _{x \to 1} \frac{(2x+1) - 3}{x-1}$$

			$$m = \lim \limits _{x \to 1} \frac{2x-2}{x-1}$$

			$$m = \lim \limits _{x \to 1} \frac{2(x-1)}{x-1} = 2$$

			So the equation is...

			$$y - 1 = 2(x - 1)$$

			$$y - 1 = 2x - 2$$

			$$y = 2x - 1$$

		% TODO Q7
		\item 
			\begin{enumerate}
				\item Find the slope of the tangent to the curve $y = 3 + 4x^2 - 2x^3$
					at the point where $x = a$

				\item Find equations of the tangent lines at the points $(1,5)$ and
					$(2,3)$.

				\item Graph the curve and both tangents on a common screen.
			\end{enumerate}
		% TODO Q8
		\item 
		\begin{enumerate}
			\item Find the slope of the tangent to the curve $y = 1/\sqrt{x}$ at the point
				where $x=a$

			\item Find the equations of the tangent lines at the points $(1,1)$ and $(4, \frac{1}{4})$.

			\item Graph the curve and both tangents on a common screen.
		\end{enumerate}
		% TODO Q9
		\item The graph shows the position function of a car. Use the shape of the graph to explain your
			answers to the following questions.
		\begin{enumerate}
			\item What was the initial velocity of the car?

			\item Was the car going faster at $B$ or $C$?

			\item Was the car slowing down or speeding up at $A,B$ and $C$?

			\item What happened between $D$ and $E$?
		\end{enumerate}
		% TODO Q10
		\item Shown are graphs of the position of two runners, A and B, who run a
			100-m race and finish in a tie.

		\begin{enumerate}
			\item Describe and compare how the runners run the race.

			\item At what time is the distance between the runners the greatest?

			\item At what time do they have the same velocity?
		\end{enumerate}

		% Q11
		\item If a ball is thrown into the air with a velocity of 10 m/s, its height
			(in meters) after $t$ seconds is given by $y = 10t - 4.9t^2$. Find the
			velocity when $t = 2$.

			$$v = \lim \limits _{h \to 0} = \frac{f(t+h) - f(t)}{h}$$
			$$v = \lim \limits _{h \to 0} = \frac{10(t+h) - 4.9(t+h)^2 - (10t - 4.9t^2)}{h}$$
			$$v = \lim \limits _{h \to 0} = \frac{10t + 10h - 4.9t^2 - 9.8th - 4.9h^2 - 10t + 4.9t^2}{h}$$
			$$v = \lim \limits _{h \to 0} = \frac{10h - 9.8th - 4.9h}{h}$$
			$$v = \lim \limits _{h \to 0} = -9.8t - 4.9 h = -9.8t$$

			$$v(t) = -9.8t$$
			$$v(2) = -19.6$$

		% Q12
		\item If an arrow is shot upward on the moon with a velocity of 58 m/s, its height (in meters)
			after $t$ seconds is given by $H = 58t - 0.83t^2$.

		\begin{enumerate}
			\item Find the velocity of the arrow after 1 second.

				$$v = \lim \limits _{h \to 0} \frac{58(t + h) - 0.83(t+h)^2 - (58t - 0.83t^2)}{h}$$

				$$v = \lim \limits _{h \to 0} \frac{58(t+h) - 0.83(t^2+th+h^2)-58t+0.83t^2}{h}$$

				$$v = \lim \limits _{h \to 0} \frac{58t+58h-0.83t^2-0.83th-0.83h^2-58t+0.83t^2}{h}$$

				$$v = \lim \limits _{h \to 0} \frac{58h - 0.83th-0.83h^2}{h}$$

				$$v = \lim \limits _{h \to 0} 58 - 0.83t -0.83h$$

				$$v(t) = 58 - 0.83t$$

				So

				$$v(1) = 58 - 0.83(1) = 57.17 \text{m/s}$$

			\item Find the velocity of the arrow when $t=a$

				$v(a) = 58-0.83a$

			\item When will the arrow hit the moon?

				The velocity of the arrow when its the moon will be the negation of its initial
				velocity so we must solve...

				$$58 - 0.83t = -58$$ 
				$$-0.83t = -58 -58$$
				$$0.83t = 58+58$$
				$$t = \frac{58+58}{0.83}$$
				$$t = \frac{116}{0.83}$$
				$$t = 139.76 \text{ seconds}$$
		\end{enumerate}

		% Q13
		\item The displacement (in meters) of a particle moving in a straight line is given by the equation
			of motion $s = 1/t^2$, where $t$ is measured in seconds. Find the velocity of the particle at
			times $t=a,t=1,t=2$ and $t=3$.

			$$v(t) = \lim \limits _{h \to 0} \frac{ \frac{1}{(t + h)^2} - \frac{1}{t^2} }{h}$$

			$$v(t) = \lim \limits _{h \to 0} \frac{t^2 - (t+h)^2}{ht^2(t+h)^2}$$

			$$v(t) = \lim \limits _{h \to 0} \frac{t^2 - (t^2 + 2th + h^2)}{ht^2(t+h)^2}$$

			$$v(t) = \lim \limits _{h \to 0} \frac{2th+h^2}{ht^2(t+h)^2} = \frac{2t}{t^2t^2}$$

			$$v(t) = \frac{2}{t^3}$$

			So now we can plug in our values...

			$$v(a) = \frac{2}{a^3}$$
			$$v(1) = 2$$
			$$v(2) = \frac{1}{4}$$
			$$v(3) = \frac{2}{27}$$

		% TODO: Q14
		\item The displacement (in meters) of a particle moving in a straight line is given 
			by $s = t^2 - 8t + 18$, where $t$ is measured in seconds.

		\begin{enumerate}
			\item Find the average velocity over each time interval:

			\begin{enumerate}
				\item $[4,3]$

				\item $[3.5,4]$

				\item $[4,5]$

				\item $[4,4.5]$
			\end{enumerate}

			\item Find the instantaneous velocity when $t = 4$.

			\item Draw the graph of $s$ as a function of $t$ and draw the secant lines
				whose slopes are the average velocities in part (a) and the tangent
				line whose slope is the instantaneous velocity in part (b).
		\end{enumerate}

		% TODO: Q15
		\item For the function $g$ whose graph is given, arrange the following numbers in
			increasing order and explain your reasoning:

			\begin{tabular}{ccccc}
				$0$ & $g'(-2)$ & $g'(0)$ & $g'(2)$ & $g'(4)$
			\end{tabular}

		% Q16
		\item Find an equation of the tangent line to the graph of $y = g(x)$ at $x = 5$ if
			$g(5) = -3$ and $g'(5) = 4$.

			The gradient of the graph at $g(5)$ is $4$ as $g'(5) = 4$. So $m = 4$.

			So

			$$y - y_1 = 4(x - x_1)$$

			Because $g(5) = -3$

			$$y - (-3) = 4(x - 5)$$

			$$y + 3 = 4x - 20$$

			$$y = 4x - 23$$

		% Q17
		\item If an equation of the tangent line to the curve $y = f(x)$ at the point where
			$a=2$ is $y = 4x-5$, find $f(2)$ and $f'(2)$

			Because the tangent line is equal to $f(x)$ at 2, $f(2) = 3$. $f'(2)$ is equal
			to the gradient of the tangent at 2 so $f'(2) = 4$.

		% Q18
		\item If the tangent to $y = f(x)$ at $(4,3)$ passes through the point $(0,2)$, find $f(4)$ and
			$f'(4)$.

			$$f(4) = 3$$

			$$m = \frac{3 - 2}{4 - 0} = \frac{1}{4}$$

			So $f'(4) = \frac{1}{4}$

		% TODO: Q19
		\item Sketch the graph of a function $f$ for which $f(0) = 0, f'(0) = 3, f'(1) = 0$, and
			$f'(2) = -1$.

		% TODO: Q20
		\item Sketch the graph of a function $g$ for which
			$g(0) = g(2) = g(4) = 0, g'(1) = g'(3) = 0,$
			$g'(0) = g'(4) = 1, g'(2) = -1, \lim _{x \to \infty} g(x) = \infty,$
			and $\lim _{x \to -\infty} g(x) = -\infty$.

		% Q21
		\item If $f(x) = 3x^2 - x^3,$ find $f'(1)$ and use it to find an equation of the tangent
			line to the curve $y = 3x^2 - x^3$ at the point $(1,2)$.

			$$\lim \limits _{h \to 0} \frac{3(x+h)^2 - (x+h)^3 - [3x^2-x^3]}{h}$$

			$$\lim \limits _{h \to 0} \frac{3(x^2+2xh+h^2)-(x+h)(x^2+2xh+h^2) - [3x^2-x^3]}{h}$$

			$$\lim \limits _{h \to 0} \frac{3(x^2+2xh+h^2)-(x^3+3x^2h+3xh^2+h^3) - [3x^2-x^3]}{h}$$

			$$\lim \limits _{h \to 0} \frac{3x^2+6xh+3h^2-x^3-3x^2h-3xh^2-h^3-3x^2+x^3}{h}$$

			$$\lim \limits _{h \to 0} \frac{6xh+3h^2-3x^2h-3xh^2+h^2}{h}$$

			$$\lim \limits _{h \to 0} 6x + 3h - 3x^2 - 3xh + h^2 = 6x - 3x^2$$

			$$f'(1) = 6(1)-3(1)^2 = 6-3 = 3$$

			$$y - 2 = 3(x - 1)$$

			$$y - 2 = 3x - 3$$

			$$y = 3x - 1$$

		% Q22
		\item If $g(x) = 1 - x^3$, find $g'(0)$ and use it to find an equation of the tangent to the 
			curve $y = 1 - x^3$ at the point $(0,1)$.

		$g(x) = 1 - x^3$ and we need to find $g'(0)$

		$$g'(0) = \lim \limits _{h \to 0} \frac{g(0+h) - g(0)}{h}$$

		$$g'(0) = \lim \limits _{h \to 0} \frac{1 - h^3 - 1}{h}$$

		$$g'(0) = \lim \limits _{h \to 0} h^2 = 0$$

		$$y - 1 = 0(x - 0)$$

		$$y = 1$$

		% TODO: Q23
		\item 
		\begin{enumerate}
			\item If $F(x) = 5x/(1 + x^2)$, find $F'(2)$ and use it to find and equation of the tangent
				line to the curve

				$$y = \frac{5x}{1 + x^2}$$

				at the point $(2,2)$.

			\item Illustrate part (a) by graphing the curve and the tangent line on the same screen.
		\end{enumerate}

		% TODO: Q24
		\item
		\begin{enumerate}
			\item If $G(x) = 4x^2 - x^3$, find $G'(a)$ and use it to find equations of the tangent lines
				to the curve $y = 4x^2 - x^3$ at the points $(2,8)$ and $(3,9)$.

			\item Illustrate part (a) by graphing the curve and the tangent lines on the same screen.
		\end{enumerate}

		\item[25-30] Find $f'(a)$

		% Q25
		\item $f(x) = 3x^2 - 4x + 1$
			
			$$f'(a) = \lim \limits _{h \to 0} \frac{f(a+h) - f(a)}{h}$$

			$$f'(a) = \lim \limits _{h \to 0} \frac{3(a+h)^2 - 4(a+h) + 1 - (3a^2 - 4a + 1)}{h}$$

			$$f'(a) = \lim \limits _{h \to 0} \frac{3(a^2+2ah+h^2) - 4a - 4h + 1 - 3a^2 + 4a - 1}{h}$$

			$$f'(a) = \lim \limits _{h \to 0} \frac{3a^2 + 6ah + 3h^2 - 4a - 4h + 1 - 3a^2 + 4a - 1}{h}$$

			$$f'(a) = \lim \limits _{h \to 0} \frac{6ah + 3h^2 - 4h}{h}$$

			$$f'(a) = \lim \limits _{h \to 0} 6a + 3h - 4$$

			$$f'(a) = 6a - 4$$

		% Q26
		\item $f(t) = t^4 - 5t$

			$$f'(a) = \lim \limits _{h \to 0} \frac{(a+h)^4 - 5(a+h) - (a^4 - 5a)}{h}$$

			$$f'(a) = \lim \limits _{h \to 0} \frac{a^4+4a^3h+6a^2h^2+4ah^3+h^4-5a-5h-a^4+5a}{h}$$

			$$f'(a) = \lim \limits _{h \to 0} \frac{4a^3h + 6a^2h^2 + 4ah^3 + h^4 - 5h}{h}$$

			$$f'(a) = \lim \limits _{h \to 0} 4a^3 + 6a^2h + 4ah^2 + h^3 - 5 = 4a^3 - 5$$

		% Q27
		\item $f(t) = \frac{2t+1}{t+3}$

			$$f'(a) = \lim \limits _{h \to 0} \frac{ \frac{2(a+h)+1}{(a+h)+3} - \frac{2a+1}{a+3}}{h}$$

			$$f'(a) = \lim \limits _{h \to 0} \frac{ \frac{2a + 2h + 1}{a + h + 3} - \frac{2a+1}{a+3} }{h}$$

			$$f'(a) = \lim \limits _{h \to 0} \frac{(2a+2h+1)(a+3) - (a+h+3)(2a+1)}{(a+h+3)(a+3)h}$$

			$$f'(a) = \lim \limits _{h \to 0} \frac{2a^2+2ah+a+6a+6h+3-2a^2-2ah-6a-a-h-3}{h(a+h+3)(a+3)}$$

			$$f'(a) = \lim \limits _{h \to 0} \frac{6h-h}{h(a+h+3)(a+3)}$$

			$$f'(a) = \lim \limits _{h \to 0} \frac{5}{(a+h+3)(a+3)} = \frac{5}{(a+3)^2}$$

		% Q28
		\item $f(x) = x^{-2}$

			$$f'(a) = \lim \limits _{h \to 0} \frac{(a+h)^{-2} - a^{-2}}{h}$$

			$$f'(a) = \lim \limits _{h \to 0} \frac{ \frac{1}{(a+h)^2} - \frac{1}{a^2} }{h}$$

			$$f'(a) = \lim \limits _{h \to 0} \frac{ a^2 - (a+h)^2 }{ha^2(a+h)^2}$$

			$$f'(a) = \lim \limits _{h \to 0} \frac{a^2 - (a^2+2ah+h^2)}{ha^2(a+h)^2}$$

			$$f'(a) = \lim \limits _{h \to 0} \frac{-2ah-h^2}{ha^2(a+h)^2}$$

			$$f'(a) = \lim \limits _{h \to 0} \frac{-2a-h}{a^2(a+h)^2} = - \frac{2a}{a^4} = - \frac{2}{a^3}$$			

		% Q29
		\item $f(x) = \sqrt{1-2x}$

			$$f'(a) = \lim \limits _{h \to 0} \frac{\sqrt{1-2(a+h)} - \sqrt{1-2a}}{h}$$

			$$f'(a) = \lim \limits _{h \to 0} \frac{\sqrt{1-2a-2h} - \sqrt{1-2a}}{h}$$

			We multiply this by $\frac{ \sqrt{1-2a-2h} + \sqrt{1-2a} }{ \sqrt{1-2a-2h} + \sqrt{1-2a}}$

			$$f'(a) = \lim \limits _{h \to 0} \frac{(1-2a-2h) - (1-2a)}{h(\sqrt{1-2a-2h} + \sqrt{1-2a}}$$

			$$f'(a) = \lim \limits _{h \to 0} \frac{-2h}{h(\sqrt{1-2a-2h} + \sqrt{1-2a}}$$

			$$f'(a) = \lim \limits _{h \to 0} \frac{-2}{\sqrt{1-2a-2h} + \sqrt{1-2a}}$$

			$$f'(a) = - \frac{1}{\sqrt{1-2a}}$$

		% Q30
		\item $f(x) = \sqrt{3x+1}$

			$$f'(a) = \lim \limits _{h \to 0} \frac{\sqrt{3(a+h)+1} - \sqrt{3a+1}}{h}$$

			$$f'(a) = \lim \limits _{h \to 0} \frac{(3(a+h)+1) - (3a+1)}{h(\sqrt{3(a+h)+1} + \sqrt{3a+1}}$$

			$$f'(a) = \lim \limits _{h \to 0} \frac{3a + 3h  + 1 - 3a - 1}{h(\sqrt{3(a+h)+1}+\sqrt{3a+1}}$$

			$$f'(a) = \lim \limits _{h \to 0} \frac{3h}{h(\sqrt{3(a+h)+1} + \sqrt{3a+1}}$$

			$$f'(a) = \lim \limits _{h \to 0} \frac{3}{\sqrt{3(a+h)+1} + \sqrt{3a+1}}$$

			$$f'(a) = \frac{3}{2\sqrt{3a+1}}$$

		\item[31-36] Each limit represents the derivative of some function $f$ at some number $a$. State
				such an $f$ and $a$ in each case.

		% Q31
		\item $\lim \limits _{h \to 0} \frac{(1+h)^{10} - 1}{h}$

			$$f(x) = x^{10}$$

			$$a = 1$$

		% Q32
		\item $\lim \limits _{h \to 0} \frac{\sqrt[4]{16+h} - 2}{h}$

			$$f(x) = \sqrt[4]{x}$$

			$$a = 16$$

		% Q33
		\item $\lim \limits _{x \to 5} \frac{2^x - 32}{x-5}$

			$$f(x) = 2^x$$

			$$a = 5$$

		% Q34
		\item $\lim \limits _{x \to \pi/4} \frac{\tan x - 1}{x - \pi/4}$

			$$f(x) = \tan x$$

			$$a = \pi/4$$

		% Q35
		\item $\lim \limits _{h \to 0} \frac{ \cos (\pi + h) + 1}{h}$

			$$f(x) = \cos x$$

			$$a = 0$$

		% Q36
		\item $\lim \limits _{t \to 1} \frac{t^4 + t - 2}{t-1}$

			$$f(x) = t^4 + t$$

			$$a = 1$$

		% TODO: Q37
		\item A warm can of soda is placed in a cold refrigerator. Sketch the
			graph of the temperature of the soda as a functionof time. Is
			the initial rate of change of temperature greater or less than
			the rate of change after an hour?

		% TODO: Q38
		\item A roast turkey is taken from an oven when its temperature has reached
			85\degree C and is placed on a table in a room where the temperature is 
			24\degree C. The graph shows how the temperature of the turkey decreases and
			eventually approaches room temperature. By measuring the slope of
			the tangent, estimate the rate of change of the temperature after an
			hour.

		% Q39
		\item The table shows the number of passengers $P$ that arrived in Ireland by air,
			in millions.

		\begin{tabular}{|c|c|c|c|c|c|}
			\hline Year & 2001 & 2003 & 2005 & 2007 & 2009 \\
			\hline $P$ & 8.49 & 9.65 & 11.78 & 14.54 & 12.84 \\
			\hline
		\end{tabular}	

		\begin{enumerate}

			\item Find the average rate of increase of $P$

			\begin{enumerate}
				\item from 2001 to 2005

				$$\frac{11.78 - 8.49}{2005 - 2001}$$
				$$\frac{3.29}{4} = 0.8225$$

				\item from 2003 to 2005

				$$\frac{11.78 - 9.65}{2005 - 2003}$$
				$$\frac{2.13}{2} = 1.065$$

				\item from 2005 to 2007

				$$\frac{14.54 - 11.78}{2007 - 2005}$$
				$$\frac{2.76}{2} = 1.38$$
			\end{enumerate}

			\item Estimate the instantaneous rate of growth in 2005 by
				taking the average of two average rates of change.
				What are its units?

					$$\frac{1.065 + 1.38}{2}$$
					$$\frac{2.445}{2} = 1.2225$$

				The units would be $P Y^{-1}$, or passengers per year.

		\end{enumerate}	

		% Q40
		\item The number $N$ of locations of a popular coffeehouse chain is given
			in the table. (The number of locations as of October 1 are given.)

			\begin{tabular}{|c|c|c|c|c|c|}
				\hline
				Year & 2004 & 2005 & 2006 & 2007 & 2008 \\ \hline
				$N$ & 8569 & 10,241 & 12,440 & 15,011 & 16,680 \\ \hline
			\end{tabular}

			\begin{enumerate}
				\item Find the average rate of growth

				\begin{enumerate}
					\item from 2006 to 2008

						$$\frac{16680 - 12440}{2008 - 2006}$$
						$$\frac{4240}{2} = 2120$$

						2190 stores per year

					\item from 2006 to 2007

						$$\frac{15011 - 12440}{2007 - 2006}$$
						$$\frac{2571}{1} = 2571$$

						2571 stores per year
					\item from 2005 to 2006

						$$\frac{12440 - 10241}{2006 - 2005}$$
						$$\frac{2199}{1} = 2199$$

						2199 stores per year
				\end{enumerate}

				\item Estimate the instantaneous rate of growth in 2006 by
					taking the average of two average rates of change.
					What are its units?

					$$\frac{2571 + 2199}{2} = \frac{4770}{2} = 2385$$

					2385 stores per year

				% TODO part c
				\item Estimate the instantaneous rate of growth in 2006 by measuring the slope of a tangent.

				% TODO: part d
				\item Estimate the instantaneous rate of growth in 2007 and compare it 
					with the growth rate in 2006. What do you conclude?
			\end{enumerate}

			% Q42
			\item If a cylindrical tank holds 100,000 liters of water, which can be drained
				from the bottom of the tank in an hour, then Toricelli's Law gives the
				volume $V$ of water remaining in the tank after $t$ minutes as

				$$V(t) = 100000(1 - \frac{1}{60}t)^2, \text{	} 0 \le t \le 60$$

				Find the rate at which the water is flowing out of the tank (the instantaneuous
				rate of change of $V$ with respect to $t$) as a function of $t$. What are
				its units? For times $t = 0, 10, 20, 30, 40, 50, \text{ and } 60 \text{ min}$,
				find the flow rate and the amount of water remaining in the tank. Summarize
				your findings in a sentence or two. At what time is the flow rate the greatest?
				The least?

				Firstly lets expand out $100000(1 - \frac{1}{60}t)^2$

				$$100000(1 - \frac{t}{30} + (\frac{1}{60}t)^2)$$

				$$100000(1 - \frac{t}{30} + \frac{t^2}{3600})$$

				$$100000(1 + \frac{3t^2 - 360t}{10800})$$
				$$100000 + \frac{100000(3t^2 - 360t)}{10800}$$

				$$100000 + \frac{1000(3t^2 - 360t)}{108}$$

				$$100000 + \frac{3000t^2 - 360000t}{108}$$

				So if we do $f(t+h) - f(t)$

				$$100000 + \frac{3000(t+h)^2 - 360000(t+h)}{108} - (100000 + \frac{3000t^2 - 360000t}{108}$$

				$$\frac{3000(t+h)^2 - 360000(t+h)}{108} - \frac{3000t^2 - 360000t}{108}$$

				So our limit is

				$$f'(t) = \lim \limits _{h \to 0} \frac{3000(t+h)^2 - 360000(t+h) + 360000t - 3000t^2}{108h}$$

				$$f'(t) = \lim \limits _{h \to 0} \frac{3000t^2 + 6000th + 3000h^2 - 360000t - 360000h + 360000t - 3000t^2}{108h}$$

				$$f'(t) = \lim \limits _{h \to 0} \frac{6000th + 3000h^2 - 360000h}{108h}$$

				$$f'(t) = \lim \limits _{h \to 0} \frac{6000t + 3000h - 360000}{108}$$

				$$f'(t) = \frac{6000t - 360000}{108}$$

				\begin{tabular}{|c|c|}
				
					\hline Time (mins) & Litres / min \\
					\hline 0 & -3333.3 \\
					\hline 10 & -2777.8 \\
					\hline 20 & -2222.2 \\
					\hline 30 & -1666.7 \\
					\hline 40 & -1111.1 \\
					\hline 50 & -555.6 \\
					\hline 60 & 0 \\
					\hline
				\end{tabular}

				As we can see the flow is getting smaller with respect to time. The
				flow is greatest at $t = 0$. The flow is the least when $t = 60$ (when the
				tank is empty and $f'(t) = 0$

			\item The cost of producing $x$ kilograms of gold from a new gold mine
				is $C= f(x)$ dollars.

			\begin{enumerate}
				\item What is the meaning of the derivative $f'(x)$? What are its units?

					$f'(x)$ is the instantaneous rate of change of the function $f(x)$ at
					the point $x$. In this case it is the marginal cost of the gold mine.
					Its units are dollars per kilogram.

				\item What does the statement $f'(50) = 36$ mean?

					This means that the instantaneous rate of change at $x = 50$ is
					36 dollars per kilogram. So when the mine is producing 50 kilograms
					of gold, then the cost per kilogram of producing more gold is 
					36 dollars.

				% TODO
				\item Do you think the values of $f'(x)$ will increase or decrease in the
					short term? What about the long term? Explain.
			\end{enumerate}

			\item The number bacteria after $t$ hours in a controlled laboratory experiment
				is $n = f(t)$.

			\begin{enumerate}
				\item What is the meaning of the derivative $f'(5)$? What are its units?

			
			\end{enumerate}

	\end{enumerate}

\end{document}
