\documentclass{article}

\renewcommand{\thesection}{}

\usepackage{amsmath}

\begin{document}

	\section{2.2 Exercises}

	\begin{enumerate}

		\item[1-2] Use the given graph to estimate the value of each derivative.
			Then sketch the graph of $f'$.

		% TODO: Q1
		\item
		\begin{enumerate}
			\item $f'(-3)$

			\item $f'(-2)$

			\item $f'(0)$

			\item $f'(1)$

			\item $f'(2)$

			\item $f'(3)$
		\end{enumerate}

		% TODO: Q2
		\item
		\begin{enumerate}
			\item $f'(0)$

			\item $f'(1)$

			\item $f'(2)$

			\item $f'(3)$

			\item $f'(4)$

			\item $f'(5)$
		\end{enumerate}

		% TODO: Q3
		\item Match the graph of each function in (a)-(d) with the graph
			of its derivative in I-IV. Give reasons for your choices.

		\item[4-11] Trace or copy the graph of the given function $f$. (Assume
				that the axes have equal scales.) Then use the method
				of Example 1 to sketch the graph of $f'$ below it.

		% TODO: Q4
		\item

		% TODO: Q5
		\item

		% TODO: Q6
		\item

		% TODO: Q7
		\item

		% TODO: Q8
		\item

		% TODO: Q9
		\item

		% TODO: Q10
		\item

		% TODO: Q11
		\item

		% TODO: Q12
		\item Shown is the graph of the population function $P(t)$ for yeast cells
			in a laboratory culture. Use the method of Example 1 to graph the
			derivative $P'(t)$. What does the graph of $P'$ tell us about the
			yeast population?

		% TODO: Q13
		\item A rechargeable battery is plugged into a charger. The graph shows $C(t)$,
			the percentage of full capacity that the battery reaches as a function
			of time $t$ elapsed (in hours).

		\begin{enumerate}
			\item What is the meaning of the derivative $C'(t)$?

			\item Sketch the graph of $C'(t)$. What does the graph tell you?
		\end{enumerate}

		% TODO: Q14
		\item The graph (from the US Department of Energy) shows how driving speed affects
			gas mileage. Fuel economy $F$ is measured in miles per gallon and speed $v$
			is measured miles per hour.

		\begin{enumerate}
			\item What is the meaning of the derivative $F'(v)$?

			\item Sketch the graph of $F'(v)$.

			\item At what speed should you drive if you want to save on gas?
		\end{enumerate}

		% TODO: Q15
		\item The graph shows how the average age of first marriage of Japanese men in
			the last half of the 20th century. Sketch the graph of the derivative function
			$M'(t)$. During which years was the derivative negative?

		% TODO: Q16
		\item Make a careful sketch of the graph of the sine function and below it sketch
			the graph of its derivative in the same manner as in Exercises 4-11. Can you
			guess what the derivative of the sine function is from its graph.

		% TODO: Q17
		\item Let $f(x) = x^2$

		\begin{enumerate}
			\item Estimate the values of $f'(0), f'(\frac{1}{2}), f'(1).$ and $f'(2)$ by
			using a graphing device to zoom in on the graph of $f$.

			\item Use symmetry to deduce the values of $f'(-\frac{1}{2}), f'(1),$ and
				$f'(-2)$.

			\item Use the results from parts (a) and (b) to guess a formula for $f'(x)$.
		\end{enumerate}

		% TODO: Q18
		\item Let $f(x) = x^3$

		\begin{enumerate}
			\item Estimate the values of $f'(0), f'(\frac{1}{2}), f'(1), f'(2),$ and
			$f'(3)$ by using a graphing device to zoom in on the graph of $f$.

			\item Use symmetry to deduce the values of $f'(-\frac{1}{2}), f'(-1), f'(-2),$
				and $f'(-3)$.

			\item Use the values from parts (a) and (b) to graph $f'$.

			\item Use the definition of a derivative to prove your guess in part (d) is correct.
		\end{enumerate}

		\item[19-27] Find the derivative of the function using the definition of derivative. State the
				domain of the function and the domain of its derivative.

		% Q19
		\item $f(x) = \frac{1}{2}x - \frac{1}{3}$

			$$f'(x) = \lim \limits _{h \to 0} \frac{ (\frac{1}{2}(x+h) - \frac{1}{3}) - (\frac{1}{2}x + \frac{1}{3}) }{h}$$

			$$f'(x) = \lim \limits _{h \to 0} \frac{ \frac{1}{2}(x+h) - \frac{1}{2}x }{h}$$

			$$f'(x) = \lim \limits _{h \to 0} \frac{ \frac{1}{2}x + \frac{1}{2}h - \frac{1}{2}x}{h}$$

			$$f'(x) = \lim \limits _{h \to 0} \frac{ \frac{1}{2}h }{h}$$

			$$f'(x) = \lim \limits _{h \to 0} \frac{1}{2} = \frac{1}{2}$$

			The domain of $f(x)$ is $(-\infty, \infty)$. The domain of $f'(x)$ is $(-\infty, \infty)$.

		% Q20
		\item $f(x) = 1.5x^2 - x + 3.7$

			$$f'(x) = \lim \limits _{h \to 0} \frac{(1.5(x+h)^2 - (x+h) + 3.7) - (1.5x^2 - x + 3.7)}{h}$$

			$$f'(x) = \lim \limits _{h \to 0} \frac{1.5(x^2 + 2xh + h^2)-x-h+3.7 - 1.5x^2 + x - 3.7}{h}$$

			$$f'(x) = \lim \limits _{h \to 0} \frac{1.5x^2 + 3xh + 1.5h^2 - x - h + 3.7 - 1.5x^2 - x + 3.7}{h}$$

			$$f'(x) = \lim \limits _{h \to 0} \frac{3xh + 1.5h^2 - h}{h}$$

			$$f'(x) = \lim \limits _{h \to 0} 3x + 1.5h - 1 = 3x - 1$$

			The domain of $f(x)$ is $(-\infty, \infty)$. The domain of $f'(x)$ is $(-\infty, \infty)$.

		% Q21
		\item $f(x) = x^3 - 3x + 5$

			$$f'(x) = \lim \limits _{h \to 0} \frac{(x+h)^3 - 3(x+h) + 5 - (x^3-3x+5)}{h}$$

			$$f'(x) = \lim \limits _{h \to 0} \frac{x^3 + 3x^2h + 3xh^2 + h^3 - 3x - 3h + 5 - x^3 + 3x - 5}{h}$$

			$$f'(x) = \lim \limits _{h \to 0} \frac{3x^2h +3xh^2 + h^3 - 3h}{h}$$

			$$f'(x) = \lim \limits _{h \to 0} 3x^2 + 3xh + h^2 - 3 = 3x^2 - 3$$

			The domain of $f(x)$ is $(-\infty, \infty)$. The domain of $f'(x)$ is $(-\infty, \infty)$.

		% Q22
		\item $f(x) = x + \sqrt{x}$

			$$f'(x) = \lim \limits _{h \to 0} \frac{ [(x+h) + \sqrt{x+h}] - (x + \sqrt{x})}{h}$$

			$$f'(x) = \lim \limits _{h \to 0} \frac{x + h + \sqrt{x+h} - x - \sqrt{x}}{h}$$

			$$f'(x) = \lim \limits _{h \to 0} \frac{\sqrt{x+h} - \sqrt{x} + h}{h}$$

			$$f'(x) = \lim \limits _{h \to 0} \frac{\sqrt{x+h} - \sqrt{x}}{h} + 1$$

			$$f'(x) = \lim \limits _{h \to 0} \frac{x + h - x}{h(\sqrt{x+h} + \sqrt{x}} + 1$$

			$$f'(x) = \lim \limits _{h \to 0} \frac{h}{h(\sqrt{x+h} + \sqrt{x}} + 1$$

			$$f'(x) \lim \limits _{h \to 0} \frac{1}{\sqrt{x+h} + \sqrt{x}} + 1$$

			$$f'(x) = \frac{1}{2\sqrt{x}} + 1$$

			The domain of $f(x)$ is $[0, \infty)$. The domain of $f'(x)$ is $(0, \infty)$.

		% Q23
		\item $g(x) = \sqrt{9-x}$

			$$f'(x) = \lim \limits _{h \to 0} \frac{\sqrt{9-(x+h)} - \sqrt{9-x}}{h}$$

			$$f'(x) = \lim \limits _{h \to 0} \frac{9-(x+h)-(9-x)}{h(\sqrt{9-(x+h)}+\sqrt{9-x})}$$

			$$f'(x) = \lim \limits _{h \to 0} \frac{h}{h(\sqrt{9-(x+h)}+\sqrt{9-x}}$$

			$$f'(x) = \lim \limits _{h \to 0} \frac{1}{\sqrt{9-(x+h)}-\sqrt{9-x}}$$

			$$f'(x) = \frac{1}{2\sqrt{9-x}}$$

		% Q24
		\item $f(x) = \frac{x^2 - 1}{2x - 3}$

			$$f'(x) = \lim \limits _{h \to 0} \frac{\frac{(x+h)^2-1}{2(x+h)-3} - \frac{x^2-1}{2x-3}}{h}$$

			$$f'(x) = \lim \limits _{h \to 0} \frac{[(x+h)^2-1](2x-3) - [2(x+h)-3](x^2-1)}{h[2(x+h)-3](2x-3)}$$

			$$f'(x) = \lim \limits _{h \to 0} \frac{(x^2+2xh+h^2-1)(2x-3)-(2x+2h-3)(x^2-1)}{h[2(x+h)-3](2x-3)}$$

			$$f'(x) = \lim \limits _{h \to 0} \frac{2x^2h+2xh^2-6xh-3h^2+2h}{h[2(x+h)-3](2x-3)}$$

			$$f'(x) = \lim \limits _{h \to 0} \frac{2x^2+2xh-6x-3h+2}{[2(x+h)-3](2x-3)}$$

			$$f'(x) = \frac{2x^2-6x+2}{(2x-3)^2}$$

		% Q25
		\item $G(t) = \frac{4t}{t+1}$

			$$G'(t) = \lim \limits _{h \to 0} \frac{\frac{4(t+h)}{(t+h)+1} - \frac{4t}{t+1}}{h}$$

			$$G'(t) = \lim \limits _{h \to 0} \frac{[4(t+h)](t+1) - 4t[(t+1)+1]}{h[(t+h)+1](t+1)}$$

			$$G'(t) = \lim \limits _{h \to 0} \frac{(4t+4h)(t+1) - (4t^2+4th+4t)}{h(t^2+ht+t+t+h+1)}$$

			$$G'(t) = \lim \limits _{h \to 0} \frac{4t^2+4th+4t+4h-4t^2-4th-4t}{h(t^2+ht+2t+h+1)}$$

			$$G'(t) = \lim \limits _{h \to 0} \frac{4}{t^2+ht+2t+h+1}$$

			$$G'(t) = \frac{4}{t^2+2t+1}$$

			$$G'(t) = \frac{4}{(t+1)^2}$$

		% Q26
		\item $f(x) = x^{3/2}$

			$$f'(x) = \lim \limits _{h \to 0} \frac{(x+h)^{3/2} - x^{3/2}}{h}$$

			$$f'(x) = \lim \limits _{h \to 0} \frac{(x+h)^{3/2} - x^{3/2}}{h} \times \frac{(x+h)^{3/2} + x^{3/2}}{(x+h)^{3/2} + x^{3/2}}$$

			$$f'(x) = \lim \limits _{h \to 0} \frac{(x+h)^3 - x^3}{h[(x+h)^{3/2} + x^{3/2}}$$

			$$f'(x) = \lim \limits _{h \to 0} \frac{x^3 - 3x^2h + 3xh^2 + h - x^3}{h[(x+h)^{3/2} + x^{3/2}]}$$	

			$$f'(x) = \lim \limits _{h \to 0} \frac{3x^2 + 3xh}{(x+h)^{3/2} + x^{3/2}}$$

			$$f'(x) = \frac{3x^2}{2x^{3/2}}$$

			$$f'(x) = \frac{3\sqrt{x}}{2}$$

		% Q27
		\item $f(x) = x^4$

			$$f'(x) = \lim \limits _{h \to 0} \frac{(x+h)^4 - x^4}{h}$$

			$$f'(x) = \lim \limits _{h \to 0} \frac{x^4 + 4x^3h + 6x^2h^2 + 4xh^3 + h^4 - x^4}{h}$$

			$$f'(x) = \lim \limits _{h \to 0} 4x^3 + 6x^2h + 4xh^2 + h^3$$

			$$f'(x) = 4x^3$$

		% TODO: Q28
		\item
		\begin{enumerate}
			\item Sketch the graph of $f(x) = \sqrt{6-x}$ by starting with the graph
				of $y = \sqrt{x}$ and using the transformations of Section 1.2

			\item Use the graph from part (a) to sketch the graph of $f'$.

			\item Use the definition of a derivative to find $f'(x)$. What are 
				the domains of $f$ and $f'$?

			\item Use a graphing device to graph $f'$ and compare with your sketch
				in part (b).	
		\end{enumerate}

		% TODO: Q29
		\item
		\begin{enumerate}
			\item If $f(x) = x^4 + 2x,$ find $f'(x)$.

			\item Check to see that your answer to part (a) is reasonable
				by comparing the graphs of $f$ and $f'$.
		\end{enumerate}

		% TODO: Q30
		\item	
		\begin{enumerate}
			\item If $f(t) = t^2 - \sqrt{t}$, find $f'(t)$

			\item Check to see that your answer to part (a) is reasonable by
				comparing the graphs of $f$ and $f'$.
		\end{enumerate}

		% Q31
		\item The unemployment rate $U(t)$ varies with time. The table gives the
			percentage of unemployed in the Australian labor force measured
			at midyear from 1995 to 2004.

			\begin{tabular}{|c|c||c|c|}
				\hline
				$t$ & $U(t)$ & $t$ & $U(t)$ \\
				\hline
				1995 & 8.1 & 2000 & 6.2 \\
				1996 & 8.0 & 2001 & 6.9 \\
				1997 & 8.2 & 2002 & 6.5 \\
				1998 & 7.9 & 2003 & 6.2 \\
				1999 & 6.7 & 2004 & 5.6 \\
				\hline
			\end{tabular}

		\begin{enumerate}
			\item What is the meaning of $U'(t)$? What are its units?

				$U'(t)$ is a function that represents the instantaneous
				rate of change of $U(t)$ (in this case the change in percentage
				per year). The units would be percent per year.

			\item Construct a table of estimated values for $U'(t)$.

			Average change from 1995-1996 = -0.1.

			Average change from 1996-1997 = 0.2

			Average change from 1997-1998 = -0.3

			Average change from 1998-1999 = -1.2

			Average change from 1999-2000 = -0.5

			Average change from 2000-2001 = 0.7

			Average change from 2001-2002 = -0.4

			Average change from 2002-2003 = -0.3

			Average change from 2003-2004 = -0.6

			\begin{tabular}{|c|c||c|c|}

				\hline
				$t$ & $U'(t)$ & $t$ & $U'(t)$ \\
				\hline
				1995 & -0.1 & 2000 & 0.1 \\
				1996 & 0.05 & 2001 & 0.15 \\
				1997 & -0.05 & 2002 & -0.35 \\
				1998 & -0.75 & 2003 & -0.45 \\
				1999 & -0.85 & 2004 & -0.6 \\
				\hline
			\end{tabular}
		\end{enumerate}

		% TODO: Q32
		\item Let $P(t)$ be the percentage of the population of the Philippines over
				the age of 60 at time $t$. The table gives projections of
				values of this function from 1995 to 2020.

			\begin{tabular}{|c|c||c|c|}
				\hline
				$t$ & $P(t)$ & $t$ & $P(t)$ \\
				\hline
				1995 & 5.2 & 2010 & 6.7 \\
			 	\hline
			\end{tabular}

		% Q45
		\item Let $f(x) = \sqrt[3]{x}$.

		\begin{enumerate}
			\item If $a \ne 0,$ use Equation 2.1.5 to find $f'(a)$

				$$f'(a) = \lim \limits _{x \to a} \frac{f(x) - f(a)}{x-a}$$

				$$f'(a) = \lim \limits _{x \to a} \frac{\sqrt[3]{x} - \sqrt[3]{a}}{x-a}$$

				Using difference of cubes we multiply by

				$$x^{2/3} + a^{1/3}x^{1/3} + a^{2/3}$$

				$$f'(a) = \lim \limits _{x \to a} \frac{x - a}{(x-a)(x^{2/3}+x^{1/3}a^{1/3}+a^{2/3}}$$

				$$f'(a) = \lim \limits _{x \to a} \frac{1}{x^{2/3}+x^{1/3}a^{1/3}+a^{2/3}}$$

				$$f'(a) = \frac{1}{3a^{2/3}}$$

			\item Show that $f'(0)$ does not exist.

				We can see from the function for $f'(a), \frac{1}{3a^{2/3}}$ that $f'(0)$
				is undefined.

			\item Show that $y = \sqrt[3]{x}$ has a vertical tangent line at $(0,0)$.

				If we take the limit $\lim \limits _{a \to 0^{+}} \frac{1}{3a^{2/3}}$ 

				As $a$ gets smaller and smaller the denominator becomes very small while
				the numerator remains constant so...

				$$\lim \limits _{a \to 0^{+}} \frac{1}{3a^{2/3}} = \infty$$

				If we take the limit $\lim \limits _{a \to 0^{-}} \frac{1}{3a^{2/3}}$

				As $a$ gets closer and closer to zero from the negative side $3a^{2/3}$ then
				the denominator becomes a very small positive number and so

				$$\lim \limits _{a \to 0^{-}} \frac{1}{3a^{2/3}} = \infty$$

				So we can see the the gradient of the tangent becomes vertical at $f'(0)$.
		\end{enumerate}

		% Q49
		\item Recall that a function $f$ is called \emph{even} if $f(-x) = f(x)$ for all $x$ in its
			domain and \emph{odd} if $f(-x) = -f(x)$ for all such $x$. Prove each of the
			following.

		\begin{enumerate}
			\item The derivative of an even function is an odd function.

				We can prove this using the chain rule.

				The definition of an even function is

				$$f(x) = f(-x)$$

				Using the chain rule...

				$$f'(x) = -f'(-x)$$

				Which gives...

				$$-f'(x) = f'(-x)$$

				Which is the definition of an odd function.

			\item The derivative of an odd function is an even function.

				We can also prove this with the chain rule.

				The definition of an odd function is

				$$f(x) = -f(-x)$$

				When we differentiate with the chain rule

				$$f'(x) = f'(-x)$$

				Which is an even function.
				
		\end{enumerate}

		% Q51
		\item Let $\ell$ be the tangent line to the parabola $y=x^2$ at the point (1,1).
			The \emph{angle of inclination} of $\ell$ is the angle $\phi$ that $\ell$
			makes with the positive direction of the $x$-axis. Calculate $\phi$
			correct to the nearest degree.

			If we differentiate $y = x^2$ we have $2x$ therefore the gradient of
			the tangent is $f'(1) = 2(1) = 2$.

			So if we construct a right angle triangle with a side on the $x$-axis
			of length 1 and a perpendicular side of length 2.

			Thus the angle of the hypotenuse with the $x$-axis is

			$\tan^{-1} (2) = 63^{\circ}$ degrees.

	\end{enumerate}
\end{document}
