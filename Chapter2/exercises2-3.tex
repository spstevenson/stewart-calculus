\documentclass{article}

\usepackage{amsmath}

\usepackage{gensymb}

\renewcommand{\thesection}{}

\begin{document}

	\section{2.3 Exercises}

	\begin{enumerate}
		\item[1-26] Differentiate the function.

		% Q1
		\item $f(x) = 2^{40}$

			Because this is a constant function

			$$\frac{d}{dx} 2^{40} = 0$$

		% Q2
		\item $f(x) = \sqrt{30}$

			Because this is a constant function

			$$\frac{d}{dx} \sqrt{30} = 0$$

		% Q3
		\item $f(t) = 3 - \frac{2}{3}t$

			$$f'(t) = \frac{d}{dt}3 - \frac{d}{dt} (\frac{2}{3}t)$$

			$$f'(t) = -\frac{2}{3} [\frac{d}{dt} t]$$

			Because $\frac{d}{dt} t = 1$

			$$f'(t) = -\frac{2}{3}$$

		% Q4
		\item $F(x) = \frac{3}{4}x^8$

			$$F'(x) = \frac{3}{4} [\frac{d}{dx} x^8]$$

			Because $\frac{d}{dx} x^8 = 8x^7$

			$$F'(x) = \frac{3}{4} (8x^7) = 6x^7$$

		% Q5
		\item $f(x) = x^3 - 4x + 6$

			$$f'(x) = \frac{d}{dx} x^3 - \frac{d}{dx} 4x + \frac{d}{dx} 6$$

			$$f'(x) = 3x^2 - 4$$	

		% Q6
		\item $f(t) = \frac{1}{2}t^6 - 3t^4 + t$

			$$f'(t) = \frac{d}{dt} [\frac{1}{2}t^6] - \frac{d}{dt} 3t^4 + \frac{d}{dt} t$$

			$$f'(t) = \frac{1}{2} [\frac{d}{dt} t^6] - 3[\frac{d}{dt} t^4] + 1$$

			$$f'(t) = \frac{1}{2} (6t^5) - 3(4t^3) + 1$$

			$$f'(t) = 3t^5 - 12t^3 + 1$$

		% Q7
		\item $f(x) = 3x^2 - 2\cos x$

			$$f'(x) = \frac{d}{dx} 3x^2 - \frac{d}{dx} 2\cos x$$

			$$f'(x) = 3 [\frac{d}{dx} x^2] - 2 [\frac{d}{dx} \cos x$$

			$$f'(x) = 3 (2x) - 2 (-\sin x)$$

			$$f'(x) = 6x + 2\sin x$$

		% Q8
		\item $y = \sin t + \pi \cos t$

			$$\frac{dy}{dt} = \frac{d}{dt} (\sin x) + \frac{d}{dt} (\pi \cos t)$$

			$$\frac{dy}{dt} = \cos x + \pi [\frac{d}{dt} \cos t]$$

			$$\frac{dy}{dt} = \cos x + \pi (- \sin t)$$

			$$\frac{dy}{dt} = \cos x - \pi \sin t$$

		% Q9
		\item $g(x) = x^2(1-2x)$

			Although we could solve this using the product rule, lets expand the brackets.

			$$g(x) = x^2 - 2x^3$$

			$$g'(x)  = \frac{d}{dx} x^2 - \frac{d}{dx} [2x^3]$$

			$$g'(x)  = 2x - 2 [\frac{d}{dx} x^3]$$

			$$g'(x) = 2x - 2 (3x^2)$$

			$$g'(x) = 2x - 6x^2$$

		% Q10
		\item $h(x) = (x-2)(2x + 3)$

			Lets expand this out into a polynomial that we can differentiate with the power rule.

			$$h(x) = 2x^2 + 3x - 4x - 6$$

			$$h(x) = 2x^2 - x - 6$$

			$$h'(x) = \frac{d}{dx} (2x^2) - \frac{d}{dx} x - \frac{d}{dx} 6$$

			$$h'(x) = 2 [ \frac{d}{dx}x^2] - 1$$

			$$h'(x) = 2(2x) - 1$$

			$$h'(x) = 4x - 1$$

		% Q11
		\item $y = x^{-2/5}$

			$$\frac{dy}{dx} = \frac{d}{dx} x^{-2/5}$$

			$$\frac{dy}{dx} = -\frac{2 x^{-7/5}}{5}$$

		% Q12
		\item $B(y) = cy^{-6}$

			Here we are treating $c$ as a constant

			$$B'(y) = \frac{d}{dy} (cy^{-6})$$

			$$B'(y) = c [\frac{d}{dy} y^{-6}$$

			$$B'(y) = c(-6y^{-7})$$

			$$B'(y) = -6cy^{-7}$$

		% Q13
		\item $A(s) = - \frac{12}{s^{5}}$

			We can express this as 

			$$A(s) = - 12s^{-5}$$

			$$A'(s) = \frac{d}{ds} [-12s^{-5}]$$

			$$A'(s) = -12 [\frac{d}{ds} s^{-5}]$$

			$$A'(s) = -12 (-5 s^{-6})$$

			$$A'(s) = 60 s^{-6}$$

		% Q14
		\item $y = x^{5/3} - x^{2/3}$

			$$\frac{dy}{dx} = \frac{d}{dx} x^{5/3} - \frac{d}{dx} x^{2/3}$$

			$$\frac{dy}{dx} = \frac{5 x^{2/3}}{3} - \frac{2}{3x^{1/3}}$$

		% Q15
		\item $R(a) = (3a + 1)^2$

			$$R(a) = 6a^2 + 6a + 1$$

			$$R'(a) = \frac{d}{da} [6a^2] + \frac{d}{da} [6a] + \frac{d}{da} 1$$

			$$R'(a) = 6 [\frac{d}{da} a^2] + 6 [\frac{d}{da} a]$$

			$$R'(a) = 6 (2a) + 6 (1) + 1$$

			$$R'(a) = 12a + 6$$	

		% Q16
		\item $y = \sqrt{x}(x - 1)$

			$$y = x \sqrt{x} - \sqrt{x}$$

			$$y = x \dot x^{1/2} - \sqrt{x}$$

			$$y = x^{3/2} - x^{1/2}$$

			$$\frac{dy}{dx} = \frac{d}{dx} x^{3/2} - \frac{d}{dx} x^{1/2}$$

			$$\frac{dy}{dx} = \frac{3\sqrt{x}}{2} - \frac{1}{2\sqrt{x}}$$

		% Q17
		\item $S(p) = \sqrt{p} - p$

			$$S'(p) = \frac{d}{dp} p^{1/2} - \frac{d}{dp} p$$

			$$S'(p) = \frac{1}{2} p^{-1/2} - 1$$

		% Q18
		\item $S(R) = 4 \pi R^2$

			$$S'(R) = \frac{d}{dR} 4 \pi R^2$$

			$$S'(R) = 4 \pi [\frac{d}{dR} R^2$$

			$$S'(R) = 4 \pi (2R)$$

			$$S'(R) = 8 \pi R$$

		% Q19
		\item $y = \frac{x^2 + 4x + 3}{\sqrt{x}}$

			$$y = (x^2 + 4x + 3)(x^{-1/2})$$

			$$y = x^{3/2} + 4x^{1/2} + 3x^{-1/2}$$

			$$\frac{dy}{dx} = \frac{d}{dx} x^{3/2} + \frac{d}{dx} 4x^{1/2} + \frac{d}{dx} x^{-1/2}$$

			$$\frac{dy}{dx} = \frac{3x^{1/2}}{2} + 4 [\frac{d}{dx} x^{1/2}] - \frac{1}{2x^{3/2}}$$

			$$\frac{dy}{dx} = \frac{3x^{1/2}}{2} + \frac{2}{x^{1/2}} - \frac{1}{2x^{3/2}}$$

		% Q20
		\item $g(u) = \sqrt{2}u + \sqrt{3u}$

			$$g'(u) = \frac{d}{du} \sqrt{2}u + \frac{d}{du} \sqrt{3u}$$

			We can work out $frac{d}{du} \sqrt{3u}$ using the chain rule or from the 
			definition of a derivative.

			$$\frac{d}{du} \sqrt{3u} = \lim \limits _{h \to 0} \frac{\sqrt{3(u+h)} - \sqrt{3u}}{h}$$

			We multiply this by $\frac{\sqrt{3(u+h)} + \sqrt{3u}}{\sqrt{3(u+h) + \sqrt{3u}}}$

			$$\frac{d}{du} \sqrt{3u} = \lim \limits _{h \to 0} \frac{3(u+h) - 3u}{h(\sqrt{3(u+h)} + \sqrt{3u}}$$

			$$\frac{d}{du} \sqrt{3u} = \lim \limits _{h \to 0} \frac{3h}{h(\sqrt{3(u+h)} + \sqrt{3u}}$$
			
			$$\frac{d}{du} \sqrt{3u} = \lim \limits _{h \to 0} \frac{3}{\sqrt{3(u+h)} + \sqrt{3u}}$$

			$$\frac{d}{du} \sqrt{3u} = \frac{3}{2\sqrt{3u}}$$

			We can use this result to solve the original derivative.

			$$g'(u) = \sqrt{2} \frac{d}{du} u + \frac{3}{2\sqrt{3u}}$$
			
			$$g'(u) = \sqrt{2} + \frac{3}{2\sqrt{3u}}$$

		% Q21
		\item $v = t^2 - \frac{1}{\sqrt[4]{t^3}}$

			$$\frac{dv}{dt} = \frac{d}{dt} t^2 - \frac{d}{dt} \frac{1}{\sqrt[4]{t^3}}$$

			$$\frac{dv}{dt} = 2t - \frac{d}{dt} t^{-3/4}$$

			$$\frac{dv}{dt} = 2t + \frac{3}{4} t^{-7/4}$$

		% Q22
		\item $y = \frac{x^2 - 2\sqrt{x}}{x}$

			$$y = \frac{x^2}{x} - \frac{2\sqrt{x}}{x}$$

			$$y = x - 2x^{-1/2}$$

			$$\frac{dy}{dx} = \frac{d}{dx} x - \frac{d}{dx} 2x^{-1/2}$$

			$$\frac{dy}{dx} = 1 + x^{-3/2}$$

		% Q23
		\item $z = \frac{A}{y^{10}} + B \cos y$

			$$\frac{dz}{dy} = \frac{d}{dy} Ay^{-10} + \frac{d}{dy} B \cos y$$

			$$\frac{dz}{dy} = A [\frac{d}{dy} y^{-10}] + B [\frac{d}{dy} \cos y$$

			$$\frac{dz}{dy} = A (-10y^{-11}) + B [ - \sin y]$$

			$$\frac{dz}{dy} = -10Ay^{-11} - B\sin y$$

		% Q24
		\item $y = \frac{\sin \theta}{2} + \frac{c}{\theta}$

			$$y = \frac{1}{2} \sin \theta + c \theta^{-1}$$

			$$\frac{dy}{d\theta} = \frac{d}{d\theta} (\frac{1}{2} \sin \theta) + \frac{d}{d\theta} (c \theta^{-1})$$

			$$\frac{dy}{d\theta} = \frac{1}{2} [ \frac{d}{d\theta} \sin \theta ] + c [ \frac{d}{d\theta} \theta^{-1}]$$

			$$\frac{dy}{d\theta} = \frac{1}{2} (\cos \theta) + c (- \theta^{-2})$$

			$$\frac{dy}{d\theta} = \frac{1}{2} \cos \theta - c \theta^{-2}$$

		% Q25
		\item $H(x) = (x + x^{-1})^{3}$

			Lets expand this out

			$$H(x) = x^3 + 3x^2(x^{-1}) + 3x(x^{-1})^{2} + (x^{-1})^{3}$$

			$$H(x) = x^3 + 3x + 3x \dot x^{-2} + x^{-3}$$

			$$H(x) = x^3 + 3x + 3x^{-1} + x^{-3}$$

			$$H'(x) = 3x^2 + 3 - 3x^{-2} - 3x^{-4}$$

		% Q26
		\item $u = \sqrt[3]{t^2} + 2\sqrt{t^3}$

			$$u = t^{2/3} + 2t^{3/2}$$

			$$\frac{du}{dt} = \frac{d}{dt} t^{2/3} + \frac{d}{dt} 2t^{3/2}$$

			$$\frac{du}{dt} = \frac{2}{3} t^{-1/3} + 2[\frac{d}{dt} t^{3/2}]$$

			$$\frac{du}{dt} = \frac{2}{3} t^{-1/3} + \frac{6}{2} t^{1/2}$$
			$$\frac{du}{dt} = \frac{2}{3} t^{-1/3} + 3t^{1/2}$$

		\item[27-28] Find equations of the tangent line and the normal line to
				the curve at the given point.

		% Q27
		\item $y = 6 \cos x, (\pi/3,3)$

			First lets differentiate the equation

			$$y' = - 6 \sin x$$

			$$y'(\pi/3) = - 6 \sin (\pi/3)$$

			$$\sin \frac{\sqrt{3}}{2}$$

			$$y'(\pi/3) = - 6 \frac{\sqrt{3}}{2} = - 3 \sqrt{3}$$

			We can then use the point - slope equation to get the tangent line.

			$$y - 3 = - 3 \sqrt{3} ( x - \frac{\pi}{3} )$$

			$$y - 3= - 3x\sqrt{3} + \pi \sqrt{3}$$

			$$y = - 3x\sqrt{3} + \pi \sqrt{3} + 3$$

			To get the equation for the normal we must find the gradient of the perpendicular

			$$ = \frac{1}{3\sqrt{3}}$$

			And use the point - slope again.

			$$y - 3 = \frac{1}{3\sqrt{3}} (x - \frac{\pi}{3})$$

			$$y - 3 = \frac{x}{3\sqrt{3}} - \frac{\pi}{\sqrt{3}}$$

			$$ = \frac{x}{3\sqrt{3}} - \frac{\pi}{\sqrt{3}} + 3$$

		% Q28
		\item $y = (1 + 2x)^2, (1,9)$

			First lets expand the equation

			$$y = 4x^2 + 4x + 1$$

			Now lets differentiate...

			$$y' = 8x + 4$$

			The gradient of the curve at 1 is...

			$$y'(1) = 8(1) + 4 = 12$$

			Using the point - slope method we can determine an equation for the tangent

			$$y - 9 = 12(x -1)$$

			$$y - 9 = 12x - 12$$

			$$y = 12x - 12 + 9 = 12x - 3$$

			The normal is perpendicular to the tangent

			$$m = -\frac{1}{12}$$

			Using the point - slope method we can determine an equation for the normal

			$$y - 9 = -\frac{1}{12}(x - 1)$$

			$$y - 9 = -\frac{1}{12}x + \frac{1}{12}$$

			$$y = -\frac{x}{12} + 9 + \frac{1}{12}$$

		\item[31-34] Find the first and sercond derivatives of the function.

		% Q31
		\item $f(x) = x^4 - 3x^3 + 16x$

			$$f'(x) = \frac{d}{dx} x^4 - \frac{d}{dx} 3x^3 + \frac{d}{dx} 16x$$

			$$f'(x) = 4x^3 - 3[\frac{d}{dx} x^3] + 16 [\frac{d}{dx} x]$$

			$$f'(x) = 4x^3 - 3[3x^2] + 16[1]$$

			$$f'(x) = 4x^3 - 9x^2 + 16$$

			$$f''(x) = \frac{d}{dx} 4x^3 - \frac{d}{dx} 9x^2 + \frac{d}{dx} 16$$

			$$f''(x) = 4[\frac{d}{dx} x^3] - 9[\frac{d}{dx} x^2]$$

			$$f''(x) = 4[3x^2] - 9[2x]$$

			$$f''(x) = 12x^2 - 18x$$

		% Q32
		\item $G(r) = \sqrt{r} + \sqrt[3]{r}$

			$$G(r) = r^{1/2} + r^{1/3}$$

			$$G'(r) = \frac{d}{dx} r^{1/2} + \frac{d}{dx} r^{1/3}$$

			$$G'(r) = \frac{1}{2} r^{-1/2} + \frac{1}{3} r^{-2/3}$$

			$$G''(r) = - \frac{1}{4} r^{-3/2} - \frac{2}{9} r^{-5/3}$$

		% Q33
		\item $g(t) = 2\cos t - 3 \sin t$

			$$g'(t) = \frac{d}{dt} 2\cos t - \frac{d}{dt} 3 \sin t$$

			$$g'(t) = 2[\frac{d}{dt} \cos t] - 3 [\frac{d}{dt} \sin t]$$

			$$g'(t) = 2[-\sin t] - 3[\cos t]$$

			$$g'(t) = 3 \cos t - 2 \sin t$$

			$$g''(t) = - 3 \sin t - 2 \cos t$$

		% Q34
		\item $h(t) = \sqrt{t} + 5 \sin t$

			$$h(t) = t^{1/2} + 5\sin t$$

			$$h'(t) = \frac{1}{2} t^{-1/2} + 5 \cos t$$

			$$h''(t) = -\frac{1}{4} t^{-3/2} - 5 \sin t$$

		% Q35
		\item Find $\frac{d^{99}}{dx^{99}} (\sin x)$

			Lets do the first few derivations

			$$\frac{d}{dx} (\sin x) = \cos x$$

			$$\frac{d^{2}}{dx^{2}} (\sin x) = -\sin x$$

			$$\frac{d^{3}}{dx^{3}} (\sin x) = -\cos x$$

			$$\frac{d^{4}}{dx^{4}} (\sin x) = \sin x$$

			$$\frac{d^{5}}{dx^{5}} (\sin x) = \cos x$$

			$$\frac{d^{6}}{dx^{6}} (\sin x) = -\sin x$$

			$$\frac{d^{7}}{dx^{7}} (\sin x) = -\cos x$$

			$$\frac{d^{8}}{dx^{8}} (\sin x) = \sin x$$

			Now, the factors of 99 are 3, 3 and 11.

			We can see above that derivatives that are multiples
			of 3 are $-\sin x$

			So $$\frac{d^{99}}{dx^{99}} (\sin x) = -\sin x$$

		% Q36
		\item Find the $n$th derivative of each function by calculating the
			first few derivatives and observing the pattern that occurs.

		\begin{enumerate}
			\item $f(x) = x^{n}$

			We will assume here that $n$ is a positive integer, and for clarity find
			the $r$th derivative

			$$f(x) = x^{n}$$

			$$f'(x) = nx^{n-1}$$

			$$f'(x) = n(n-1)x^{n-2}$$

			$$f'(x) = n(n-1)(n-2)x^{n-3}$$

			So we can see the exponent will be $n-r$ while the
			coefficient will be $n! - (n-r)!$

			$$[n! - (n-r)!]x^{n-r}$$

			\item $f(x) = \frac{1}{x}$

			If we calculate the first few derivatives

			$$f(x) = \frac{1}{x}$$

			$$f'(x) = \frac{-1}{x^2}$$

			$$f''(x) = \frac{(-1)(-2)}{x^{3}}$$

			$$f'''(x) = \frac{(-1)(-2)(-3)}{x^{4}}$$

			So the $n$th derivative is

			
			$$\frac{(-1)(-2)...(-n)}{x^{n+1}}$$
		\end{enumerate}

		% Q37
		\item For what values of $x$ does the graph of $f(x) = x + 2\sin x$
			have a horizontal tangent?

			To find the horizontal tangents we need to find the points at
			the derivatives $f'(x) = 0$

			$$f'(x) = 1 + 2\cos x$$

			$$1 + 2\cos x = 0$$

			$$2 \cos x = -1$$

			$$\cos x = - \frac{1}{2}$$

			On the unit circle $\cos (\frac{2\pi}{3}) = -\frac{1}{2}$ and
			$\cos (\frac{4\pi}{3}) = -\frac{1}{2}$

			So the horizontal tangents are at...

			$$\frac{2\pi}{3} + k\pi, \frac{4\pi}{3} + k\pi$$ where $k$ is
			an integer.

		% Q38
		\item For what values of $x$ does the graph of 
			$f(x) = x^3 + 3x^2 + x + 4$ have a horizontal tangent?

			First we need to find the derivative.

			$$f'(x) = 3x^2 + 6x + 1$$

			And find the roots of the equation.

			$$3x^2 + 6x + 1 = 0$$

			Which we can do by completing the square.

			$$3x^2 + 6x = -1$$

			$$x^2 + 2x = -\frac{1}{3}$$

			$$x^2 + 2x + 1 = 1 - \frac{1}{3}$$

			$$(x+1)^2 = 1 - \frac{1}{3}$$

			$$(x+1)^2 = \frac{2}{3}$$

			$$x + 1 = \pm \sqrt{\frac{2}{3}}$$

			$$x = -1 \pm \sqrt{\frac{2}{3}}$$

		% Q39
		\item Show that the curve $y = 6x^3 + 5x - 3$ has no tangent
			line with slope 4.


			$$y = 6x^3 + 5x - 3$$

			$$y' = 18x^2 + 5$$

			$$18x^2 + 5 = 4$$

			$$18x^2 + 1 = 0$$

			We can use the quadratic descriminant to show there are no solutions.

			$$b^2 - 4ac$$

			$$0 - 4 (18) (1)$$

			$$-72 < 0$$

		% Q40
		\item Find an equation of the tangent line to the curve $y = x\sqrt{x}$ that
			is parallel to the line $y = 1 + 3x$

			The gradient of the tangent line is 3.

			We need to find the point on the curve where the gradient is 3.

			$$y = x . x^{1/2} = x^{3/2}$$

			$$y' = \frac{3\sqrt{x}}{2}$$

			So we need to solve for $x$...

			$$\frac{3\sqrt{x}}{2} = 3$$

			$$3\sqrt{x} = 6$$

			$$\sqrt{x} = 2$$

			$$x = 4$$

			And we can put this in our original equation to get $y$

			$$y = 4\sqrt{4} = 8$$

			And now use point-slope line equation form to find a parallel tangent

			$$y - 8 = 3(x - 4)$$

			$$y - 8 = 3x - 12$$

			$$y = 3x - 4$$

		% Q41
		\item Find an equation of the normal line to the parabola $y = x^2 - 5x + 4$ that
			is parallel to the line $x - 3y = 5$.

			We are looking for a line parallel to $x - 3y = 5$

			If we rearrange this so that $y$ is on the left we have $y = \frac{1}{3} - \frac{5}{3}$.

			So we are looking for a line with a gradient of $\frac{1}{3}$, the perpendicular will
			have a gradient of $-3$. So to find a normal on the curve that is perpendicular to
			the line we must find a point on the curve where the tangent has a gradient of $-3$.

			If we differentiate the curve we get $y' = 2x - 5$. We want to find the value of $x$
			where $y' = -3$, so

			$$2x - 5 = -3$$

			$$2x = 2$$

			$$x = 1$$

			We must also know the value of $y$ when $x = 1$ so

			$$(1)^2 - 5(1) + 4 = 0$$

			So the normal equation is...

			$$y - 0 = \frac{1}{3}(x - 1)$$

			$$y = \frac{x}{3} - \frac{1}{3}$$

		% TODO Q42

		% Q43
		\item The equation of motion of a particle is $s = t^3 - 3t$, where $s$ is in meters and
			$t$ is in seconds. Find

		\begin{enumerate}
			\item the velocity and acceleration as functions of $t$.

			The velocity is the first derivative so

			$$s' = 3t^2 - 3$$

			The acceleration is the second derivatrive so...

			$$s'' = 6t$$

			\item the acceleration after 2 s

			$$s''(2) = 6(2) = 12 \text{ metres per second per second}$$

			\item the acceleration when the velocity is 0.

				$$3t^2 - 3 = 0$$

				$$3t^2 = 3$$

				$$t^2 = 1$$

				$$t = \pm 1$$

				So if we plug this into the equation of acceleration

				$$6(1) = 6$$
		\end{enumerate}

		% TODO: Q44
		% TODO: Q45
		% TODO: Q46

		% Q47
		\item The position function of a particle is given by $s = t^3 - 4.5t^2 - 7t, t \geq 0$.

			\begin{enumerate}
				\item When does the particle reach a velocity of 5 m/s?

				First we find the derivative...

				$$s' = 3t^2 - 9t -7$$

				And find the value of $t$ when $3t^2 - 9t - 7 = 5$

				$$3t^2 - 9t - 2 = 0$$

				$$t^2 - 3t - \frac{2}{3} = 0$$

				Let's find the value of $t$ by completing the square.

				$$t^2 - 3t = \frac{2}{3}$$

				$$t^2 - 3t + \frac{9}{4} = \frac{2}{3} + \frac{9}{4}$$

				$$(t - \frac{3}{2})^2 = \frac{35}{12}$$

				$$t - \frac{3}{2} = \pm \sqrt{\frac{35}{12}}$$

				$$t = \frac{3}{2} = \pm \sqrt{\frac{35}{12}}$$

				The only positive result is

				$$t = \frac{3}{2} + \sqrt{\frac{35}{12}} = 3.208$$

				\item When is the acceleration 0? What is the significance
					of this value of $t$?

					When we differentiate to get the equation for acceleration
					we get $s'' = 6t$

					Thus the acceleration is 0 when $t = 0$. This is significant
					as it is the initial value of $t$.
			\end{enumerate}

			% Q48
			\item If a ball is given a push so that it has an initial velocity of 5 m/s
				down a certain inclined plane, then the distance it has rolled down
				after $t$ seconds is $s = 5t + 3t^2$/

			\begin{enumerate}
				\item Find the velocity after 2 s.

				$$s' = 5 + 6t$$

				$$s'(2) = 5 + 6(2) = 17$$

				\item How long does it take for the velocity to reach 35 m/s?

					$$5 + 6t = 35$$

					$$6t = 30$$

					$$t = \frac{30}{6} = 5$$
			\end{enumerate}

			% Q49
			\item If a rock is thrown vertically upward from the surface of Mars with
				velocity 15 m/s, its height after $t$ seconds is $h = 15t - 1.86t^2$.

				\begin{enumerate}
					\item What is the velocity of the rock after 2 s?

						First we have to differentiate the position equation to
						get the velocity equation

						$$h' = 15 - 3.72t$$

						$$h'(2) = 15 - 3.72(2) = 7.56 \text{ m/s }$$

					\item What is the velocity of the rock when its height is 25 m
						on its way up? On its way down?

						We need to find the value of $t$ when $h = 25$.

						$$15t - 1.86t^2 = 25$$

						$$15t - 1.86t^2 - 25 = 0$$

						The solutions to this equation (from an equation solver)
						are $h = 2.353, 5.711$.

						We can find the velocity by plugging these values into the 
						first derivative.

						$$15 - 3.72(2.353) = 6.247 \text{ m/s }$$

						The second value is the velocity on the way down.

						$$15 - 3.72(5.711) = -6.245 \text{ m/s }$$

				\end{enumerate}

		% Q50
		\item If a ball is thrown vertically upward with a velocity of 24.5 m/s, then its height
			after $t$ seconds is $s = 24.5t - 4.9t^2$.

				\begin{enumerate}

					\item What is the maximum height reached by the ball?

						First we differentiate and find the value of $t$ where the
						gradient is 0.

						$$s' = 24.5 - 9.8t$$

						$$24.5 - 9.8t = 0$$

						$$9.8t = 24.5$$

						$$t = \frac{24.5}{9.8} = 2.5$$

						Then we plug that into our original equation

						$$24.5(2.5) - 4.9(2.5)^2 = 30.625$$

					\item What is the velocity of the ball when it is 29.4 m above
						the ground on its way up? On its way down?

						$$24.5t - 4.9t^2 = 29.4$$

						$$4.9t^2 - 24.5t + 29.4 = 0$$

						$$t^2 - 5t + 9 = 0$$

						$$(t-2)(t-3) = 0$$

						Velocity on the balls way up is

						$$24.5 - 9.8(2) = 4.9$$

						Velocity on the way down is

						$$24.5 - 9.8(3) = -4.9$$
				\end{enumerate}
					
			% Q51
			\item The cost, in dollars, of producing $x$ metres of a certain fabric
				is 

				$$C(x) = 1200 + 12x - 0.1x^2 + 0.0005x^3$$

			\begin{enumerate}
				\item Find the marginal cost function.

					We need to differentiate the function.

					$$C'(x) = \frac{d}{dx} 1200 + \frac{d}{dx} 12x - \frac{d}{dx} 0.1x^2 + \frac{d}{dx} 0.0005x^3$$

					$$C'(x) = 12[\frac{d}{dx} x] - 0.1[\frac{d}{dx}x^2] + 0.0005[\frac{d}{dx} x^3]$$

					$$C'(x) = 12[1] - 0.1[2x] + 0.0005[3x^2]$$

					$$C'(x) = 12 - 0.2x + 0.0015x^2$$

				\item Find $C'(200)$ and explain its meaning. What does it predict?

					$$C'(200) = 12 - 0.2(200) + 0.0015(200)^2$$

					$$C'(200) = 12 - 40 + 0.0015(40000)$$

					$$C'(x) = 12 - 40 + 6000 = 5972$$

					This is the marginal cost of of producing the 200th metre of fabric, it 
					predicts how much it will cost to produce the 200th metre.
			\end{enumerate}

			% Q52
			\item The cost function for a certain commodity is

				$$C(x) = 84 + 0.16x - 0.0006x^2 + 0.000003x^3$$

				\begin{enumerate}
					\item Find and interpret $C'(100)$

					$$C'(x) = \frac{d}{dx} 84 + \frac{d}{dx} 0.16x - \frac{d}{dx} 0.0006x^2 + \frac{d}{dx} 0.000003x^3$$

					$$C'(x) = 0.16[\frac{d}{dx} x] - 0.0006[\frac{d}{dx} x^2] + 0.000003[\frac{d}{dx} x^3]$$

					$$C'(x) = 0.16[1] - 0.0006[2x] + 0.000003[3x^2]$$

					$$C'(x) = 0.16 - 0.0012x + 0.000009x^2$$

					$$C'(100) = 0.16(100) - 0.0012(100) + 0.000009(100)^2$$

					$$C'(100) = 0.16 - 0.12 + 0.000009(10000)$$

					$$C'(100) = 0.16 - 0.12 + 0.09 = 0.13$$

					This is the approximate cost of producing the 10th item.

					\item Compare $C'(100)$ with the cost of producing the 101st item.

					$$C(101) = 84 + 0.16(101) - 0.0006(101)^2 + 0.000003(101)^3 = 97.130303$$

					$$C(100) = 97$$

					We can see the values are very close.

				\end{enumerate}

				% Q53
				\item A spherical balloon is being inflated. Find the rate of increase of the surface
					area ($S=4\pi r^2$) with respect to the radius $r$ when $r$ is (a) 20 cm, (b)
					40 cm, and (c) 60 cm. What conclusion can you make?

					First lets differentiate the function

					$$S' = 8\pi r$$

					$$S'(20) = 160\pi \text{ cm}^2 /\text{ cm}$$

					$$S'(40) = 320\pi \text{ cm}^2 /\text{ cm}$$

					$$S'(60) = 480\pi \text{ cm}^2 /\text{ cm}$$

				% Q54
				\item If a tank holds 5000 litres of water, which drains from the bottom of the tank
					in 40 minutes, then Toricelli's Law gives the volume $V$ of water remaining
					in the tank after $t$ minutes as

					$$V = 5000(1 - \frac{1}{40}t)^2 \text{	} 0 \leq t \leq 40$$

					Find the rate at which water is draining from the tank after (a) 5 min,
					(b) 10 min, (c) 20 min, and (d) 40 min. At what time is the water
					flowing out the fastest? The slowest? Summarize your feelings.

					First lets differentiate $V = 5000(1 - \frac{1}{40}t)^2$.

	$$V' =  \lim _{h \to 0} \frac{5000(1 - \frac{1}{40}(t + h))^2 - 5000(1 - \frac{1}{40}t)^2}{h}$$

	$$V' = \lim _{h \to 0} \frac{5000(1 - \frac{2(t+h)}{40} + (\frac{(t+h)}{40})^2) - 5000(1 - \frac{2t}{40} + (\frac{t}{40})^2)}{h}$$

	$$V' = \lim _{h \to 0} \frac{5000(1 - \frac{2(t+h)}{40} + \frac{(t+h)^2}{1600}) - 5000(1 - \frac{2t}{40} + (\frac{t^2}{1600})}{h}$$

	$$V' = \lim _{h \to 0} \frac{5000(1 - \frac{2(t+h)}{40} + \frac{t^2 + 2th + h^2}{1600}) - 5000 + \frac{10000t}{40} - \frac{5000t^2}{1600}}{h}$$

	$$V' = \lim _{h \to 0} \frac{5000(1 - \frac{2(t+h)}{40} + \frac{t^2}{1600} + \frac{2th}{1600} + \frac{h^2}{1600}) - 5000 + \frac{10000t}{40} - \frac{5000t^2}{1600}}{h}$$

	
	$$V' = \lim _{h \to 0} \frac{5000(1 + \frac{40t^2 + 80th + 40h^2 - 3200t - 3200h}{64000}) - 5000 + \frac{10000t}{40} - \frac{5000t^2}{1600}}{h}$$

	$$V' = \lim _{h \to 0} \frac{5000 + \frac{5000(40t^2 + 80th + 40h^2 - 3200t - 3200h}{64000} - 5000 + \frac{16000000t - 200000t^2}{64000}}{h}$$

	$$V' = \lim _{h \to 0} \frac{\frac{200000t^2 + 400000th + 200000h^2 - 16000000t - 16000000h}{64000} + \frac{16000000t - 200000t^2}{64000}}{h}$$

	$$V' = \lim _{h \to 0} \frac{200000t^2 + 400000th + 200000h^2 - 16000000t - 16000000h + 16000000t - 200000t^2}{64000h}$$

	$$V' = \lim _{h \to 0} \frac{400000th + 200000h^2 - 16000000h}{64000h}$$

	$$V' = \lim _{h \to 0} \frac{400000t + 200000h - 16000000}{64000}$$

	$$V' = \frac{400000t - 16000000}{64000}$$

	$$V' = \frac{400t - 16000}{64}$$

	$$V' = \frac{25t - 1000}{4}$$

	Now we can work out the rate of drainage for the various values of $t$

	$$V'(5) = \frac{25(5) - 1000}{4} = \frac{1000 - 125}{4} = \frac{-875}{4} = - 218.75$$

	$$V'(10) = \frac{25(10) - 1000}{4} = \frac{1000 - 250}{4} = \frac{-750}{4} = - 187.5$$

	$$V'(20) = \frac{25(20) - 1000}{4} = \frac{1000 - 500}{4} = \frac{-500}{4} = - 125$$

	$$V'(40) = \frac{25(40) - 1000}{4} = \frac{1000 - 1000}{4} = 0$$

	We can see that the water is flowing fastest in the beginning and slowest at the end. As
	$t$ approaches 40, the rate of flow decreases.

		% Q55
		\item Boyle's Law states that when a sample of gas is compressed at a constant
			temperature, the pressure $P$ of the gas is inversely proportional to the
			volume $V$ of the gas.

		\begin{enumerate}
			\item Suppose that the pressure of a sample of air that occupies 
				0.106 m\textsuperscript{th} at 25 \degree C is 50 kPa. Write $V$ as
				a function of $P$.

				$P$ is inversely proportional to $V$ so

				$V = k\frac{1}{P}$ 

				So $0.106 = k\frac{1}{50}$

				$$k = 50 \times 0.106 = 5.3$$

				Therefore

				$$V = \frac{5.3}{P}$$

			\item Calculate $dV/dP$ when $P = 50 \text{ kPa}$. What is the meaning of this
				derivative? What are its units?

				So $V = 5.3P^{-1}$

				$$V' = -5.3P^{-2}$$

				$$V' = - \frac{5.3}{P^2}$$

				$$V'(50) = - \frac{5.3}{(50)^2}$$

				$$V'(50) = - \frac{5.3}{2500} = - 0.00212$$

				The units are m\textsuperscript{3} / kPa. This is the instantaneous rate of change at
				50 kPa at 25 \degree C.
		\end{enumerate}

			% Q56
			\item Newton's Law of Gravitation says that the the magnitude $F$ of the force exerted by a body
				of mass $M$ is 

				$$F = \frac{GmM}{r^2}$$

				where $G$ is the gravitational constant and $r$ is the distance between the bodies.

			\begin{enumerate}
				\item Find $dF/dR$ and explain its meaning. What does the minus sign indicate.

					$$F = GmMr^{-2}$$

					$$\frac{dF}{dr} = -2GmMr^{-3}$$

					$$\frac{dF}{dr} = - \frac{2GmM}{r^3}$$

					The minus signs indicate that the marginal change in force is negative. The rate at
					which the force decreases gets smaller.

				\item Suppose it is known that the earth attracts an object with a force that decreases at the
					rate of 2 N/km when $r$ = 20,000 km. How fast does this force change when $r$ = 10,000?

					$$- \frac{k}{20000^{3}} = 2$$

					$$- k = 2 \times 20000^{3}$$

					$$k = - 1.6 \times 10^{13}$$

					$$- \frac{-1.6 \times 10^{13}}{10000^{3}} = 16 \text{ N/km}$$
			\end{enumerate}

			% Q57
			\item Use the definition of a derivative to show that if $f(x) = 1/x$, then $f'(x) = - 1/x^2$. (This proves
				the Power Rule for the case $n = -1$.)

				$$f'(x) = \lim \limits _{h \to 0} \frac{ \frac{1}{(x+h)} - \frac{1}{x} }{h}$$

				$$f'(x) = \lim \limits _{h \to 0} \frac{ \frac{x - (x+h)}{x(x+h)} }{h}$$

				$$f'(x) = \lim \limits _{h \to 0} \frac{x - (x+h)}{xh(x+h)}$$

				$$f'(x) = \lim \limits _{h \to 0} \frac{-h}{xh(x+h)}$$

				$$f'(x) = \lim \limits _{h \to 0} - \frac{1}{x(x+h)}$$

				$$f'(x) = - \frac{1}{x^2}$$

			% Q58
			\item Prove, using the defintion of derivative, that if $f(x) = \cos x$, then $f'(x) = -\sin x$.

				$$f'(x) = \lim \limits _{h \to 0} \frac{\cos (x+h) - \cos x}{h}$$

				$$\cos (x+h) = \cos x \cos h - \sin x \sin h$$

				$$f'(x) = \lim \limits _{h \to 0} \frac{\cos x \cos h - \sin x \sin h - \cos h}{h}$$

				$$\cos x \cos h - \cos x = \cos x (\cos h - 1)$$

				$$f'(x) = \lim \limits _{h \to 0} \frac{\cos x ( \cos h - 1) - \sin x \sin h}{h}$$

				$$f'(x) = \lim \limits _{h \to 0} \cos x (\frac{\cos h - 1}{h}) - \sin x (\frac{\sin h}{h})$$

				$$\lim \limits _{h \to 0} \frac{\sin h}{h} = 1$$

				$$\lim \limits _{h \to 0} \frac{\cos h - 1}{h} = 0$$

				$$f'(x) = (0)\cos x - (1) \sin x = - \sin x$$
			% Q59
			\item The equation $$y'' + y' - 2y = \sin x$$ is called a differential equation because
				it involves an unknown function $y$ and its derivatives $y'$ and $y''$. Find constants
				$A$ and $B$ such that the function $y = A \sin x + B \cos x$ satisfies this
				equation.

				If $y = A \sin x + B \cos x$

				then

				$y' = A \cos x - B \sin x$ and $y'' = - A\sin x - B \cos x$

				If we substitute these equations into our original equation

				$$(-A \sin x - B \cos x) + (A \cos x - B \sin x) - 2(A \sin x + B \cos x = \sin x$$

				$$A \cos x - A \sin x - B \cos x - B \sin x - 2A \sin x + 2B \cos x = \sin x$$

				Gathering together trigonometric terms...

				$$(A - B + 2B) \cos x +(-A - B - 2A) \sin x = \sin x$$

				$$(A+B) \cos x + (-3A - B) \sin x = \sin x$$

				For this to be true $A+B=0$ and $-3A - B = 1$

				$$A = -B$$

				Substituting into the other equation

				$$3B - B = 1$$

				$$B = \frac{1}{2}$$

				Finding $A$...

				$$A + \frac{1}{2} = 0$$

				$$A = - \frac{1}{2}$$

			% Q60
			\item Find constants $A$,$B$ and $C$ such that the function $y = Ax^2 + Bx + C$ satisfies the
				differential equation $y'' + y' - 2y = x^2$

				First we need to find the first and second derivatives of $y$.

				$$y' = 2Ax + B$$

				$$y'' = 2A$$

				Now we can substitute these values in

				$$2A + 2Ax + B - 2(Ax^2 + Bx + C) = x^2$$

				$$2A + 2Ax + B - 2Ax^2 - 2Bx + 2C = x^2$$

				$$2A(-x^2 + x + 1) + B(1-2x) - 2C = x^2$$

				Because the only term that contains an $x^2$ factory $A$ must be $- \frac{1}{2}$

				If $A = -\frac{1}{2}$

				$$x^2 - x - 1 + B(1-2x) - 2C = x^2$$

				We must cancel out the $x$ term so $B$ must be $-\frac{1}{2}$

				If $B = -\frac{1}{2}$

				$$x^2 - x - 1 + - \frac{1}{2} + x - 2C = x^2$$

				$$x^2 - \frac{3}{2} - 2C = x^2$$

				We must cancel out the constant so $C$ must be $-\frac{3}{4}$

			% TODO: Q61

			% TODO: Q62
			\item
			\begin{enumerate}
				\item Find equations of both lines through the point (2, -3) that
					are tangent to the parabola $y = x^2 + x$.
			\end{enumerate}	

			% Q63
			\item For what values of \emph{a} and \emph{b} is the line $2x + y = b$
				tangent to the parabola $y = ax^2$ when $x = 2$?

				First lets re-arrange the linear equation.

				$$2x + y = b$$

				Is equal to...

				$$y = -2x + b$$

				From this we know the gradient of the line is -2.

				The derivative of the parabola is...

				$$y' = 2ax$$

				The line is a tangent at $x = 2$ so

				$$2a(2) = -2$$

				$$4a = -2$$

				$$a = \frac{-2}{4} = -\frac{1}{2}$$

				To find $b$ we must find the point where the line meets the
				parabola.

				$$y = -\frac{1}{2}(2)^2 = -2$$

				So...

				$$-2 = -2(2) + b$$

				$$b = 2$$

			% Q64
			\item Find a parabola with equation $y = ax^2 + bx + c$ that has slope 4
				at $x = 1$, slope -8 at $x = -1$, and passes through point $(1,15)$.

				Firstly the derivative of the quadratic is...

				$$y' = 2ax + b$$

				Therefore

				$$2a + b = 4$$
				$$-2a + b = -8$$

				From the first equation

				$$2a = 4 - b$$
				$$a = \frac{4-b}{2}$$

				Plugging this into the second equation

				$$-2(\frac{4-b}{2}) + b = -8$$
				$$-(4-b) + b = -8$$
				$$b-4+b = -8$$
				$$2b - 4 = - 8$$
				$$2b = -4$$
				$$b = -2$$

				If we plug the value for $b$ into the first equation

				$$2a -2 = 4$$
				$$2a = 6$$
				$$a = 3$$

			% TODO: Q65
			\item Find a cubic function $y = ax^3 + bx^2 + cx + d$ whose graph has horizontal
				tangents at the points $(-2,6)$ and $(2,0)$.

				First we find the derivative of the equation

				$$y' = 3ax^2 + 2bx + c$$

				We know that...

				$$3a(-2)^2 + 2b(-2) + c = 0$$
				$$12a -4b + c = 0$$

				and...

				$$3a(2)^2 + 2b(2) + c = 0$$
				$$12a + 4b + c = 0$$

			% Q67
			\item Find the parabola with equation $y = ax^2 + bx$ whose tangent line at $(1,1)$
				has equation $y = 3x - 2$

				So the gradient at $x = 1$ is 3

				The derivative of the parabola is...

				$$y' = 2ax + b$$

				So
	
				$$2a(1) + b = 3$$

				$$2a + b = 3$$

				From the equation of the parabola

				$$a(1)^2 + b(1) = 1$$
				$$a + b = 1$$

				$$a = 1 - b$$

				Plugging in the value for $a$ into the other formula

				$$2(1-b) + b = 3$$
				$$2-2b+b=3$$
				$$2-b = 3$$
				$$b=-1$$

				Plugging this value for $b$ back into the first equation

				$$a-1 = 1$$
				$$a = 2$$

			% Q68
 			\item A tangent line is drawn to the hyperbola $xy = c$ at a point $P$.
			\begin{enumerate}
				\item Show that the midpoint of the line segment cut from this
					tangent line by the coordinate axes is $P$.

					$$y = \frac{c}{x}$$

					Let's pick a point at $x = a$

					$$P = (a, \frac{c}{a})$$

					The equation of the line is...

					$$(y - \frac{c}{a}) = - \frac{c}{a^2}(x - a)$$

					$$y - \frac{c}{a} = - \frac{cx}{a^2} + \frac{c}{a}$$

					$$y = - \frac{c}{a^2}x + \frac{2c}{a}$$

					$$y = - \frac{c}{a^2}x + \frac{2c}{a}$$

					So the $y$ intercept is $\frac{2c}{a}$

					The $x$ intercept is 

					$$- \frac{c}{a^2}x + \frac{2c}{a} = 0$$
					$$\frac{c}{a^2}x = \frac{2c}{a}$$

					$$x = \frac{a^2}{c} \times \frac{2c}{a}$$
					$$x = 2a$$

					Because the $y$ intercept is $2P_{y}$ and the $x$ intercept
					is $2P_{x}$, $P$ is the midpoint of the line.

				\item Show that the triangle formed by the tangent line and the coordinate
					axes always has the same area, no matter where $P$ is located on the
					hyperbola.

					The equation for the area of a right angled triangle is $A = \frac{1}{2}mn$

					From part (a) we know that the height of the triangle is $\frac{2c}{a}$ the
					width of the triangle is $\frac{2c}{a}$

					$$\frac{1}{2}(2a \times \frac{2c}{a}) = c$$

					As $c$ is a constant, the area is constant.
			\end{enumerate}

			% Q69
			\item Evaluate $\lim \limits _{x \to 1} \frac{x^{1000} - 1}{x - 1}$

				This has a similar form to the definition of the derivative

				$$f'(x) = \lim \limits _{x \to a} \frac{f(x) - f(a)}{x-a}$$
				
				If we take $f(x) = x^{1000}$ then

				$$\frac{d}{dx} x^{1000} = \lim \limits _{x \to 1} \frac{x^{1000} - 1}{x-1}$$

				So the limit is equal to $$\frac{d}{dx} x^{1000} = 1000x^{999}$$

				$$1000(1)^{9999} = 1000$$
	\end{enumerate}	
\end{document}
